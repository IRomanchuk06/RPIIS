\documentclass{article}

\newcommand{\RomanNumeralCaps}[1]
    {\MakeUppercase{\romannumeral #1}}


\usepackage{indentfirst}
\usepackage{graphicx} % Required for inserting images
\usepackage[left=19mm, top=22mm, right=19mm, bottom=25mm]{geometry}
\usepackage{enumitem}
\usepackage{multicol}
\usepackage{parskip}
\usepackage{amsmath}
\setcounter{page}{102}
\usepackage{setspace}
\setlist[itemize]{noitemsep, topsep=0pt, parsep=0pt, partopsep=0pt}



\begin{document}
\begin{multicols}{2}



$\Rightarrow$ \hspace{0.5cm}  \textit{second domain}*: \newline
\hspace*{2.8em} \textit{metric finite semantic space}

\textbf{\textit{metric finite semantic superspace'}} \newline
$\Rightarrow$ \hspace{0.5cm} \textit{ first domain}*: \newline
\hspace*{2.8em} \textit{inclusion of metric finite 
semantic spaces}* \newline
$\Rightarrow$ \hspace{0.5cm}  \textit{second domain}*:
 \newline
\hspace*{2.8em} \textit{metric finite semantic space}


\hspace{0.27cm} A metric finite syntactic space can be constructed by
[4] according to the string processing model and metric
definitions given in [5].




\begin{tabbing}
 \textbf{\textit{pseudometric}} \\ $\Rightarrow$ \hspace{0.5cm} \= \textit{explanation}*: \\ \>[A pseudometric is a function of two arguments \\ \> that takes values on a (linearly) ordered group \\ \> support, is non-negative, symmetric, and satisfies \\ \> the triangle inequality.]
\end{tabbing}



\begin{tabbing}
\textbf{\textit{psudometric space}} \\ $\Rightarrow$ \hspace{0.5cm} \= \textit{explanation}*: \\ \>[A pseudometric space is a set with a function \\ \> of two arguments defined on it, which is a \\ \> pseudometric [16] taking values on the ordered \\ \> the support of the group.]
\end{tabbing}



\begin{tabbing}
\textbf{\textit{pseudometric finite semantic space}} \\ $\Rightarrow$ \hspace{0.5cm} \= \textit{explanation}*: \\ \>[A pseudometric finite semantic space of the\\ \> SC-code is a pseudometric space with a finite \\ \> support whose elements are designations (sc- \\ \> elements), and the value of the pseudometric \\ \> cannot be determined through the incidence \\ \> relations of elements without taking into account \\ \> their semantic type.]

\end{tabbing}

\hspace{0.27cm}Some models of more complex structures that take
into account non-factors [17] associated with space-time
have been successfully proposed in [4]. The proposed
models rely on a representation capable of expressing the
semantics of variable notation and operational semantics
by extended means of the alphabet. To build such models,
in addition to the extended alphabet tools, it is proposed to
rely on models that describe the processes of integration
and formation of knowledge [18], on knowledge specification tools [3], [4], focused on consideration of finite
structures, which allow proceeding with consideration of
complex metric relationships within the semantic space
meta-model (see Fig. 2).

\hspace{0.27cm}The possibility of considering the metric in the semantic
space allows speaking about the semantic metric, which,
along with activity, scaling, interpretability, and the
presence of a complex structure and coherence, is a
hallmark of knowledge.


\includegraphics{ctr102.jpg}


\begin{center}
\scriptsize Figure 2. Models providing integration.
\end{center}

\begin{tabbing}
\textbf{\textit{semantic metric}} \\ $:=$ \hspace{0.5cm} [semantic similarity] \\   $\Rightarrow$ \hspace{0.5cm} \= \textit{explanation}*: \\ \>[Semantic metric is a metric defined on signs and\\ \> quantitatively expressing the proximity of their \\ \> meanings.]
\end{tabbing}

\begin{spacing}{0}
\hspace*{0,27cm}In addition to factual knowledge (facts), rules are used\!
in knowledge bases.Within logical models of knowledge\!
processing, rules are represented as logical formulas. Thus,\!
the transition to the integration of such types of knowledge\!
as (logical) rules allows talking about the integration of\!
knowledge processing models (problem-solving models).\!
\end{spacing}



\begin{center}
\RomanNumeralCaps{3}. \RomanNumeralCaps{1}\scriptsize NTEGRATION OF LOGICAL PROCESSING MODELS \\
AS PROBLEM SOLVING MODELS
\end{center}


\hspace{0.27cm}In order to solve the problem of integrating problem-solving models, the concept of a formal model for knowledge processing is proposed, which is a development for
the concept of a formal model of information processing.
The approach is used in the works of V. Kuzmitsky
[19] and A. Kalinichenko [20]. A meta-model for the
integration of formal models of knowledge processing is
proposed.\\
\vspace{0.5pt}
\hspace{0.27cm}The integration of knowledge processing models boils\\down to the following steps:
\begin{itemize}
    \item For each state of the integrating model, its one-to-one $\left(i\right)$ representation is constructed in the model of
the unified semantic representation of knowledge.
     \item Next, a mapping $\pi$ of this representation to a set of sc-texts immersed in a metric semantic subspace is constructed, and a one-to-one mapping i of operations $i\left(\rho\right)$ of this model to operations $\rho$ on sc-texts from this set is constructed, so that:\newline
\end{itemize}


\centerline{$i \circ i^{-1} \subseteq I = \{\langle x, x \rangle \,|\, \exists y \, \langle x, y \rangle \in i^{-1} \cup i\}$}
\vspace{19pt}
\centerline{$i^{-1} \circ i \subseteq I$}



\centerline{$\forall \rho \left(i^{-1} \circ i \circ \rho \circ i \circ i^{-1} = \rho\right)$} 
\vspace{19pt}
\centerline{$\forall \rho \exists i(\rho) \left(\pi^{-1} \circ i^{-1} \circ \rho \subseteq i \circ \pi \circ i(\rho)\right)$ } 
\vspace{19pt}
\centerline{$\forall \rho \exists i(\rho) \left(i \circ \pi \circ i(\rho) \subseteq \rho \circ i \circ \pi\right)$}
\vspace{19pt}
\centerline{$\forall \rho \exists i(\rho) \left(\rho = i \circ \pi \circ i(\rho) \circ \pi^{-1} \circ i^{-1}\right)$}
\vspace{19pt}
\centerline{$\forall \rho \exists i(\rho) \left(i(\rho) = \pi^{-1} \circ i^{-1} \circ \rho \circ i \circ \pi\right)$}
 
\vspace{0.5pt}
\begin{itemize}
    \item Syntactic relations are distinguished on the elements
of sc-texts.\!
    \item Interpretation functions are built on the states of the
original model (in projective semantics) or on their
representation in sc-texts (in reflexive semantics).\!
    \item The metric is set in accordance with the metric of
the metric semantic subspace.\!
    \item In addition to the specified requirements, additional
requirements $\tau$ and $\sigma$ can be specified in accordance
with a given scale on the set of states of the
integrating information processing model: bijection
(trivial order), out-degree, in-degree, etc.\!
\end{itemize}


\centerline{$\forall \rho \exists i(\rho) \left(\rho \circ \tau = i \circ \pi \circ i(\rho) \circ \pi^{-1} \circ i^{-1}\right)$}
\vspace{19pt}
\centerline{$\forall \rho \exists i(\rho) \left(i(\rho) \circ \sigma = \pi^{-1} \circ i^{-1} \circ \rho \circ i \circ \pi\right)$}

\vspace{0.5pt}
It should be noted that in the previous article [21], the
mapping requirements were considered to be quite strong
($\tau$ = \textit{I} and $\sigma$ = \textit{I} ):


\centerline{$\forall \rho \exists i (\rho) (\rho \subseteq i \circ \pi \circ i(\rho) \circ \pi^{-1} \circ i^{-1})$}
\vspace{19pt}
\centerline{$\forall \rho \exists i(\rho) (i(\rho) \subseteq \pi^{-1} \circ i^{-1} \circ \rho \circ i \circ \pi)$}


\hspace{0.27cm}The current text contains proposal for weakening these
requirements. Other additional requirements (including
the quantitative properties of the information) may also
be taken into account.\newline
\vspace{0.5pt}
\hspace{0.27cm}Let us consider some examples (Fig. 3–25).\newline
\vspace{0.5pt}
\hspace{0.25cm}From the point of view of topological properties, for
each state of the model, there is its topological closure
with respect to the set of operations. Moreover, these
topological properties are preserved during integration.
Thus, integration is a continuous mapping. However, for
classical logical models of information processing, it is
known that the closure with respect to deducibility is not
topological closure (not additive):

\begin{center}
$[S] \cup [T] \neq [S \cup T]$ 
\end{center}

\hspace{0.27cm}The seeming contradiction can be resolved if we notice
that in the first case, the elements of the closure are
\begin{center}
\includegraphics[width=0.4\textwidth, keepaspectratio]{103_1.JPG}
\end{center}

\scriptsize Figure 3. The reconvergent integration of non-deterministic knowledge
processing operation as non-deterministic one ((green) vertical lines)
with the divergent integration of deterministic knowledge processing
operation as non-deterministic operation one ((red) horizontal lines).
Rhombuses are subtext (substates). Triangles and the bottom blue disk
and square are states of integrating models. Others disks and squares
are the states (text) of the integrated model.

\begin{center}
\includegraphics[width=0.4\textwidth, keepaspectratio]{103_2.JPG}
\end{center}

\scriptsize Figure 4. The convergent integration of deterministic knowledge
processing operation as deterministic one ((green) vertical lines) with the
divergent integration of deterministic knowledge processing operation
as non-deterministic operation one ((red) horizontal lines). Rhombuses
are subtext (substates). Triangles and the bottom blue disk and square
are states of integrating models. Others disks and squares are the states
(text) of the integrated model.

\begin{center}
\includegraphics[width=0.4\textwidth, keepaspectratio]{103_3.JPG}
\end{center}

\scriptsize Figure 5. The reconvergent integration of non-deterministic knowledge
processing operation as deterministic one ((green) vertical lines) with the
divergent integration of deterministic knowledge processing operation
as deterministic operation one ((red) horizontal lines). Rhombuses are
subtext (substates). Triangles and the bottom blue disk and square are
states of integrating models. Others disks and squares are the states
(text) of the integrated model.


\begin{center}
\includegraphics[width=0.43\textwidth, keepaspectratio]{104_1.JPG} 
\end{center}
\vspace{0.5pt}
\scriptsize Figure 6. The convergent integration of non-deterministic knowledge
processing operation as deterministic one ((green) vertical lines) with the
divergent integration of deterministic knowledge processing operation
as deterministic operation one ((red) horizontal lines). Rhombuses are
subtext (substates). Triangles and the bottom blue disk and square are
states of integrating models. Others disks and squares are the states
(text) of the integrated model.

\begin{center}
\includegraphics[width=0.43\textwidth, keepaspectratio]{104_3.JPG}
\end{center}
\vspace{0.5pt}
\scriptsize Figure 7. The convergent integration of ndeterministic knowledge
processing operation as deterministic one ((green) vertical lines) with the
divergent integration of deterministic knowledge processing operation
as deterministic operation one ((red) horizontal lines). Rhombuses are
subtext (substates). Triangles and the bottom blue disk and square are
states of integrating models. Others disks and squares are the states
(text) of the integrated model.

\begin{center}
\includegraphics[width=0.43\textwidth, keepaspectratio]{104_5.JPG}
\end{center}
\vspace{0.5pt}
\scriptsize Figure 8. The reconvergent integration of non-deterministic knowledge
processing operation as deterministic one ((green) vertical lines) with the
divergent integration of deterministic knowledge processing operation
as deterministic operation one ((red) horizontal lines). Rhombuses are
subtext (substates). Triangles and the bottom blue disk and square are
states of integrating models. Others disks and squares are the states
(text) of the integrated model.

\vspace{20pt}

\begin{center}
\includegraphics[width=0.42\textwidth, keepaspectratio]{104_2.JPG} 
\end{center}
\vspace{0.5pt}
\scriptsize Figure 9. The asymmetrical reconvergent integration of non-deterministic knowledge processing operation as deterministic one
(green lines) with the divergent integration of deterministic knowledge
processing operation as deterministic operation one (red lines).

\begin{center}
\includegraphics[width=0.45\textwidth, keepaspectratio]{104_4.JPG}
\end{center}
\vspace{0.5pt}
\scriptsize Figure 10. The reconvergent integration of deterministic knowledge
processing operation as deterministic one (green lines) with the
divergent integration of deterministic knowledge processing operation
as deterministic operation one (red lines).
\vspace{0.5cm}

\begin{center}
\includegraphics[width=0.45\textwidth, keepaspectratio]{104_6.JPG}
\end{center}
\vspace{0.5pt}
\scriptsize Figure 11. The reconvergent integration of deterministic knowledge
processing operation as deterministic one (green lines) with the asymmetrical divergent integration of deterministic knowledge processing
operation as non-deterministic operation one (red lines).

\end{multicols}
\end{document}
