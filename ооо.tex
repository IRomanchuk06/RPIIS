\documentclass[a4paper,10pt,twocolumn]{article}

\usepackage[utf8]{inputenc}

\usepackage[pdftex]{graphicx, caption}

\usepackage{float}

\usepackage{enumitem}

\usepackage{geometry}

 \graphicspath{{D:/man/}}
 \setlist{nolistsep}
 \geometry{top = 20mm}
 \geometry{left = 25mm}
 \geometry{right = 25mm}
 \geometry{bottom = 25mm}
\begin{document}
\twocolumn
\frenchspacing
\setcounter{page}{264}
\begin{enumerate}
	\item[]{applications. This leads to increased use, reuse, and
	maintainability of the information systems.}
	\item[3)] {Easy to relate to object-oriented programming
	paradigm or database design: Those who are familiar
	with the object-oriented programming paradigm
	or database design can easily relate the ontological
	representation of the domain entities to classes or
	database schemas. The classes are generic representations
	of the entities that encapsulate properties
	and behaviors.}
	\item[4)] {Semantic model of data: One major advantage of
	using a domain ontology is its ability to define
	a semantic model of the data combined with the
	associated domain knowledge. Ontologies can also
	be used to define links between different types of
	semantic knowledge. Thus, ontologies can be used
	in formulating some data searching strategies.}
\end{enumerate}

Overall, ontological modeling in intellectual information
systems provides a powerful tool for representing
knowledge in a way that is intuitive and easy to understand,
both for humans and machines.

Ontology editors are applications designed to assist
in the creation or manipulation of ontologies. They use
one or more ontology languages to create, visualize, and
manipulate ontologies. These editors have features such
as visual navigation possibilities within the knowledge
model, inference engines and information extraction,
support for modules, the import and export of foreign
knowledge representation languages for ontology matching,
and support of meta-ontologies such as OWL-S
and Dublin Core. Additionally, there are various tools
used for ontological modeling of intelligent information
systems. Let’s consider the most popular of ontology
editors. 
\begin{center}
\large{II. O}\small{NTOLOGY} \large{E}\small{DITORS}
\end{center}

To create ontologies, specialized software products are
widely used - ontology editors. Let’s look at the most
popular of them. We will compare the characteristics of
the following software products.

There are several popular ontology editors that can
help authors create their ontologies. Some of them provide
additional functions and plugins that can be useful
when working with ontologies. Some of the most popular
ontology editors are listed below:
\begin{enumerate} 
\item[1)] {NeOn Toolkit is an ontology editor with many
plugins available, especially suitable for large
scale projects (eg multi-module ontologies, multilingual,
ontology integration, etc.).}
\item[2)]{Neologism is an online dictionary editor and publishing
platform.}
\item[3)]{Vitro is an integrated editor for ontologies and
semantic web applications.}
\item[4)]{Knoodl is a community-oriented ontology and
knowledge base editor.}
\item[5)]{Fluent Editor is a comprehensive tool for editing and manipulating complex ontologies
	that uses controlled natural language. It provides an instant
	natural language representation of OWL/SWRL,
	which improves performance and makes editable
	ontologies easier to read and understand.}
\end{enumerate}

Some other popular ontology editors that can be mentioned
include:
\begin{enumerate}
	\item[1)] {Eddy}
	\item[2)] {OntoME}
	\item[3)] {OntoStudio (formerly known as OntoEdit)}
	\item[4)] {Protégé}
\end{enumerate}

Let’s take a closer look at some of the editors from
the list above.

\textbf{NeOn Toolkit}.

NeOn Toolkit is an ontology editor that allows users
to create and edit ontologies. It offers a variety of tools
and features to support the development of ontologies,
including visual modeling, ontology debugging, and ontology
testing. Some of the advantages of using NeOn
Toolkit are:
\begin{itemize}
	\item{User-friendly interface: NeOn Toolkit has an intuitive
	interface that makes it easy to create and edit}
	ontologies.
	\item{Visual modeling: The tool offers a visual modeling
	environment that allows users to create and edit
	ontologies using graphical representations.}
	\item{Collaboration: NeOn Toolkit supports collaboration
	between users, making it easy to work on ontologies
	as a team.}
	\item{Ontology debugging: The tool provides debugging
	capabilities that help users identify and fix errors in
	their ontologies.}
	\item{Ontology testing: NeOn Toolkit includes a testing
	framework that allows users to test their ontologies
	to ensure they are working correctly.}
	\item{Ontology testing: NeOn Toolkit includes a testing
	framework that allows users to test their ontologies
	to ensure they are working correctly.}
\end{itemize}

Despite its advantages, NeOn Toolkit also has some
drawbacks. For example:
\begin{itemize}
\item{Steep learning curve: While NeOn Toolkit is userfriendly,
it can still be challenging to learn for users
who are new to ontology development.}
\item{Limited documentation: Some users have reported
that the documentation for NeOn Toolkit is limited,
making it difficult to troubleshoot issues or learn
about advanced features.}
\item{Limited support: NeOn Toolkit is an open-source
tool, which means that support is limited to user
forums and community resources.}
\end{itemize}
NeOn Toolkit is primarily used in the field of semantic
web development, specifically in the development of
ontologies. It is commonly used in research and academic
\begin{figure}[H]
	\centering
	\includegraphics[width=0.5\textwidth]{ikl.png}
	\caption*{\small{Figure 1. Fluent Editor.First start.}}
\end{figure} 

settings for ontology development, but it can also be used
in industry settings for knowledge management and data
integration.

\textbf{Fluent Editor.}

Fluent Editor is an ontology editor developed by the
Polish company Cognitum. It is used for editing complex
ontologies created using a controlled natural language
(CNL). The editor allows users to create ontologies by
entering phrases in natural language.

Some of the advantages of Fluent Editor are:
\begin{itemize}
	\item{User-friendly interface: Fluent Editor has a simple
	and intuitive graphical user interface that allows
	even non-experts to create and edit ontologies.}
	\item{Support for natural language: Fluent Editor supports
	natural language input, which makes it easier to
	create ontologies for non-experts.}
	\item{Advanced features: Fluent Editor includes advanced
	features such as automated reasoning, which can
	help identify inconsistencies and errors in ontologies.}
\end{itemize}

However, there are also some limitations to Fluent
Editor:
\begin{itemize}
	\item{Limited documentation: Fluent Editor has limited
	documentation available, which can make it difficult
	for users to learn how to use the editor}.
	\item{Limited support: Cognitum provides limited support
	for Fluent Editor, which can make it difficult for
	users to get help if they encounter problems.}
	\item{Limited customization: Fluent Editor has limited
	support for customization, which can make it difficult
	for users to tailor the editor to their specific
	needs. }
\end{itemize}

Fluent Editor is primarily used for creating and editing
complex ontologies [0][2][4]. It can be used in various
domains, including healthcare, finance, and engineering,
where ontologies are used to represent and organize
knowledge.

\textbf{Eddy.}

Eddy is an ontology editor that allows users to create
and edit ontologies. Ontologies are used to define concepts
and relationships in a specific domain of knowledge.
Eddy is a web-based application that can be used
by anyone with an internet connection. It has several
advantages, including:
\begin{itemize}
	\item{User-friendly interface: Eddy has a simple and intuitive
	interface that makes it easy for users to create}
	and edit ontologies.
	\item{Collaboration: Eddy allows multiple users to work
	on the same ontology simultaneously, making it
	ideal for collaborative projects.}
	\item{Integration: Eddy can be integrated with other tools
	and applications, making it a versatile tool for
	ontology development. }
\end{itemize}

However, Eddy also has some limitations:
\begin{itemize}
	\item{Limited functionality: Eddy’s functionality is limited
	compared to other ontology editors.}
	\item{Learning curve: Although Eddy is user-friendly,
	there is still a learning curve for users who are new
	to ontology editing.}
	\item{Lack of support: Eddy does not have a large community
	of users, which means that there is limited
	support available for users who encounter problems.}
\end{itemize}

Eddy is primarily used in the field of knowledge engineering,
which involves the creation and management
of knowledge-based systems. It can be used in a variety
of domains, including:

\begin{itemize}
	\item{Healthcare: Eddy can be used to create ontologies
	for medical terminology and patient data.}
	\item{E-commerce: Eddy can be used to create ontologies
	for product catalogs and online marketplaces.}
	\item{Education: Eddy can be used to create ontologies for
	educational resources and curriculum development.}
\end{itemize}

Overall, Eddy is a useful tool for ontology editing, but
it may not be the best option for all users depending on
their specific needs and requirements.

\textbf{OntoME.}

OntoME is an ontology editor that enables users
to create, edit and publish ontologies. Here are some
advantages and disadvantages of ontoME:
Advantages:
\begin{itemize}
	\item{OntoME provides a user-friendly interface for creating
	and editing ontologies, allowing users to create
	ontologies without programming knowledge.}
	\item{It supports multiple ontology formats, such as OWL,
	RDF, and RDFS.}
	\item{It offers a range of features, such as the ability to
	import and export ontologies, search for terms, and
	visualize ontologies.}
	\item{OntoME allows for collaboration and sharing of
	ontologies through a web-based interface.}
	\item{It provides support for versioning and change management
	of ontologies.}
\end{itemize}

Disadvantages:
\begin{itemize}
	\item{OntoME may not be suitable for heavy-weight
	projects, such as multi-modular ontologies or ontology
	integration, as it does not offer advanced
	features for such projects.}
	\item{It may not be suitable for users with advanced
	programming knowledge, as it does not allow for
	direct editing of the ontology code.}
	\item{The web-based interface may have limitations in
	terms of performance and speed compared to
	desktop-based ontology editors.}
\end{itemize}

The scope of OntoME is to provide a user-friendly and
accessible ontology editor for users with varying levels
of ontology development expertise. It is suitable for
small to medium-sized ontology projects with moderate
complexity. For more complex projects, users may need
to consider other ontology editors that offer advanced
features and customization options.

\textbf{OntoStudio.}

OntoStudio is an ontology editor that provides a
professional environment for ontology development. It
supports W3C standards such as OWL, RDF, and RDFS,
and F-Logic for the logic-based processing of rules. OntoStudio
also comes with many connectors to databases,
documents, file systems, applications, and web services.

The modular design of OntoStudio enables users to
enrich it with self-developed modules and customize it
according to their personal needs. It has modeling tools
for ontologies and rules, as well as components for the
integration of heterogeneous data sources. OntoStudio is
available with a free evaluation license.

In a study that compared five ontology editors, including
OntoStudio, the main criterion for comparison was
the convenience for users. The study described the basic
features and structure of the editors, as well as their way
of use. OntoStudio was found to be a convenient tool for
users.

OntoStudio’s advantages are its support for various
ontology languages and its modular design that allows
customization. Its disadvantages are not mentioned in the
sources. OntoStudio’s scope is in the development and
operation of semantic applications that involve ontology
learning, reasoning, and text mining. It is also useful
for the storage and management of semantic data and
metadata.

Specific features of OntoStudio’s modeling tools for
ontologies and rules are not explicitly mentioned in
the sources. However, OntoStudio combines modeling
tools for ontologies and rules and is built on top of a
powerful internal ontology model that allows the user to
edit a hierarchy of concepts or classes. OntoStudio also
supports W3C standards such as OWL, RDF, and RDFS,
and F-Logic for the logic-based processing of rules.

The user-friendliness of OntoStudio for beginners in
ontology modeling is not directly mentioned in the
sources. However, a study that compared five ontology
editors, including OntoStudio, found that the main criterion
for comparison was the convenience for users. The
study described the basic features and structure of the
editors, as well as their way of use. OntoStudio was
found to be a convenient tool for users. It is worth noting
that OntoEdit is considered simpler than OntoStudio
but lacks some important features as apprenticeship on
ontologies grows.

After reviewing the editors discussed above, we came
to the conclusion that the Protégé editor would be the
most suitable for our purposes. Consideration of the
methodology for creating an ontology in this editor is
devoted to the next section of our work.
\begin{center}
	\large{III. O}\small{NTOLOGY} \large{E}\small{DITOR} \large{P}\small{ROTÉGÉ}
\end{center}
Protégé is an open-source ontology editor that allows
users to create, edit, and manipulate ontologies. An ontology
is a formal representation of a domain’s knowledge
that specifies a set of concepts and their relationships.
Creating an ontology involves several steps, including
defining the domain, identifying the key terms, and
creating the class hierarchy.

To create an ontology in Protégé, one can follow these
steps:
\begin{enumerate}
	\item[1)]{Define the domain: The first step is to determine
	the scope of the ontology, including the types
	of questions it should answer and the purpose it
	serves.}
	\item[2)]{Identify key terms: After defining the domain, the
	next step is to identify the key terms that will be
	used in the ontology. This can be done by analyzing
	literature or consulting with experts. These
	terms should be organized into a table, including
	their properties or characteristics.}
	\item[3)]{Create the class hierarchy: The next step is to create
	the class hierarchy, which involves identifying
	the most general concepts and gradually refining
	them into more specific ones. There are several
	approaches to creating a class hierarchy, including
	top-down, bottom-up, and combined approaches.}
	\item[4)]{Define properties and characteristics: Once the
	class hierarchy is established, the next step is to
	define the properties and characteristics of each
	class. These may include attributes such as weight,
	habitat, and population size.}
	\item[5)]{Add instances: Finally, instances of each class
	can be added to the ontology. These instances are
	specific examples of the classes in the ontology.}
	\item[6)]{In addition to creating an ontology, Protégé also
	allows for merging ontologies and performing operations
	on classes, such as defining equivalent or
	inverse classes and transitive properties. }
\end{enumerate}

The advantages of using Protégé to create an ontology
are:

 
\end{document}