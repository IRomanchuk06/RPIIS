\documentclass{article}
\usepackage[T2A]{fontenc}
\usepackage[utf8]{inputenc}
\usepackage[english,russian]{babel}
\usepackage{enumitem}
\usepackage{multicol}
\usepackage{ulem}
\usepackage{graphicx}
\usepackage{geometry}
\geometry{left=15mm, right=15mm, top=8mm, bottom=10mm, nohead, nofoot}
\setcounter{page}{232}
\begin{document}
\begin{multicols}{2}
\noindent the basic functions of the CAS. Originally published in
1988 and updated in Mathematica 5, “The Mathematica
Book by Stephen Wolfram”, Fifth Edition, 2003, is 1,488
pages long. A list of books by S. Wolfram’s books can
be viewed at at [25].
\par
\uline{Help system in Mathematica.} The Documentation
Center, Function Navigator, and Virtual Book are
part of the Help system. These modules provide
all necessary information to guide users through
the language and functionality of Mathematica.
Built-in documentation contains more than
\uline{150,000 representative and illustrative examples of
Wolfram code.} All documents are fully interactive; they
are Mathematica notebooks in which the user can try
out their own code and modify the examples directly in
the Help system.
\par
\textbf{The theses outlined above are important from
the position of developers of computer systems for
artificial intelligence to understand the current state
in closely related fields,} in particular because computer
algebra systems, which implement intelligent computations with the help of a computer, are also one of the (and
quite successfully developed) areas to adopt artificial
intelligence.
\begin{center}
	V. EXAMPLES OF INTEGRATING WM TOOLS INTO
			OSTIS APPLICATIONS
\end{center}
\textit{A. Wolfram Mathematica. Current state}
\par
Building on over thirty years of research, development,
and use around the world, Mathematica and Wolfram
are geared for the long term and especially successful in
computational mathematics The roughly 6,000 functions
(symbols) built into Wolfram allow the user to represent
and manipulate a huge variety of computational objects –
from special functions to graphics and geometric regions.
\par
In addition, the Wolfram knowledge base [26] and
its associated entity structure [27] allows to explain,
interpret, and formalize hundreds of specific “things”
(facts, situations, objects). For example: people, cities,
food, structures, planets, etc. appear as objects that can
be manipulated and counted.
\par
\textit{B. Wolfram knowledge base. Coverage areas}
\par
The growing Wolfram Data Repository (WDR), based
on Wolfram Alpha and the Wolfram Language, is now
the world’s largest repository of computable knowledge.
Covering thousands of fields, the WDR contains carefully selected expert knowledge obtained directly from
primary sources. It includes not only trillions of data elements, but also a huge number of algorithms that encapsulate methods and models from virtually every domain.
The Wolfram Knowledge Base is based on Wolfram’s
three decades of accumulated computable knowledge. All
data in the Wolfram KB can be used immediately for
Wolfram computations. Every millisecond of every day.
\par
\noindent the Wolfram Knowledge Base is updated with the latest
data.
\par
Major coverage areas of WDR [26] are shown in Fig. 1.
\par
\includegraphics[width=80mm, height=30mm]{Proceedings OSTIS-2023-99.jpg}
\begin{center}
{\tiny Figure 1. Coverage areas of WDR.}
\end{center}
\par
In [28] typical options for working with WDR in
Education are outlined, as well as examples of interaction
with Wikipedia.
\par
With extensive statistics on hundreds of thousands of
educational institutions around the world, Wolfram|Alpha
can calculate answers to complex questions about education. For example, you can query what academic degrees students receive at prestigious universities, average
enrollment figures by year for selected majors. In the
examples [28] illustrations of the response to the query
about the number of students in the Republic of Belarus,
quantitative indicators for the leading universities of BSU
and BSUIR are given. Ways to present knowledge and
access to it are described. It is noted that access to the
Wolfram knowledge base is deeply integrated in Wolfram
Language (WL). Free-form linguistics makes it easy to
identify many millions of entities and many thousands of
properties, and automatically generates accurate Wolfram
Language representations suitable for extensive further
computation. WL also supports custom entity stores that
allow you to perform the same computations as the
built-in knowledge base and can be linked to external
relational databases.
\par
People interact with each other through speech and
text, and this is called natural language. Computers
understand people’s natural language using Natural Language Processing (NLP). NLP is the process of manipulating human speech and text with artificial intelligence
so that computers can understand them. In [28] the basic
NLP tools implemented in Wolfram Mathematica are
noted. In particular: Speech recognition; Voice assistants and chatbots. Auto-substitution and auto-prediction.
Email filtering. Sentiment analysis. Divertissements for
the target audience. Translation. Social media analytics. Recruitment (staffing). Text summary (abstracting).
Several representative examples with explanations of the
functions of WL groups Structural Text Manipulation,
Text Analysis, Natural Language Processing are mentioned.

\end{multicols}
\end{document}
	
