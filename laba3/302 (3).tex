\documentclass{article}
\usepackage[utf8]{inputenc}
\usepackage[left=17mm, top=17mm, right=17mm, bottom=9mm, nohead, nofoot]{geometry}
\usepackage{enumitem}
\usepackage{multicol}
\usepackage{ulem}
\usepackage{mathtools}
\usepackage{titlesec}
\setlist[itemize]{itemsep=0pt, parsep=0pt, partopsep=0pt, topsep=1.4pt}
\setcounter{page}{302}
\date{}
\begin{document}
\begin{multicols}{2}

    \begin{description}[leftmargin=!, labelwidth=1cm, itemsep=-1.5mm]
        \item[$\supset$] terrain object is located under another terrain object*
        \item[$\supset$] terrain object is located closer than another terrain object*
        \item[$\supset$] terrain object is located further than another terrain object*
    \end{description}

    \noindent \textit{\textbf{relation of the main directions of terrain objects}}
    \vspace{-0.3cm}
    \begin{description}[leftmargin=!, labelwidth=1cm, itemsep=-1.5mm]
        \item[$\subset$] oriented relation
        \item[$\coloneqq$] \textbf{[}allows determining which main direction one terrain object occupies in relation to another terrain object\textbf{]}
        \item[$\supset$] \textit{terrain object in relation to another terrain object occupies the main north direction*}
        \item[$\supset$] \textit{terrain object in relation to another terrain object occupies the main north-east direction*}
        \item[$\supset$] \textit{terrain object in relation to another terrain object occupies the main east direction*}
        \item[$\supset$] \textit{terrain object in relation to another terrain object occupies the main south-east direction*}
        \item[$\supset$] \textit{terrain object in relation to another terrain object occupies the main south direction*}
        \item[$\supset$] \textit{terrain object in relation to another terrain object occupies the main south-west direction*}
        \item[$\supset$] \textit{terrain object in relation to another terrain object occupies the main west direction*}
        \item[$\supset$] \textit{terrain object in relation to another terrain object occupies the main north-west direction*}
    \end{description}

    \noindent \textit{\textbf{metric spatial relation}}
    \vspace{-0.3cm}
    \begin{description}[leftmargin=!, labelwidth=1cm, itemsep=-1.5mm]
        \item[$\coloneqq$] \textbf{[}characterizes information about the distance between terrain objects\textbf{]}
        \item[$\Rightarrow$] \textit{measurement}*: \\ \textit{kilometer}
        \item[$\Rightarrow$] \textit{measurement}*: \\ \textit{meter}
        \item[$\supset$] \textit{scale metric spatial relation}
    \end{description}

        \noindent \textit{\textbf{metric spatial relation}}
    \vspace{-0.3cm}
    \begin{description}[leftmargin=!, labelwidth=1cm, itemsep=-1.5mm]
        \item[$\coloneqq$] \textbf{[}coordinate system used to determine the location of objects on the Earth\textbf{]}
        \item[$\ni$] example$'$: \\ WGS84
        \vspace{-0.3cm}
        \begin{description}[leftmargin=!, labelwidth=1cm, itemsep=-1.5mm]
            \item[$\coloneqq$] \textbf{[}The world system of geodetic parameters of the Earth, 1984, which includes a system of geocentric coordinates, and unlike local systems, it is a single system for the entire planet\textbf{]}
        \end{description}
        \item[$\ni$] example$'$: \\ CK-95
    \end{description}

    \begin{center}
        V. FORMALIZATION OF TOPOLOGICAL SPATIAL SEMANTIC RELATIONS IN GEOINFORMATION SYSTEMS
    \end{center}    

    Between instances of terrain objects, it is possible to establish topological spatial relations: \\
    \columnbreak
    
    \noindent \textit{\textbf{topological spatial relation}}
    \vspace{-0.3cm}
    \begin{description}[leftmargin=!, labelwidth=1cm, itemsep=-1.5mm]
        \item[$\coloneqq$] \textbf{[}spatial relation class, defined over terrain objects that are in relation of connectivity and adjacency between terrain objects\textbf{]}
        \item[$\ni$] \textit{inclusion}*
        \vspace{-0.3cm}
        \begin{description}[leftmargin=!, labelwidth=1cm, itemsep=-1.5mm]
            \item[$\supset$] \textit{inclusion of a point terrain object in an area terrain object*}
            \item[$\supset$] \textit{inclusion of a linear (multilinear) terrain object in an area terrain object*}
            \item[$\supset$] \textit{inclusion of an area terrain object in an area terrain object*}
        \end{description}
        \item[$\ni$] \textit{border*}*
        \item[$\ni$] \textit{intersection}*
        \vspace{-0.3cm}
        \begin{description}[leftmargin=!, labelwidth=1cm, itemsep=-1.5mm]
            \item[$\supset$] intersection of two linear (multilinear) terrain objects*
            \item[$\supset$] intersection of linear (multilinear) and area terrain objects*
        \end{description}
        \item[$\supset$] \textit{adjacency*}
    \end{description}

    The “inclusion*” relation will be set between \textit{area} and \textit{linear}, \textit{area} and \textit{point}, \textit{area terrain objects}. The “intersection*” relation will be set between \textit{linear} and \textit{area} and \textit{linear terrain objects}. The “border*” relation will be established between \textit{area terrain objects}. The “adjacency*” relation is established between \textit{linear terrain objects}. For all \textit{cartographic relations}, there are structures for storing them.

    \begin{center}
        VI. SUBJECT DOMAIN AND ONTOLOGY OF TERRAIN OBJECTS
    \end{center}
    
    For the purpose of \textit{integration of subject domains} with spatial components of \textit{geoinformation systems}, respectively increasing \textit{interoperability} of these systems, a \textit{hybrid knowledge model} is proposed. By this model we will understand a \textit{stratified model of the information space of terrain objects} described in the work [11]. \\

    \noindent \textit{\textbf{terrain object}}
    \vspace{-0.3cm}
    \begin{description}[labelwidth=0.5cm, itemsep=-1mm]
        \item[$\Rightarrow$] \textit{subdiving*}: \vspace{0mm} \\
        \textbf{\textit{Typology of terrain objects by topic}}
        \vspace{-0.3cm}\\
        \item[$=$]
        \begin{description}
            \begin{itemize}
                \textbf{\{} \hspace{-2.3mm} $\bullet$ \hspace{4.7mm} water terrain object (facility) \\
                            $\bullet$ \hspace{0.5cm} populated terrain object \\
                            $\bullet$ \hspace{1.5mm} industrial (agricultural or socio-cultural) terrain object \\
                            $\bullet$ \hspace{0.5cm} road network (facility) \\
                            $\bullet$ \hspace{0.5cm} vegetation cover (soil) \par
                \hspace{-3mm} \textbf{\}} \\
            \end{itemize}
        \end{description}
    \end{description}
        \vspace{-4mm} \par
    The basis for building the ontological model of \textit{terrain objects} is grounded on the classifier of topographic information displayed on topographic maps and city plans developed and currently functioning in the \textit{Republic of Belarus} [12]. In accordance with this circumstance, the objects of classification are the \textit{terrain objects} to which the map objects correspond, as well as the signs (characteristics) of these objects. For this purpose, in the ontological model, terrain objects are divided by

\end{multicols}
\end{document}
