\documentclass{article}
\usepackage[utf8]{inputenc}
\usepackage[left=17mm, top=17mm, right=17mm, bottom=25mm, nohead]{geometry}
\usepackage{enumitem}
\usepackage{multicol}
\usepackage{ulem}
\usepackage{mathtools}
\setlist[itemize]{itemsep=0pt, parsep=0pt, partopsep=0pt, topsep=1.4pt}
\setcounter{page}{306}
\date{}
\begin{document}
\begin{multicols}{2}
\begin{description}[leftmargin=!, labelwidth=0.7cm, itemsep=-1.5mm]
tems for detecting attacks at the network level and promptly responding to them. AI uses the accumulated statistics and knowledge base about threats;
\end{description}
\begin{itemize}
    \item SOAR (Security Orchestration and Automated Response) — systems that allow you to identify information security threats and automate incident response. In solutions of this type, unlike SIEM systems, AI helps not only to analyze, but also automatically respond appropriately to identified threats;
    \item Application Security — systems that allow you to identify threats to the security of application applications, manage the process of monitoring and eliminating such threats;
    \item Antifraud — platforms detect threats in business processes and fraudulent transactions in real time. AI is used to identify deviations from identified business processes in order to detect intrusions or process vulnerabilities and increase adaptability to changing business process logic and metrics.
\end {itemize}
\par
The paper [3] proposes a method for constructing a neuroimmune system for analyzing information security incidents that combines data collection and storage (compression) modules, an information security event analysis and correlation module, and a network attack detection subsystem based on convolutional neural networks. The use of machine learning technologies in information security creates bottlenecks and system vulnerabilities that can be exploited and has the following disadvantages [4]:
\begin{itemize}
    \item  data sets that must be formed from a significant
number of input samples, which requires a lot of
time and resources;
     \item requires a huge amount of resources, including
memory, data and computing power;
     \item frequent false positives that disrupt the operation and
generally reduce the effectiveness of such systems;
     \item organized attacks based on artificial intelligence
(semantic viruses).
\end{itemize}
\par 
\textbf{Organization of information security in intelligent
systems of a new generation}
\par
Let’s define the goals of ensuring the information security of new generation systems.
\par
From the monograph [5] the objectives of ensuring the information security of traditional intelligent systems are:
\begin{itemize}
    \item ensuring the confidentiality of information in accordance with the classification;
    \item ensuring the integrity of information at all stages of related processes (creation, processing, storage, transfer and destruction) in the provision of public services;
    \item ensuring timely availability of information in the
provision of public services;
    \item ensuring observability aimed at capturing any activity
of users and processes;
    \item ensuring the authenticity and impossibility of refusal of transactions and actions performed by participants in the provision of public services;
     \item accounting for all processes and events related to the input, processing, storage, provision and destruction of data.
\end{itemize}
\par Since intelligent systems of the new generation will interact with similar systems while understanding what the request is about, the goals of the provision will look different. The goals of ensuring the information security of new generation intelligent systems are:
\begin{itemize}
    \item ensuring the safety of the semantic compatibility of information;
    \item ensuring the availability of information at different
levels of the intellectual system;
    \item minimization of damage from events that pose a
threat to information security.
\end{itemize}
\par Currently, classical approaches and principles have
been developed to ensure the security of knowledge bases
(data), communication interfaces (information exchange)
between the components of intelligent systems, such as
encryption of transmitted data, filtering of unnecessary
(redundant) content, and data access control policy.
\par The information security system should be created on
the following principles:
\begin{itemize}
    \item the principle of equal strength - means ensuring
the protection of equipment, software and control
systems from all types of threats;
    \item the principle of continuity - provides for continuous
security of information resources of the system for
the continuous provision of public services;
    \item the principle of reasonable sufficiency - means the
application of such measures and means of protection
that are reasonable, rational and the costs of which
do not exceed the cost of the consequences of
information security violations;
    \item the principle of complexity - to ensure security in
all the variety of structural elements, threats and
channels of unauthorized access, all types and forms
of protection must be applied in full;
    \item the principle of comprehensive verification - is to
conduct special studies and inspections, special engineering analysis of equipment, verification studies of
software. Emergency messages and error parameters
should be continuously monitored, hardware and
software equipment should be constantly tested,
as well as software integrity control, both during
software loading and during operation;
    \item the principle of reliability - methods, means and
forms of protection must reliably block all penetration routes and possible channels of information
\end{itemize}

\begin{description}[leftmargin=!, labelwidth=0.7cm, itemsep=-1.5mm]
leakage; for this, duplication of means and security
measures is allowed;
\end{description}
\par
\vspace{-0.4cm} \begin{itemize}
    \item the principle of universality - security measures
should block the path of threats, regardless of the
place of their possible impact;
    \item the principle of planning – planning should be carried
out by developing detailed action plans to ensure
the information security of all components of the
system for the provision of public services;
    \item the principle of centralized management – within
a certain structure, the organized and functional
independence of the process of ensuring security in
the provision of public services should be ensured;
    \item the principle of purposefulness – it is necessary to
protect what must be protected in the interests of a
specific goal;
    \item the principle of activity - protective measures to
ensure the safety of the service delivery process
must be implemented with a sufficient degree of
persistence;
    \item the principle of service personnel qualification –
maintenance of equipment should be carried out
by employees who are trained not only in the
operation of equipment, but also in technical issues
of information security;
    \item the principle of responsibility - the responsibility
for ensuring information security must be clearly
established, transferred to the appropriate personnel
and approved by all participants as part of the
information security process.
\end{itemize}
 \begin{center} III.  THE PRINCIPLES UNDERLYING THE INFORMATION
SECURITY OF OSTIS SYSTEMS
\end{center}
The OSTIS ecosystem is a collective of interacting:
\begin{itemize}
    \item ostis-systems;
    \item  users of ostis systems (end users and developers);
    \item other computer systems that are not ostis-systems,
but are additional information resources or services
for them.
\end{itemize}
\par 
The core of OSTIS technology includes the following
components:
\begin{itemize}
    \item semantic knowledge base OSTIS, which can describe
any kind of knowledge, while it can be easily
supplemented with new types of knowledge;
    \item OSTIS problem solver based on multi-agent approach. This approach makes it easy to integrate
and combine any problem solving models;
    \item ostis-system interface, which is a subsystem with its
own knowledge base and problem solver.
\end{itemize}
\par
The presented architecture of the OSTIS Ecosystem
implements:
    \begin{itemize}
        \item  all knowledge bases are united into the Global
Knowledge Base, the quality of which (logicality,
correctness, integrity) is constantly checked by many
agents. All problems are described in a single knowledge base, and specialists are involved to
eliminate them, if necessary;
    \item each application associated with the OSTIS Ecosystem has access to the latest version of all major
OSTIS components, components are updated automatically;
    \item each owner of the OSTIS Ecosystem application can
share a part of their knowledge for a fee or for free.
\end{itemize}
    \par
It is important to note that information security is
closely related to the architecture of the built system: a
well-designed and well-managed system is more difficult
to hack. Therefore, it is very important to develop an
information security system at the stage of designing
the architecture and structure of a future next-generation
intelligent system.
    \par 
The OSTIS Ecosystem is a community where ostis
systems and users interact, where rules must be established and controlled. Illegal and destabilizing actions by
all members of the community should not be allowed.
The user cannot directly interact with other ostis systems,
but only through a personal agent. This agent stores all
personal data of the user and access to them should be
limited.
    \par 
In the OSTIS Ecosystem, all agents must be identified.
It should be noted that the personal user agent in the
Ecosystem solves the problem of identifying the user
himself.
    \par
    In the considered OSTIS Ecosystem, it is required to
organize information security at each of the levels of
interaction: data exchange, data access rights, authentication of Ecosystem clients, data encryption, obtaining
data from open sources, ensuring the reliability and
integrity of stored and transmitted data, monitoring the
violation of communications in knowledge base, tracking
vulnerabilities in the system.
\vspace{0.7cm}
\par
\noindent \textit{\textbf{threat in ostis-system}}
 \begin{description}[leftmargin=!, labelwidth=1cm, itemsep=-1.5mm]
   \item[$\supset$] \textit {threat. breach of confidentiality of information}
   \hspace{0.5cm} $\Rightarrow$  \hspace{0.5cm} \textit{explanation}:* 
   \par \vspace{-0.1cm} \hspace{1cm} \textbf{[}unauthorized access to read information\textbf{]}
    \item[$\supset$] \textit {threat. violation of the integrity of information} \\
   \vspace{0.1cm} \hspace{-0.23cm}  $\Rightarrow$  \hspace{0.5cm} \textit{explanation}:* 
     \par
      \begin{description}[leftmargin=!, labelwidth=1cm, itemsep=-1.5mm]
     \item[ ] \vspace{-0.4cm} \textbf{[}unauthorized or erroneous change, distortion or destruction of information, as well
as unauthorized impact on technical and
software information processing tools\textbf{]}
\end{description}
 \item[$\supset$] \vspace{-0.2cm} \textit {threat. accessibility violation} \\
   \vspace{0.1cm} \hspace{-0.23cm}  $\Rightarrow$  \hspace{0.5cm} \textit{explanation}:* 
   \par 
   \begin{description}[leftmargin=!, labelwidth=1cm, itemsep=-1.5mm]
   \item \vspace{-0.5cm} \textbf{[}blocking access to the system, its individual components, functions or information,
as well as the impossibility of obtaining
information in a timely manner (unacceptable delays in obtaining information)\textbf{]}
\end{description}
 \item[$\supset$] \vspace{-0.2cm} \textit {threat. violation of semantic compatibility} \\
   \vspace{0.1cm} \hspace{-0.23cm}  $\Rightarrow$  \hspace{0.5cm} \textit{explanation}:* 
\end{description}
\end{multicols}
\begin{figure}
    \centering
    \includegraphics{image/Снимок экрана 2023-11-08 124534.png}
    \caption{OSTIS Ecosystem Architecture}
    \label{fig:enter-label}
\end{figure}
\begin{multicols}{2}
\begin{description}[leftmargin=!, labelwidth=1cm, itemsep=-1.5mm]

\begin{description}[leftmargin=!, labelwidth=1.2cm, itemsep=-1.5mm]
   \item \textbf{[}violation of the generality of concepts and
in the generality of basic knowledge\textbf{]}
\end{description}
 \vspace{-0.2cm} \item[$\supset$] \textit{threat. destruction of knowledge base semantics
(semantic viruses)}

   \vspace{-0.2cm}  $\Rightarrow$  \hspace{0.5cm} \textit{explanation}:* 
   \par
   \begin{description}[leftmargin=!, labelwidth=0.9cm, itemsep=-1.5mm]
  \item \vspace{-0.3cm}  \textbf{[}substitution or removal of nodes and links
between them in the knowledge base\textbf{]}
\end{description}


   \vspace{-0.2cm} \item[$\supset$] \textit {threat. excessive amount of incoming information} \\
    \vspace{-0.4cm}\item[$\supset$] \textit {threat. breach of non-repudiation} \\
   \vspace{0.1cm} \hspace{-0.23cm}  $\Rightarrow$  \hspace{0.5cm} \textit{explanation}:* 
  \begin{description}[leftmargin=!, labelwidth=0.9cm, itemsep=-1.5mm]
   \item \vspace{-0.4cm}\textbf{[}issuance of unauthorized actions as legal,
as well as concealment or substitution of
information about the actions of subjects\textbf{]}
\end{description}
    
\item[$\supset$] \vspace{-0.2cm} \textit {threat. breach of accountability} \\
\vspace{0.1cm} \hspace{-0.23cm}  $\Rightarrow$  \hspace{0.5cm} \textit{explanation}:* 
 \begin{description}[leftmargin=!, labelwidth=0.9cm, itemsep=-1.5mm]
   \item \vspace{-0.4cm}\textbf{[}unauthorized or erroneous change, distortion or destruction of information about
the performance of actions by the subject\textbf{]}
\end{description}

\item[$\supset$] \vspace{-0.1cm} \textit {threat. violation of authenticity (authenticity)} \\
\vspace{0.1cm} \hspace{-0.23cm}  $\Rightarrow$  \hspace{0.5cm} \textit{explanation}:* 
\begin{description}[leftmargin=!, labelwidth=0.9cm, itemsep=-1.5mm]
   \item \vspace{-0.4cm}\textbf{[}performing actions in the system on behalf
of another person or issuing unreliable
resources (including data) as genuine\textbf{]}
\end{description}

\item[$\supset$] \vspace{-0.1cm} \textit {threat. breach of credibility
} \\
\vspace{0.1cm} \hspace{-0.23cm}  $\Rightarrow$  \hspace{0.5cm} \textit{explanation}:*
\begin{description}[leftmargin=!, labelwidth=0.9cm, itemsep=-1.5mm]
   \item \vspace{-0.4cm}\textbf{[}intentional or unintentional provision and
use of erroneous (incorrect) or irrelevant
(at a specific point in time) information, as
well as the implementation of procedures
in violation of the regulations (protocol)\textbf{]}
\end{description}
\end{description}

\par 
\vspace{-0.1cm} Let’s present the main directions of ensuring the
information security of ostis-systems to prevent emerging
threats:

\begin{itemize}
   \vspace{0.18cm} \item limitation of information traffic analyzed by the 
intelligent system;
    \item policy of differentiation of access to the knowledge
base;
    \item connectivity;
    \item introduction of semantic metrics;
    \item  semantic compatibility;
    \item activity.
\end{itemize}
\par
It should be noted that at the design stage of the
OSTIS technology itself, the basic principles of ensuring
information security were already laid down as part of
the design of individual components of the system. So
already initially, support for semantic compatibility and
cohesion is provided in ostis systems due to the system’s
ability to detect malicious processes in the knowledge
base
\par
\textbf{Restriction of information traffic analyzed by the
intelligent system
}
\par
The exponential growth of the volume of information
circulating in information flows and resources under the
conditions of well-defined quantitative restrictions on
the capabilities of the means of its perception, storage,
transmission and transformation forms a new class of information security threats characterized by the redundancy
of the total incoming information traffic of intelligent
systems.
\par
As a result, the overflow of information resources of
an intelligent system with redundant information can
provoke the spread of distorted (destructive semantic)
information. The general methodology for protecting
intelligent systems from excessive information traffic is
carried out through the use of axiological filters that
implement the functions of numerical assessment of the
value of incoming information, selection of the most
valuable and screening (filtering) of less valuable (useless
or harmful) using well-defined criteria.


\end{multicols}
\end{document}