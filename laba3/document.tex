\documentclass{article}
\usepackage[utf8]{inputenc}
\usepackage[top=2cm, bottom=1.8cm, left=2.1 cm, right=2.1 cm, footskip=12mm]{geometry}
\usepackage{hyperref}
\usepackage{multicol}
\usepackage{enumitem}
\title{\textbf{Factor of Digital Culture in the Application of
Artificial Intelligence in Economics and
Education}}
\author{Boris Panshin \\ \textit{ Department of Digital economic of Economic faculty} \\ \textit{Belarusian State University} \\ Minsk, Belarus \\ Email: \href{mailto: panshin@bsu.by}{panshin@bsu.by}}
\date{}
\begin{document}
\maketitle
\begin{multicols}{2}
    \textbf{\textit{Abstract}—The purpose of the article is to attract the
attention of young researchers of research on the problems
of digital culture as one of the most important factors in
the successful development and implementation of artificial
intelligence technologies.It is shown that the relevance of
the development and application of AI technologies is due
to the growing complexity of the objectively necessary tasks
of economic management and the ever-increasing pace and
scale of digitalization.The importance of the digital culture
factor in the formation of a digital environment comfortable
for life and interaction is noted, which determines the
effectiveness of synergistic processes of self-assembly and
self-organization of complex dynamic systems, which are
modern society and economy in the context of global
digitalization.The role of digital culture is analyzed as a
science about the relationship of people to each other in
the digital environment and the environment itself with the
outside world and as an institution for achieving excellence
in the creation and application of digital technologies and
artificial intelligence technologies.}

\textbf{\textit{Keywords}—artificial intelligence, digital culture, synergistic effects in the economy and society, digital culture in the economy and education, digital culture of the enterprise.}
\begin{center}
\vspace{-5pt}
    \large{\textsc{I. Introduction}}
    \vspace{-5pt}
\end{center}

Currently, artificial intelligence (AI) is entering a new
phase of development and is increasingly becoming one
of the main catalysts for change in the economy and
education.

At the same time, decisions on how to use this
technology, to balance risks and opportunities, are made
primarily by large corporations, relegating to the background research on assessing the risks of developing and implementing applications with AI technologies. As a
result, the development of AI is very contradictory and
zigzag.

If in 2022 32 significant industrial machine learning
models were created, then only 3 by scientific centers. There is a tendency to reduce the departments dealing with ethics and security issues in corporations
like Microsoft or Google. This is mainly due to the
fact that AI technology has begun to require more and
more resources: personnel, information (databases) and computing power necessary to create such applications. At the same time, there is a growing interest in the
regulation of AI from the public administration: an
analysis of the situation in 127 countries showed that the
number of laws adopted in different countries containing
the phrase “artificial intelligence” has grown from 2nd
in 2016 to 37th in 2022 \href{https://www.mckinsey.com}{[1]}.

It is obvious that with the growing concerns about the
impact of AI on the labor market, the regulation of the
use of AI will be improved in various directions, which
requires the development of an appropriate methodological framework for assessing the risks of AI and choosing directions for the development and application
of applications with AI technologies. In terms of areas, in
our opinion, these are economics and education. In terms
of methodology, this is the application of the principles
of synergetics to the construction of qualitative models
(phase portraits) of economic development under the
influence of the growth in the use of AI in various fields
of activity with an assessment of the impact of the level
of human development on the consistency and success
of society development.
\begin{center}
\vspace{-5pt}
    \large{\textsc{II. The objective necessity of} AI}
    \vspace{-5pt}
\end{center}

According to the conclusions made by academician
V.M. Glushkov [2] in the early 70s of the last century, the
complexity of objectively necessary management tasks
is growing faster than the square of the number of
people employed in the economy of people. Since, as
a result of the constant development of technologies and
organizations, the complication and deepening of specialization and cooperation between economic entities, new
connections arise, and hence new management tasks. At
the same time, the number and complexity of emerging
tasks and the number of people employed in the economy
have a non-linear relationship, which makes it difficult
to build integrated planning and economic management
systems in the context of digitalization: building a digital
state plan and a digital strategic management system.
\setcounter{page}{27}
Appropriate training and retraining of personnel is of key
importance. If in recent centuries mental abilities were
more important than emotional skills and the ability to
work with hands, then now, there is a reversal of the
trend — emotional (social) skills, such as empathy, the
ability to build relationships and persuade, come to the
fore. At the same time, the balance between cognitive and
social skills will change significantly even in traditionally
intellectual professions, which necessitates appropriate
changes in the education system related to the development of emotional intelligence among schoolchildren
and students and its adaptation to activities in the digital
environment. Emotional intelligence is considered as
important as mental ability (IQ) because it helps to establish teamwork and achieve synergies in production and
management while reducing the number of employees.
With regard to AI, it is assumed that intelligent digital
technologies will replace people in routine tasks, and
people will be successful in activities that require good
social skills (soft skills) and interdisciplinary experience.
\begin{center}
\vspace{-5pt}
    \large{\textsc{III. Synergetics is a ”gathering point” of
opinions on the directions of digitalization
and the application of artificial intelligence
technologies}}
\vspace{-5pt}
\end{center}

The main goal of numerous theories of digital transformation and AI is an attempt to build certain models to predict the development directions of this process in order to increase the efficiency of the economy and the
sustainable development of society [3].

In the last two decades, in the analysis of various
problems, the principles of synergetics are increasingly
used — the science of the processes of development
and self-organization of complex systems, which, undoubtedly, is any modern enterprise. The main thing in
synergetics is the self-organization of the components of
complex systems when a certain variety of elements and
relationships between them are reached, and the presence
of a certain degree of culture of production participants.

Self-assembly refers to the process of combining system components into horizontal structures. And selforganization is the emergence of qualitatively new structures (bifurcation) as a result of multiple interactions
of components of lower hierarchical levels in order to
form a production environment that is comfortable for
interaction — convenient, fast, with minimal barriers.

In practice, various irrational phenomena in the course
of digital transformation are encouraged to turn to synergetics, which are not always amenable to clear definitions and explanations, but are definitely determined
by the level of organizational and digital culture of the
employees of the enterprise.

In a more simplified cybernetic representation, an
enterprise is a program that ideally works according to
self-assembly and self-organization algorithms through
the dynamic formation of a cultural environment and
cultural code (skills, abilities, traditions, values, ethics,
aesthetics). This program ”works” in the direction of
ever faster, more diverse and simplified interaction of
the elements of the enterprise to create higher forms
of organization in accordance with the laws of natural
harmony.

With this approach to considering the digital transformation of an enterprise, it becomes possible to analyze
production and get an explanation of the reasons and
forecasts for the development of modern digitized enterprises.

It is important to note the interdisciplinary nature of
the synergetic method, which requires the joint efforts
of scientists and specialists from various fields — from
philosophers, artists, mathematicians, engineers to managers and system analysts.

It can be assumed that both the industrial revolution
in the past and the modern information and digital
revolution are the result of the adaptation of people and
industries to new technologies. In turn, the problem of
creating an information and digital culture is to form the
environment and people’s skills (digital competencies) as
a condition for self-assembly and self-organization. The
understanding of technologies should turn into a desire
to use them effectively.

At the same time, as practice shows, the main engine of self-assembly and self-organization is artificial intelligence (AI), on the level of development and application of which economic growth and quality of life depend.
\begin{center}
\vspace{-5pt}
    \large{\textsc{IV. Principles of synergetics in the
development of the economy}}
\vspace{-5pt}
\end{center}

The structure of the economic system and society as
a whole is determined by the nature of the interaction
between its elements. From the point of view of synergetics, the goal of the economy as a subsystem of
society is the self-organization of producers and consumers of goods and services, which works according
to the pricing model ”goods — money — goods”. The
current market economy is focused on the concentration
of ownership, which leads to the division of players into
active and passive ones. In the time of Adam Smith, the
economy was a market economy as long as producers
and consumers strictly followed the postulates of the
Protestant ethic [4]. Over time, the ethical culture of selforganization of the economy under the influence of the
imperative of profit was increasingly eroded, leading to
an increase in inequality in society, which was offset by
an increase in consumption and a decrease in the birth
rate in the most civilized countries.
In a planned economy, self-organization was hampered
by excessive administrative regulation of the forms of
interaction between economic agents and prices for products and services. Therefore, it is necessary to move to a new paradigm - a digital sharing economy, which
will give a new impetus to self-organization due to the
greater information content of consumers and producers
and the growth of opportunities for their cooperation in
the production of goods and services. That is, cultural
self-organization is required first, and then economic selforganization takes place.

The digital economy of the future is an inclusive, solidarity and sharing economy, which means the maximum
involvement of the population in production, the distribution of income in accordance with collective interests, the
maximum efficiency in the use of the country’s resources
through cooperation between enterprises and individuals
based on trust and transferring responsibility for work
to lower levels of decision making. As a result, there
is more space for the emergence and development of
new ways of entrepreneurship and cooperation between
producers of goods and services and taking advantage of
the digital environment.
\begin{center}
\vspace{-5pt}
    \large{\textsc{V. Features of the digital environment}}
    \vspace{-10pt}
\end{center}

The digital environment is an integral part of the
natural and virtual worlds surrounding a person and becomes as significant as the natural world. The speed and
extent of change is critical. The forecast of unregulated
development of the digital environment is not optimistic.
It is not those who provide quality content that win, but
those who quickly gain a critical mass of consumers. The
intelligence level of data processing systems (“digital
footprints”) is growing, but trust in the system is decreasing. Deepfakes and chatbots increase the entropy of the
environment, which goes into the turbulence stage. The
environment is no longer conducive to the production
and perception of new information, ceases to be useful
for development and increases cognitive degradation. The
pattern of thinking and the pattern of behavior are changing. New synergies are emerging, and the challenge is to
predict bifurcation points and define digital development
trajectories. The maximum effort is to determine what
measures should be taken. To do this, it is necessary to
define some research framework and methodology. For
a qualitative assessment of development, synergetics is
most applicable, among the priority measures is raising
the level of digital culture [5].

The four new laws of robotics formulated in Frank
Pasquale’s book also confirm the growing importance of
culture in the creation and application of AI systems [6]:
\vspace{-20pt}
\begin{enumerate}
\setlength\itemsep{-4pt}
\item[1)] Robotic systems and AI should complement professionals, not replace them.

\item[2)] Robotic systems and AI should not pretend to be
people.

\item[3)] Robotic systems and AI should not fuel a zero-sum
arms race.

\item[4)] Robotic systems and AI must always contain an
indication of the identity of its creator (or creators),
operator (or operators) and owner (or owners).
\end{enumerate}
\vspace{-8pt}
\begin{center}
\vspace{-5pt}
    \large{\textsc{VI. Digital culture}}
 \vspace{-5pt}  
\end{center}

Digital culture is the science of the relationship of
people to each other in the digital environment and the
environment itself with the outside world. In the most
general sense, digital culture can be viewed, on the one
hand, as an institution for achieving excellence in the
creation and application of digital technologies, on the
other hand, as a set of practices for regulating the behavior of people and communities in the digital environment.
The methodology for creating an environment with such
characteristics is based on the synergistic principles of
self-assembly and self-organization of complex dynamic
systems, such as modern society and the economy in
the context of global digitalization. Synergetics makes it
possible to connect the humanities and natural sciences
and gives an understanding that we live in a highly nonlinear world, that social systems are historical and depend
on their ”trajectory” in the past [7].

The phenomenon of digital culture, due to its complexity, should be considered at three levels: a person,
an enterprise (community) and society as a whole:
\vspace{-8pt}
\begin{enumerate}
\setlength\itemsep{-4pt}
\item[1)]  With regard to an individual, the essence of culture
is the development of imaginative thinking, which
allows you to create an ethical coordinate system
for life in a digital environment. Culture creates
appreciation and self-esteem of the individual in
digital interactions. Digital culture is the ability to
understand the patterns of development of digital
systems, which gives a person additional vitality to
solve complex problems and determine their role
in shaping the digital environment. It is conscious
activity in the digital environment that gives rise
to digital culture.
\item[2)] In relation to the enterprise, digital culture is what
employees do, what they believe in and how they
behave over time, that is, it is the attitudes, behaviors and habits associated with digital technologies
that employees repeat over time.
\end{enumerate}
\vspace{-8pt}
For an enterprise, digital culture is to some extent a
task, after completing which one can begin to solve the
technical and organizational problems of introducing new
technologies into production and management. That is,
understanding that the digital environment predetermines
both the appropriate organizational structure of the business and the behavior of the employee in terms of his
competencies and values.

Cultural costs have a strong impact on the development
of the traditional economy and have even greater consequences in the digital economy. The more complex the
technology, the higher the requirements for qualification
and quality of interactions. Studies show that over 30%
of the obstacles to successful digital transformation of
enterprises are due to the cultural and behavioral problems of employees and the unwillingness of managers to
communicate effectively in the digital environment.
\end{multicols}
\end{document}