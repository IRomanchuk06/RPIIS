\documentclass{article}
\usepackage[utf8]{inputenc}
\usepackage[left=17mm, top=17mm, right=17mm, bottom=25mm, nohead]{geometry}
\usepackage{enumitem}
\usepackage{multicol}
\usepackage{ulem}
\usepackage{mathtools}
\setlist[itemize]{itemsep=0pt, parsep=0pt, partopsep=0pt, topsep=1.4pt}
\setcounter{page}{306}
\date{}
\begin{document}
\begin{multicols}{2}
    \begin{description}[leftmargin=!, labelwidth=0.7cm, itemsep=-1.5mm]
tems for detecting attacks at the network level and promptly responding to them. AI uses the accumulated statistics and knowledge base about threats;
    \end{description}

\begin{itemize}
    \item SOAR (Security Orchestration and Automated Response) — systems that allow you to identify information security threats and automate incident response. In solutions of this type, unlike SIEM systems, AI helps not only to analyze, but also automatically respond appropriately to identified threats;
    \item Application Security — systems that allow you to identify threats to the security of application applications, manage the process of monitoring and eliminating such threats;
    \item Antifraud — platforms detect threats in business processes and fraudulent transactions in real time. AI is used to identify deviations from identified business processes in order to detect intrusions or process vulnerabilities and increase adaptability to changing business process logic and metrics.
\end {itemize}
\par

The paper [3] proposes a method for constructing a neuroimmune system for analyzing information security incidents that combines data collection and storage (compression) modules, an information security event analysis and correlation module, and a network attack detection subsystem based on convolutional neural networks. The use of machine learning technologies in information security creates bottlenecks and system vulnerabilities that can be exploited and has the following disadvantages [4]:

\begin{itemize}
    \item  data sets that must be formed from a significant
number of input samples, which requires a lot of
time and resources;
     \item requires a huge amount of resources, including
memory, data and computing power;
     \item frequent false positives that disrupt the operation and
generally reduce the effectiveness of such systems;
     \item organized attacks based on artificial intelligence
(semantic viruses).
\end{itemize}

\par 
\textbf{Organization of information security in intelligent
systems of a new generation}
\par
Let’s define the goals of ensuring the information security of new generation systems.
\par
From the monograph [5] the objectives of ensuring the information security of traditional intelligent systems are:
\begin{itemize}

    \item ensuring the confidentiality of information in accordance with the classification;
    \item ensuring the integrity of information at all stages of related processes (creation, processing, storage, transfer and destruction) in the provision of public services;
    \item ensuring timely availability of information in the
provision of public services;
    \item ensuring observability aimed at capturing any activity
of users and processes;
    \item ensuring the authenticity and impossibility of refusal of transactions and actions performed by participants in the provision of public services;
     \item accounting for all processes and events related to the input, processing, storage, provision and destruction of data.
\end{itemize}

\par Since intelligent systems of the new generation will interact with similar systems while understanding what the request is about, the goals of the provision will look different. The goals of ensuring the information security of new generation intelligent systems are:
\begin{itemize}
    \item ensuring the safety of the semantic compatibility of information;
    \item ensuring the availability of information at different
levels of the intellectual system;
    \item minimization of damage from events that pose a
threat to information security.
\end{itemize}

\par Currently, classical approaches and principles have
been developed to ensure the security of knowledge bases
(data), communication interfaces (information exchange)
between the components of intelligent systems, such as
encryption of transmitted data, filtering of unnecessary
(redundant) content, and data access control policy.
\par The information security system should be created on
the following principles:

\begin{itemize}
    \item the principle of equal strength - means ensuring
the protection of equipment, software and control
systems from all types of threats;
    \item the principle of continuity - provides for continuous
security of information resources of the system for
the continuous provision of public services;
    \item the principle of reasonable sufficiency - means the
application of such measures and means of protection
that are reasonable, rational and the costs of which
do not exceed the cost of the consequences of
information security violations;
    \item the principle of complexity - to ensure security in
all the variety of structural elements, threats and
channels of unauthorized access, all types and forms
of protection must be applied in full;
    \item the principle of comprehensive verification - is to
conduct special studies and inspections, special engineering analysis of equipment, verification studies of
software. Emergency messages and error parameters
should be continuously monitored, hardware and
software equipment should be constantly tested,
as well as software integrity control, both during
software loading and during operation;
    \item the principle of reliability - methods, means and
forms of protection must reliably block all penetration routes and possible channels of information
\end{itemize}

\end{multicols}
\end{document}