\documentclass{article}
\usepackage[utf8]{inputenc}
\usepackage[left=17mm, top=17mm, right=17mm, bottom=15mm, nohead, nofoot]{geometry}
\usepackage{enumitem}
\usepackage{multicol}


\setlist[itemize]{itemsep=0pt, parsep=0pt, partopsep=0pt, topsep=1.4pt}
\setcounter{page}{270}


\begin{document}
\begin{multicols}{2}
\noindent the field of automatic generation of test questions. The
basics of how to use the knowledge base to automatically
generate objective questions are described in detail in the
literature [10], [12].

The main problems with the existing approaches to
test question generation are as follows:
    \begin{itemize}
        \item the approach of using electronic documents to automatically generate test questions requires a large number of sentence templates;
        \item the creation of text corpus requires a lot of human resources to collect and process various knowledge;
        \item existing approaches only allow to generate simple objective questions [11], [13].
    \end{itemize}

\noindent\textit{B. Automatic verification of user answers}

Automatic verification of user answers is divided into verification of answers to objective questions and verification of answers to subjective questions. The basic principle of verification of answers to objective questions is to determine whether the string of standard answers matches the string of user answers. The basic principle of verification of answers to subjective questions is to calculate the similarity between standard answers and user answers, and then to implement automatic verification of user answers based on the calculated similarity and the evaluation strategy of the corresponding test questions [14], [15]. The verification of answers to subjective questions was classified according to the approach to calculating the similarity between answers as follows:
    \begin{itemize}
        \item Based on keyword phrases
    
        This type of approach first allows to split the sentences into keyword phrases and then calculate the similarity between them according to the matching relationship of keyword phrases between sentences [16].
        \item Based on vector space model (VSM)
    
        The basic principle of VSM is to use machine learning algorithms to first convert sentences into vector representations, and then calculate the similarity between them [17].
        \item Based on deep learning 
    
        This type of approach allows the use of neural network models to calculate the similarity between sentences. Representative neural network models include: Tree-LSTM, Transformer and BERT [18].
        \item Based on semantic graph
    
        The basic principle of calculating the similarity  between answers (i.e., sentence or short text) using this type of approach is to first convert the answers into a semantic graph representation using natural language processing tools, and then calculate the similarity between them. A semantic graph is a network that represents semantic relationships between concepts. In the ostis-systems, the semantic graph is constructed using SC-code (as a basis for knowledge representation within the OSTIS Technology, a unified coding language for information of any kind based on semantic networks is used, named SC-code) [1]. The main advantage of this type of approach is computing the similarity between answers based on semantics. One of the most representative approaches is SPICE (Semantic Propositional Image Caption Evaluation) [19].
    \end{itemize}

The main disadvantages of the existing methods are as
follows:
    \begin{itemize}
        \item keyword phrase and VSM based approaches do not take into account semantic similarity between answers;
        \item semantic graph-based approaches supporting only the description of simple semantic structure;
        \item these approaches cannot determine the logical equivalence between answers;
        \item these approaches are dependent on the corresponding natural language.
    \end{itemize}

In ITS information is described in the form of semantic graphs and stored in the knowledge base. Therefore based on existing approaches and OSTIS Technology an approach to automatically generate test questions using knowledge bases and verify user answers based on the similarity between semantic graphs is proposed in this article.

\begin{center}
    III. PROPOSED APPROACH
\end{center} 

The subsystem can be divided into two parts according to the functions to be implemented: automatic generation of test questions and automatic verification of user answers. In the following, the functions of these two parts will be introduced separately.

\noindent\textit{A. Automatic generation of test questions}

The basic principle of automatic generation of test questions in the ostis-systems is to first extract the corresponding semantic fragments from the knowledge base using a series of test question generation strategies, then add some test question description information to the extracted semantic fragments, and finally store the semantic fragments describing the complete test questions in the universal subsystem. The subsystem allows a series of test questions to be extracted from the subsystem and formed into test papers according to the user’s requirements when test papers need to be generated. Test papers consisting of semantic graphs of test questions are converted to natural language descriptions using a nature language interface. An approach to developing natural language interface using OSTIS Technology is described in the literature [6]. In the following, the basic principles of automatic generation of test questions in the ostis-systems will be introduced using test question generation strategy based on class as examples.

\end{multicols}

\end{document}
