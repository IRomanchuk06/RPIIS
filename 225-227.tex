\documentclass{article}
\usepackage[utf8]{inputenc}
\usepackage[left=20mm, top= 30 mm, right=20mm, bottom= 30 mm, nohead]{geometry}
\usepackage{enumitem}
\usepackage{mathtools}
\usepackage{multicol}
\usepackage[russian]{babel}
\setcounter{page}{225} 

\title{\huge\bfseries Examples of Integrating Wolfram Mathematica \\ Tools into OSTIS Applications \\} 
\author{
  Valery B. Taranchuk\\
 Department of Computer Applications \\ and Systems\\
 Belarusian State University\\
 Minsk, Republic of Belarus\\
\itshape taranchuk@bsu.by\\ \and
 Vladislav A. Savionok \\ 
\itshape Department of Software\\
\itshape for Information Technologies\\
 Belarusian State University \\ 
 of Informatics and Radioelectronics \\ 
 Minsk, Republic of Belarus \\ 
 v.savenok@bsuir.by}\\
\date{}

\begin{document}
\maketitle  
\begin{multicols}{2}
\textbf{\small\textit{Abstract}—Within the concept of convergence and unification of intelligent computer systems of the new generation,
technical solutions are discussed, examples of development
and modernization, integration of Ecosystem OSTIS tools
with Wolfram Mathematica (WM) computer algebra system (CAS) are provided.}
\par
\textbf{\small
On the example of integration with specialized complex
of intellectual educational resource for the discipline “Computer Systems and Networks” the possibilities of using WM tools in ostis-system are discussed when solving problems
related, in particular, to topology of info-communication
networks. The application of WM tools for visualization of
network topology, as well as emulation of the search for
the optimal route for data transmission is shown.
} \par
\small\textbf{\textit{Keywords}—technological production process, adaptive
control, neural network, reinforcement learning, Industry 4.0, standard}\\
\begin{center}
    I. INTRODUCTION \\
\end{center}
\par Following the assessment of the current state of work
in the field of Artificial Intelligence (AI), it is possible
to affirm active local development of various directions
(non-classical logics, formal ontologies, artificial neural
networks, machine learning, soft computing, multi-agent
systems, etc.), however, a comprehensive increase in the
intelligence of modern intelligent computer systems does
not occur [1].
\par The key reasons of methodological problems of current state of Artificial Intelligence, as well as a number of actions required to solve them are outlined in [1].
What actions are needed to improve the current state?
First of all, it is necessary to converge and integrate
all directions of Artificial Intelligence and corresponding
construction of a general formal theory of intelligent
computer systems (ICS), the transformation of modern
variety of frameworks for development of different ICS
components into a single technology of complex design
and support of the full life cycle of these systems,
which guarantees the compatibility of all developed
components, as well as the compatibility of the ICS
as independent subjects, interacting between each other.
Convergence and unification of new generation of intellectual computer systems and their components is
required.
\par Convergence and unification of new generation intelligent computer systems and their components is necessary. At the same time, convergent solutions basically
mean optimized complexes that include everything necessary to solve AI tasks, organized and configured for
efficient use of information resources, simplification of
implementation processes, meeting the requirements of
maximum performance, availability of intelligent interface, simple and understandable for all categories of
users.
\par Supporting the outlined concepts, we note that such
problems can be effectively solved by developing, improving, regularly updating the content of intelligent
systems by incorporating CAS means. Below are a few
methodological and technical solutions for integration of
different types of knowledge, implemented by inclusion
of Wolfram Mathematica functions in the ostis-system
of support and maintenance of the teaching process of
one of the basic disciplines in high school, illustrated by
examples.
\begin{center}
    II. CONCRETIZATION AND VARIANTS FOR
INTEGRATING THE ECOSYSTEM OSTIS WITH CAS \\
\end{center}
\par Integration of the Ecosystem OSTIS with any service
means the ability to use the functionality of the service to
change the internal state of the system’s knowledge base.
Within Ecosystem OSTIS full and partial integration
levels are acceptable.
\par According to the Technology, full integration of the
Ecosystem OSTIS with any service implies the possibility of executing service’s function at platformindependent level using SCP language. That is, the task
of integration of such a service is reduced to allocation
of a graph structure processing algorithm and its implementation within a system’s knowledge base. As a result
of such integration there is no need to use a third-party
service, in fact, an Ecosystem component is used.
\par Partial integration means changing the state of the
system’s knowledge base at the stages of service function
225 execution. The depth of integration can vary. In some
cases, a service can refer to the knowledge base to get
additional information or to record intermediate results.
In the simplest case, a knowledge base can change
only once, after a result of the service’s function is
received. 
In case of partial integration it is supposed that
particular ostis-systems are to play the role of system
integrators of included resources and services of other
computer systems, as the level of intelligence of ostissystems allows them to specify the computer systems
being integrated to a sufficient degree of detail and,
consequently, to “understand” adequately what each of
them knows and/or can do.
\par Separately, let us note that following the Technology,
ostis-systems are capable (and it should be used) to coordinate the activities of a third-party resource and service sufficiently well, to provide a “relevant” search for the
required component. The systems themselves can also
perform the role of intelligent help-systems – assistants
and consultants for efficient operations with functional
capabilities, when the user interface is implemented with
non-trivial semantics in the unique tasks of complex
subject areas. Such help systems can be made intelligent
intermediaries between the relevant computer systems
and their users.
\par The systems themselves can also act as intelligent
help systems. Their relevance is dictated by the high
complexity of subject areas and the non-triviality of
some unique tasks. Such conditions require the design
of appropriate unique and nontrivial user interfaces with
additional information support for their use. Such help
systems can be made intelligent intermediaries between
the relevant computer systems and their users, and the
homogeneity of the technologies used ensures seamless
integration with the existing system.
\par \textbf{Solving the issues of data format coordination.} A
tedious problem of functional service integration when
forming a digital ecosystem of multiple interacting services is the difference in data formats that participants of this Ecosystem work with. Two services, which imply
data processing from one subject area, are likely to have
different data formats. The problem of coordinating the
data format of different services significantly complicates
the development of the services themselves and leads to
an increase in time costs. Such issues can be effectively
solved using CAS import and export functions, such as
in Wolfram Mathematica, which supports more than 100
data formats, including graphics, video, and more.
\par At the current stage of development and usage of
the Ecosystem OSTIS, one of the priority areas appears to be the integration of the capabilities of computer algebra systems and \underline{intelligent learning systems
constructed in ostis-application.} The importance of this
is due to the relevance, the requirements of the intellectualization of educational resources on the one hand,
and on the other – contents of computer algebra systems,
which have an undoubted advantage and great opportunities for solving problems relevant to educational systems
for virtually all natural-science and technical disciplines,
involving the use of complex mathematical apparatus.
\par It can be stated that, despite the popularity of topics related to automation and intellectualization of educational
activities in science disciplines and the development of
appropriate computer systems, at the moment the market is practically lacking tested \underline{intellectual} educational
systems capable of \underline{self generating} and solving various
problems, and verifying the correctness of the solution
provided by the user. As prototypes, there are some
systems that consider non-trivial problems, such as geometry [2], [3] and graph theory [4]. But, to be fair,
it should be noted that there is no intelligence in the
mentioned systems (in fact, only a specific set of actions
is implemented, the tasks are not generated in the applications themselves), there is no means of verification of solutions with even minor deviations of the design rules.
\par One of the variants for interaction between Ecosystem OSTIS and CAS can be approaches similar to the
integration of artificial neural networks in ostis-systems
(see [5]).
\par Developing the aforementioned implementations, the
following methodological and technical solutions can be
considered:
\begin{itemize}
  \item Black-box integration, when the knowledge base
of the ostis-system contains the specification of
the used kernel function of the computer algebra
system, as well as the specification of the method
of calling this function (for example, specifying
through which software interface the interaction
with this external system is performed). This integration variant is the easiest to implement and
generally has the advantages listed below. At the
same time, this variant has a disadvantage that the
ostis-system does not contain means of analysis
and explanation of how a certain step of solving
a problem that is realized by a used CAS function
was taken.
\item A tighter integration, in which a particular function
is still a part of a third-party CAS, when not only the
result of its performance is loaded to the knowledge
base of ostisystem, but also all possible specification
of it, e.g. explanation of the problem solution step,
indication of particular algorithms and formulas
which can be involved in the solution, description
of possible alternative solution variants, evaluation
of solution efficiency and so on. In this variant of
integration, the ostis-system gets more opportunities
of analysis and explanation of the problem solution
process. (Note that this doesn’t apply specifically to
CAS Wolfram Mathematica, because it always has
detailed explanations for all solutions, and allows
226 for step-by-step execution.)
\item Full integration, which translates computer algebra functions in use from this system’s internal
language into the ostis-system. This variant is the
most labor-intensive and complicated in terms of
updating the capabilities of computer algebra systems in the corresponding ostis-systems taking into
account their constant development. At the same
time this integration variant, in comparison with
the two previous ones, has an important advantage:
it ensures platform-independent solution and allows
using all the advantages of the approaches proposed
within the OSTIS Technology in solving a concrete
problem, in particular the possibility of multi-user
parallel knowledge processing and the possibility to
optimize a problem solution plan or its fragments
directly during the solution.
\end{itemize} \par
The approach to solving the problems of intellectualization of educational activity, based on the integration of ostis-systems and computer algebra systems, has several
advantages:
\begin{itemize}
    \item When developing ostis-systems, the need to program many functions that have already been implemented and tested in CAS is eliminated. This
is fundamental because computer algebra systems
are developed by highly qualified specialists in
the relevant fields, the implementation of similar
functions in ostis-systems may require significant
financial and time expenditures.
\item A concrete ostis-system using individual functions
of CAS, due to the approach to the development
of hybrid problem solvers in OSTIS Technology,
gets the possibility to self plan the course of problem solving provided that some of its steps are
implemented by means of the attached functions.
From the point of view of the approach proposed
within the framework of OSTIS Technology, each
function of the computer algebra systems, computer
mathematics systems (CMS) becomes a method
for solving problems of some class. This class of
problems is described in the knowledge base of
the ostis-system and allows it, when solving a concrete problem, to independently draw a conclusion
about the expediency of applying one or another
CAS function. Such integration with ostis-systems
will make it possible to eliminate a possible disadvantage of individual computer algebra systems
noted earlier (determined by which CAS are used
– explained below in the overview of computer
mathematics systems, conditions of their application
and access to individual components).
\end{itemize} \par
We emphasize that these integration variants are not
mutually exclusive and can be combined. In addition,
integration can be deepened step by step taking into
account the above advantages and disadvantages as well
as the relevance of using certain functions of computer
algebra systems in solving specific tasks within the
Ecosystem OSTIS and corresponding ostis-systems.
\par In general case, step-by-step integration of CAS with
the Ecosystem OSTIS implies, as a minimum, description of the specification of the basic functions of the
selected computer algebra system by means of the
OSTIS Technology, in other words – development of
the ontology of external functions. In case of Wolfram
family systems, the process of developing such ontology
can be automated due to the presence of the Wolfram
Language formal language and good documentation of
system functions.
\par Summarizing the above, we state: the integration of
educational systems developed on the basis of OSTIS
Technology and computer algebra systems will allow
us to create systems with intelligent properties in a
shorter time, and with the use of carefully developed
(mathematically, algorithmically) and repeatedly tested
tools.
\begin{center}
    III. FUNDAMENTALS, TERMINOLOGY. COMPUTER
MATHEMATICS SYSTEMS, COMPUTER ALGEBRA
SYSTEMS
\end{center}
\par In the mid-twentieth century, at the junction of mathematics and computer science, a fundamental scientific
trend, computer algebra, the science of efficient algorithms for calculating mathematical objects, emerged and
intensively developed. Synonyms for the term “computer
algebra” are: “symbolic calculations”, “analytical calculations”, “analytical transformations”, and sometimes
“formal calculations”. The field of computer algebra is
represented by theory, technology, software tools. Applied results include developed algorithms and software
for solving problems using a computer in which the
original data and results take the form of mathematical
expressions, formulas. The basic product of computer algebra became software computer algebra systems (CAS).
The range of mathematical problems solvable with the
help of CAS is constantly expanding. Considerable effort
is devoted to developing algorithms for computing topological invariants of varieties, nodes, algebraic curves,
cohomology of different mathematical objects, and arithmetic invariants of rings of integers in the fields of
algebraic numbers. Another direction of modern research
is quantum algorithms, which sometimes have polynomial complexity, while existing classical algorithms have
exponential complexity.
\par Research and development of theoretical foundations
and technologies for implementing methods and software
implementations of computer algebra tools continues.
Terms, definitions, names in descriptions of functions
and tools of these systems also undergo changes, some
formulations earlier given in separate manuals, reviews
of tool capabilities are being not only refined, but also


\end{multicols}
\end{document}
