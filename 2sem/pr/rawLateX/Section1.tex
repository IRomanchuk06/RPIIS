%Пример

\begin{SCn}
\begin{small}

\textbf{ВВЕДЕНИЕ В ГЛАВУ 7.8}

Современные программно-технологические комплексы геоинформационных систем очень эффективны, но сложны в освоении и применении, поэтому требуют специальной профессиональной подготовки конечных пользователей. Для внедрения систем геопространственного назначения в различные области знаний и сферы применения необходимо, чтобы специалисты различного вида деятельности без особых сложностей и дополнительного обучения могли решать характерные для геоинформационных систем задачи.
Для этого необходим переход от традиционных геоинформационных систем к геоинформационным системам нового поколения переход от традиционных геоинформационных систем к геоинформационным системам нового поколения, имеющим удобный пользовательский интерфейс.

\bigskip

\textbf{§ 7.8.1. Требования, предъявляемые к интеллектуальным геоинформационным
                 системам нового поколения}

\scnheader{геоинформационная система}
\scnidtf{географическая информационная система}
\scnidtf{ГИС}
\scnidtf{программная компьютерная система, обеспечивающая ввод, манипулирование, анализ 
     и вывод пространственно-соотнесенных данных (геоданных) о территории, социальных   
     и природных явлениях при решении задач, связанных с инвентаризацией, анализом, 
     моделированием, прогнозированием и управлением окружающей средой и 
     территориальной организацией общества}
\scnidtf{geographic information system}


    \scnrelfrom{разбиение}{ГИС по территориальному охвату}
    \begin{scnindent}
    \begin{scneqtoset}
    \scnitem{глобальные ГИС}
    \scnitem{субконтинентальные ГИС}
    \scnitem{национальные ГИС}
    \scnitem{региональные ГИС}
    \scnitem{субрегиональные ГИС}
    \scnitem{локальные ГИС}
    \scnitem{местные ГИС}
    \end{scneqtoset}
    \end{scnindent}
  
    \scnrelfrom{разбиение}{ГИС по уровню управления}
    \begin{scnindent}
    \begin{scneqtoset}
    \scnitem{федеральные ГИС}
    \scnitem{региональные ГИС}
    \scnitem{муниципальные ГИС}
    \scnitem{корпоративные ГИС}
    \end{scneqtoset}
    \end{scnindent}

    

    \scnrelfrom{разбиение}{ГИС по функциональности}
    \begin{scnindent}
    \begin{scneqtoset}
    \scnitem{полнофункциональные ГИС}
    \scnitem{ГИC для просмотра данных}
    \scnitem{ГИC для ввода и обработки данных}
    \scnitem{специализированными ГИC с дополнительными функциями}
    \end{scneqtoset}
    \end{scnindent}

    \scnrelfrom{разбиение}{ГИС по предметной области}
    \begin{scnindent}
    \begin{scneqtoset}
    \scnitem{городские ГИС}
    \scnitem{муниципальные ГИС}
    \scnitem{картографические ГИС}
    \scnitem{недропользовательские ГИС}
    \scnitem{горно-геологические ГИС}
    \scnitem{природоохранные ГИС}
    \scnitem{туристические ГИС}
    \scnitem{земельные ГИС}
    \end{scneqtoset}
    \end{scnindent}

\begin{scndecomposition}
    \scnitem{ГИС по проблемной ориентации}
    \scnitem{полимасштабные ГИС}
    \scnitem{пространственно-временные ГИС}

\end{scndecomposition}


\scnheader{данные в геоинформационных системах}
\scnidtf{data in the geographic information system}
\begin{scnsubdividing}
    \scnitem{растровые данные}
    \scnitem{векторные данные}
    \scnitem{точки}
    \scnitem{полилинии}
    \scnitem{многоугольники}
    \scnitem{семантические данные }
\end{scnsubdividing}

\scnheader{задача геоинформационной системы}
\begin{scndecomposition}
    \scnitem{задача анализа в геоинформационной системе}
    \scnitem{задача моделирования в геоинформационной системе}
    \scnitem{задача прогнозирования в геоинформационной системе}
    \scnitem{задача управления в геоинформационной системе}
\end{scndecomposition}
\scnnote{в геоинформационных системах основным объектом исследования являются знания и 
     данные об объектах местности, которые рассматриваются не только как  
     пространственные данные и знания, но и являются интеграционной основой для 
     различных предметных областей}
\newpage

\scnheader{пространственно-распределенная информация}
\scnidtf{spatially distributed information}
\scnidtf{пространственные данные}
\scnidtf{геопространственные данные}
\scnidtf{информация, связанная с местоположением, географическими характеристиками или координатами}
\begin{scnsubdividing}
    \scnitem{карты и изображения}
    \scnitem{данные о землепользовании, землевладении, дорогах, зданиях и других физических объектах}
    \scnitem{демографические данные}
    \scnitem{экологические данные}
    \scnitem{данные о природных ресурсах}
    \scnitem{данные измерения и наблюдения, полученные с помощью сенсоров, датчиков и мониторинговых систем}
\end{scnsubdividing}

\scnheader{геопространство}
\scnidtf{geospace}
\scnidtf{концепция, которая описывает физическое пространство, в котором находятся различные географические объекты, явления и процессы}

\scnheader{геоинформация}
\scnidtf{geoinformation}
\scnidtf{информация, связанная с определённым местоположением, географической позицией или физическими характеристиками территории}

\scnheader{геообъект}
\scnidtf{geo-entity}
\scnidtf{картографичесий объект}
\begin{scnsubdividing}
    \scnitem{площадные}
    \begin{scnindent}
     \scntext{пояснение}{объекты, площадь которых выражена в масштабе карты}
    \end{scnindent}
    \scnitem{полилинейные}
     \begin{scnindent}
      \scniselement{линейные}
     \begin{scnindent}
     \scntext{пояснение}{объекты, длина которых выражена в масштабе карты}
    \end{scnindent}
     \end{scnindent}
    \scnitem{точечные}
    \begin{scnindent}
     \scntext{пояснение}{объекты, которые не могут быть выражены в масштабе}
    \end{scnindent}
\end{scnsubdividing}


\scnheader{пространственные семантические отношения над объектами}
\scnidtf{spatial semantic relations over objects}
\begin{scnsubdividing}
    \scnitem{картографические (топологические) отношения, инвариантные
к топологическим преобразованиям объектов связи}
    \scnitem{метрические отношения с точки зрения расстояния и направления}
    \scnitem{отношения пространственной регулярности, описываемые предлогами before, behind, above и below}
\end{scnsubdividing}


\scnheader{пространственный запрос}
\scnidtf{spatial query}
\scnidtf{особый вид запроса к базам пространственных данных, который включает в себя пространственные условия и ограничения}
     
\scnheader{интеллектуальная геоинформационная система}
\scnidtf{информационная система, основным объектом исследования которой являются знания     и данные об объектах местности, выступающие интеграционной основой для решения 
    прикладных задач в различных предметных областях}
    \scnidtf{intelligent geographic information system}
    
\begin{scnsubdividing}
    \scnitem{растровые данные}
    \scnitem{векторные данные}
    \scnitem{точки}
    \scnitem{полилинии}
    \scnitem{многоугольники}
    \scnitem{семантические данные }
\end{scnsubdividing}    

\scnsuperset{интеллектуальная геоинформационная ostis-система}
\scnidtf{интеллектуальная геоинформационная система, разработанная по принципам     
        Технологии OSTIS}
\begin{scndecomposition}
    \scnitem{база знаний интеллектуальной геоинформационной ostis-системы}
    \scnitem{решатель задач интеллектуальной геоинформационной ostis-системы}
    \scnitem{картографический интерфейс интеллектуальной геоинформационной ostis-системы}
\end{scndecomposition}
\begin{scnatasks}
    \scnitem{[проектирование пространственных онтологий и на основе их решение проблемы семантической совместимости знаний предметных областей]}
    \scnitem{[решение задачи управления метаданными и совершенствования поиска, доступа и обмена в условиях растущих объемов пространственной информации и сервисов, предоставляемых многочисленными источниками геоинформации]}
    \scnitem{[осуществление вывода знаний с использованием пространственной и тематической информации как составляющих знаний объектов местности с использованием Языка вопросов]}
    \scnitem{[внедрение картографического интерфейса в интеллектуальные ostis-системы как естественного для человека способа представления информации об объектах местности]}
\end{scnatasks}

\newpage

\scnheader{подход к построению
интеллектуальных геоинформационных систем на основе технологии OSTIS}
\scnnote{особенностью данного подхода
является описание геообъектов и определение
пространственных семантических связей, описание формальной
онтологии семантики обозначений на картографическом языке,
что, в свою очередь, позволяет устанавливать семантику
отображения геообъектов на картах в зависимости от типов
объектов местности. Особое внимание уделяется формальной
онтологии объектов рельефа как необходимому условию для
обеспечение интеграции с предметными областями в интересах
ГИС}


\scnheader{геоинформационный проект}
\scnidtf{geoinformation project}
\scnidtf{наполнение геоинформационной системы пространственными данными и сведениями  
     об объектах в привязке к пространственным данным}
\begin{scnetaps}
    \scnitem{предпроектные исследования, включающие изучение функциональных требований, оценку функциональных возможностей геоинформационных систем, технико-экономическое обоснование}
    \scnitem{системное проектирование, включая стадию пилотного проекта, при необходимости — разработку геоинформационных систем или расширение существующих}
    \scnitem{тестирование на небольшом территориальном фрагменте, или тестовом участке, прототипирование, или создание опытного образца, или прототипа}
    \scnitem{внедрение}
    \scnitem{эксплуатация}
\end{scnetaps}
 

\end{small}
\end{SCn}
