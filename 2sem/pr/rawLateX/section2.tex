\begin{SCn}
\begin{small}

\scnheader{Гусева А.В.ГеоинфС-2013ст}
\begin{scnrelfromlist}{ключевой знак}
    \scnitem{геоинформационные системы}
    \scnitem{пространственно-распределенная информация}
    \scnitem{геопространство}
    \scnitem{геоинформация}
    \scnitem{пространственный запрос}
\end{scnrelfromlist}
\scntext{аннотация}{Описаны геоинформационные системы, области их применения и перспективы развития. Предполагается использование ГИС с целью ознакомления читателя с новейшими средствами IT-коммуникаций.}
\scntext{цитата}{Пространственно-распределенная информация -- это то, с чем человек сталкивается практически каждый день вне зависимости от рода своей деятельности. Это может быть схема метро или план здания, топографическая карта или схема взаимосвязей между офисами компании, атлас автомобильных дорог или контурная карта и многое другое. ГИС дает возможность накапливать и анализировать подобную информацию, оперативно находить нужные сведения и отображать их в удобном для использования виде. Применение ГИС-технологий позволяет резко увеличить оперативность и качество работы с пространственно-распределенной информацией по сравнению с традиционными методами картографирования.}
\begin{scnindent}
    \scnrelto{пояснение}{ГИС}
\end{scnindent}
\scntext{цитата}{Геопространство - разновидность пространства, характеризующаяся протяженностью, динамичностью, структурностью, непрерывностью.}
\begin{scnindent}
    \scnrelto{пояснение}{Геопространство}
\end{scnindent}
\scntext{цитата}{Геоинформация - это координированная информация о геопространстве и его объектах в цифровой компьютерно-воспринимаемой форме, предназначенная в качестве исходного материала для моделирования геопространства.}
\begin{scnindent}
    \scnrelto{пояснение}{Геоинформация}
\end{scnindent}
\scntext{цитата}{ГИС-технология объединяет традиционные операции при работе с базами данных, такими как запрос и статистический анализ, с преимуществами полноценной визуализации и географического (пространственного) анализа, которые предоставляет карта. Возможность визуализации и пространственного анализа отличают ГИС от других информационных систем и обеспечивают уникальные возможности для ее применения в широком спектре задач.}
\begin{scnindent}
    \scnrelto{пояснение}{ГИС}
\end{scnindent}

\newpage

\scnheader{Самодумкин С.А.Next-genIGS-2022ст}
\begin{scnrelfromlist}{ключевой знак}
    \scnitem{OSTIS}
    \scnitem{интеллектуальная геоинформационная система}
    \scnitem{частная технология проектирования}
    \scnitem{онтология}
\end{scnrelfromlist}
\scntext{аннотация}{В статье рассматривается подход к строительству
интеллектуальных геоинформационных систем на базе OSTIS
Рассматривается технология. Формальная онтология синтаксиса
язык отображения задан явно, что, в свою очередь, позволяет
определение типов картографических объектов и настройка пространственного
семантические отношения; формальная онтология обозначения
задана семантика языка отображения, что, в свою очередь, позволяет
установление семантики отображения геообъектов на
карты в зависимости от типов объектов рельефа; формальный
в качестве необходимого условия задана онтология объектов рельефа
для интеграции с предметными областями в интересах ГИС.}
\scntext{цитата}{Для расширения задач, решаемых геоинформационными системами, унификации различных типов представления информации
в ГИС о пространстве, времени и Земле
необходимо интегрировать существующие веб-геосервисы и
технологии проектирования интеллектуальных систем с целью разработки
геоинформационных систем нового поколения как класса
интеллектуальных компьютерных систем, основанных на едином способе
кодирования информации и функциональной совместимости (interoperability)
что является необходимым требованием.}
\scntext{цитата}{Для решения проблем, поставленных в рамках этой статьи, предлагается
разработать сложную предметную область геоинформатики
и соответствующую онтологию объектов рельефа.}
\scntext{цитата}{Основой для построения онтологической модели объектов
рельефа является классификатор топографической информации, отображаемой на топографических картах и планах городов, разработанный
и действующий в настоящее время в Республике Беларусь NCRB
012-2007 [10].}


\end{small}
\end{SCn}