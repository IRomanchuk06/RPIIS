\documentclass{article}
\usepackage[T2A]{fontenc}
\usepackage[utf8]{inputenc}
\usepackage[left=10mm, top=9mm, right=10mm, bottom=10mm, nohead, nofoot]{geometry}
\usepackage{enumitem}
\usepackage{float}
\usepackage{multicol}
\usepackage[justification=centering]{caption}

\setlist[itemize]{itemsep=0pt, parsep=0pt, partopsep=0pt, topsep=1.4pt}
\setcounter{page}{277}


\begin{document}
\begin{multicols}{2}
\vspace{-0.6cm}

\begin{table}[H]
\caption{TABLE. STATISTICAL RESULTS OF STUDENT SCORES}
\begin{center}
\vspace{-0.3cm}
\begin{tabular}{|p{1.3cm}|p{0.5cm}|p{0.6cm}|p{0.7cm}|p{0.7cm}|p{0.7cm}|p{0.7cm}|p{0.75cm}|}
\hline
Score & \textless 40 & [40-49] & [50-59] & [60-69] & [70-79] & [80-89] & $\geq$ 90 \\
\hline
Total number of students (40) & 0 & 1 & 4 & 10 & 14 & 8 & 3 \\
\hline
\leftskip=-0.1cm \small{Proportion} & 0 & 2.5\% & 10\% & 25\% & 35\% & 20\% & 7.5\%\\
\hline
Average score & \multicolumn{7}{|c|}{72.85} \\
\hline
\end{tabular}
\end{center}
\end{table}
\noindent of test paper is moderate and that the actual knowledge
level of the user can be checked objectively and fairly

In order to evaluate the closeness between the automatic scoring and manual scoring of user answers to the subjective questions, we decided to first enter the 40 students’ answers to the subjective questions into the subsystem, then use the subsystem to automatically verify the students’ answers, and finally count the error between the automatic scoring and manual scoring of user answers to the subjective questions (Table III).
\begin{table}[H]
\caption{TABLE. RESULTS OF SCORING ERROR STATISTICS FOR USER ANSWERS TO SUBJECTIVE QUESTIONS}
\begin{center}
\vspace{-0.3cm}
\begin{tabular}{|p{1cm}|p{1cm}|p{1cm}|p{1cm}|p{1cm}|p{0.7cm}|p{1.1cm}|}
\hline
Error range (ф) & \hspace{-0.35cm} \small{Definition explanation question 1} & \hspace{-0.35cm} \small{Definition explanation question 2} & Proof question 1 & Proof question 2 & Total & \hspace{-0.4cm} \small Proportion\\
\hline
Ф $\leq$ 1 & 35 & 31 & 26 & 28 & 120 & 75\% \\
\hline
(1-1.5] & 2 & 4 & 8 & 8 & 22 & 13.75\%\\
\hline
(1.5-2] & 2 & 3 & 5 & 3 & 13 & 8.125\%\\
\hline
Ф \textgreater 2 & 2 & 3 & 5 & 3 & 13 & 8.125\%\\

\hline
\end{tabular}
\end{center}
\end{table}

The formula for calculating the error Ф is shown below(4):

\vspace{-0.2cm}
\begin{center}
Ф = |x - y| \hspace{3cm} (4)
\end{center}

The parameters are defined as shown below:
\begin{itemize}
    \item x — is the manual scoring of user answers to the test questions;

    \item y — is the automatic scoring of user answers to the test questions;
\end{itemize}

From the Table III, it can be seen that the automatic scoring and manual scoring of user answers to subjective questions in the tutoring system for discrete mathematics generally remained consistent, and that when the maximum score for a subjective question was 10, the sample size with an error Ф $\leq$ 1.5 between scores was over 88\%.

The above experimental results show that the developed subsystem can satisfy the conditions for practical applications.
 \begin{center}
    VII. CONCLUSION
\end{center}

An automated approach to checking the knowledge level of users in tutoring systems developed using OSTIS Technology is proposed in this article. Based on the proposed approach, a universal subsystem for automatic generation of test questions and automatic verification of user answers is developed. Using the developed subsystem, the entire process can be automated from test question generation, test paper generation to automatic verification of user answers and automatic scoring of test papers.

Finally the effectiveness of the developed subsystem was evaluated in terms of the availability of the generated test questions, the difficulty of the generated test papers and the closeness between the automatic scoring and the manual scoring of the test questions in the discrete mathematics ostis-system. From the evaluation results, it can be seen that the developed subsystem can meet the conditions for practical application.

 \begin{center}
     ACKNOWLEDGMENT
 \end{center}

The work in this article was done with the support of research teams of the Department of Intelligent Information Technologies of Belarusian State University of Informatics and Radioelectronics. Authors would like to thank every researcher in the Department of Intelligent Information Technologies.

\begin{center}
    REFERENCES
\end{center}
\fontsize{9}{11}\selectfont


\leftskip = 0.5cm [1] V. V. Golenkov and N. A. Guljakina, “Proekt otkrytoj semanticheskoj tehnologii komponentnogo proektirovanija intellektual’nyh sistem. chast’ 1: Principy sozdanija project of open semantic technology for component design of intelligent systems. part 1: Creation principles],” Ontologija proektirovanija [Ontology of design], no. 1, pp. 42–64, 2014.

[2] V. Golenkov, N. Gulyakina, “Next-generation intelligent computer systems and technology of complex support of their life cycle,” in Otkrytye semanticheskie tehnologii proektirovanija intellektual’nyh sistem [Open semantic technologies for intelligent systems], V. Golenkov, Ed., BSUIR. Minsk, BSUIR, 2022, pp. 27–40.

[3] D. Shunkevich, “Metodika komponentnogo proektirovaniya sistem, upravlyaemyh znaniyami,” in Otkrytye semanticheskie tehnologii proektirovanija intellektual’nyh sistem [Open semantic technologies for intelligent systems], V. Golenkov, Ed., BSUIR. Minsk, BSUIR, 2015, pp. 93–110. 

[4] D. Shunkevich, “Hybrid problem solvers of intelligent computer systems of a new generation,” in Otkrytye semanticheskie tehnologii proektirovanija intellektual’nyh sistem [Open semantic technologies for intelligent systems], V. Golenkov, Ed., BSUIR. Minsk, BSUIR, 2022, pp. 119–144.

[5] (2023, Mar) Ims.ostis metasystem. [Online]. Available: https://ims.ostis.net

[6] Longwei. Qian, “Ontological Approach to the development of natural language interface for intelligent computer systems,” in Otkrytye semanticheskie tehnologii proektirovanija intellektual’nyh sistem [Open semantic technologies for intelligent systems], V. Golenkov, Ed., BSUIR. Minsk, BSUIR, 2022, pp. 217–238.

[7] Wenzu. Li, “A semantics-based approach to automatic generation of test questions and automatic verification of user answers in the intelligent tutoring systems,” in Otkrytye semanticheskie tehnologii proektirovanija intellektual’nyh sistem [Open semantic technologies for intelligent systems], V. Golenkov, Ed., BSUIR. Minsk, BSUIR, 2022, pp. 369–374.

[8] Li. W., Qian L. Semantic Approach to User Answer Verification in Intelligent Tutoring Systems. Communications in Computer and Information Science, 2022, Vol. 1625, pp. 242–266.

\end{multicols}
\end{document}
