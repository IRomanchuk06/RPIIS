\documentclass{article}
\usepackage[utf8]{inputenc}
\usepackage[left=20mm, top=10mm, right=20mm, bottom=10mm, nohead, nofoot]{geometry}
\usepackage{enumitem}
\setlist[itemize]{itemsep=2pt, parsep=0pt, partopsep=5pt, topsep=0pt}
\usepackage{mathtools}
\usepackage{multicol}
\usepackage[russian]{babel}
\setcounter{page}{291}

\title{\huge\bfseries Adaptive Control System for Technological \\ Process within OSTIS Ecosystem \\} 
\author{
 Valery Taberko and Dzmitry Ivaniuk\\
 \itshape JSC «Savushkin Product»\\
 Brest, Belarus\\
 Email: tab@pda.savushkin.by,\\
 id@pda.savushkin.by\\ \and
 Viktor Smorodin and Vladislav Prokhorenko\\
\itshape Department of Mathematical Problems\\
\itshape of Control and Informatics\\
\itshape Francisk Skorina Gomel State University\\
 Gomel, Belarus\\
 Email: smorodin@gsu.by, snysc@mail.ru}\\
\date{}

\begin{document}
\maketitle  
\begin{multicols}{2}
\textbf{\small\textit{Abstract}—In this paper an approach to building a hybrid intellectual computer system for adaptive control of a
technological production cycle is being proposed in the
form of an ostis-system solver based on the ontology of
the "technological production processes with probabilistic
attributes" domain knowledge. The idea of development and
implementation of mathematical models of neural network
regulators for control optimization problems is the basis for
the system solver. Such an implementation makes it possible
to integrate the proposed solution with other developed
solutions as well as the company’s software in order to
allow building intellectual systems for automated control,
recommendation systems and decision support systems,
information support systems for the company’s personnel.} \par
\small\textbf{\textit{Keywords}—technological production process, adaptive
control, neural network, reinforcement learning, Industry 4.0, standard}\\
\begin{center}
    I. INTRODUCTION \\
\end{center}
\par The constant efforts of scientific and technological
advances in the sphere of optimization of production
systems operation require the development of up-to-date
approaches to adaptive control of production processes
that would include elements of artificial intelligence,
neural network modeling and development of intellectual
computer systems of new generation.
\par In this paper an approach is being proposed for building
a hybrid intellectual computer system for adaptive control
of a technological production cycle within the framework
of OSTIS Ecosystem. The formal description of the
control object is based on ontologies of the "technological
production processes with probabilistic attributes" domain
knowledge and ISA 5.1, ISA-88, ISA-95 standards,
implemented through the means of OSTIS Technology.
\par Control adaptation is implemented using the means of
software-hardware coupling of neural network regulators
and hardware control system of the technological object
in real-time.
\par The idea of the hybrid intellectual system for adaptive
control is based upon development and implementation
of mathematical models of neural network regulators for
solving problems of control optimization, implementation
of methods and algorithms of synthesis of feedback
control for technological cycle, depending on changing
parameters of the control object operation.
\begin{center}
    II. TERMS AND DEFINITIONS \\
\end{center}
\par The central concepts within the knowledge domain under consideration are technological process (technological
cycle) and probabilistic technological process [2].
\begin{itemize}
    \renewcommand{\labelitemi}{ 1)}
    \item A technological process (TP) (technological cycle) of production is a sequence of interconnected operations, that is defined by technological documentation, directed at the object of process with purpose of producing the required output. Within ISA-88 standard’s procedural control module TP is defined as a procedure, that produces a batch of product (that may be a final product or intermediate product used in the further stages of production of the final product).
    \renewcommand{\labelitemi}{2)}
    \item A technological operation is part of a technological process, that is being run continuously at one workplace on one or more simultaneously processed. or assembled products by one or more operators. Within ISA-88 standard it is defined as operation. that results in substantially altered product properties. Operations can be run by an operator who can start, pause and resume them. At each moment of time only one operation can be active within a machine (unit).
    \renewcommand{\labelitemi}{3)}
    \item Microtechnological operation is a finite sequence of elementary operations which constituent the contents of a technological operation, that is run continuously at one workplace. Within ISA-88 standard it is a phase that results in small changes in the properties of a product. Those steps can be run in parallel, in sequence or as a combination of two. Operator can’t directly control the steps (start/pause/resume).
    \renewcommand{\labelitemi}{4)}
    \item A Probabilistic technological process is a technological process that has probabilistic parameters of operation; a technological process with a structure. that may change during its operation.
\end{itemize}
\end{multicols}
\newpage
\begin{multicols}{2}
\begin{itemize}
    \renewcommand{\labelitemi}{ 5)}
    \item A Control system is a well-defined set of hardwaresoftware means for control of a technological object, that makes it possible to collect readings of its state and to influence its operation in order to achieve the given goals.
    \renewcommand{\labelitemi}{6)}
    \item  Adaptive control is a set of methods and algorithms that allow synthesizing the control feedback connections that can change parameters (control structure) of the neuroregulator based on the control actions and external disturbances.
    \renewcommand{\labelitemi}{7)}
    \item Automated control system of a technological process (ACSTP) is a complex of technical and software means that makes it possible for technological units to operate in automated mode based upon the chosen control criteria.\\
\end{itemize}
\textbf{technological process}\\
\(\coloneqq\) \quad [technological cycle] \\
\(\coloneqq\) \quad [is a sequence of interconnected operations, that is \par\quad defined by technological documentation, directed \par\quad at the object of process with purpose of producing \par\quad the required output] \\
\(\coloneqq\) \quad [a set of technological operations \(\{TCO_i_j\}\), where \par\quad i, j = \(\overline{1, N,}\) as well as the resources consumed by \par\quad those operations] \\ \\
\textbf{probabilistic technological process}\\
\(\coloneqq\) \quad[PTP] \\
\(\subset\) \quad technological process\\
\(\coloneqq\) \quad[technological process that has probabilistic \par\quad parameters of operation]\\
\(\coloneqq\) \quad[a technological process with a structure that may \par\quad
change during its operation] \\ \\
\textbf{technological operation}\\
\(\coloneqq\) \quad[TCO]\\
\(\subset\) \quad technological process\\
\(\coloneqq\) \quad[subset of a technological process, that is being \par\quad run continuously at one workplace on one or more \par\quad simultaneously processed or assembled products by \par\quad one or more operators]\\ \\
\textbf{microtechnological operation}\\
\(\coloneqq\) \quad[MTCO]\\
\(\subset\) \quad technological operation\\
\(\coloneqq\) \quad[finite sequence of elementary operations which \par\quad constituent the contents of a technological \par\quad operation, that is run continuously at one \par\quad workplace]\\ 
\\ \\ \\ \\ \\ \\ \\ \\ \\ \\
\begin{center}
    \small III. PROBLEMS OF ADAPTIVE CONTROL OF PRODUCTION PROCESSES IN THE CONTEXT OF NEW GENERATION COMPUTER SYSTEMS DEVELOPMENT
\end{center} 
\par
Enterprise automation tools must be able to adapt
to any changes in the production process itself with
minimal costs and time delays. Such changes may include
expansion or reduction in production volumes, changes
in production nomenclature, replacement or changes of
technological units, alteration of the overall production
structure, changes of interactions between suppliers and
consumers, changes of the legal acts and standards, as
well as other unforeseen circumstances of different nature. \par
Analysis of the field in the sphere of the modern
controlled production systems research demonstrates
that the problem of determining the real-time operation
parameters of such research objects emerges primarily
in the cases of complex technical products production
that requires high manufacturing precision and high labor
productivity. \par
In such cases when solving a multicriteria control
optimization problem high standards need to be applied to
the algorithms of the operation of the production process,
minimization of human factor impact on the technological
production cycle operation quality, prevention of occurrence of technogenic emergencies. Such a case is typical
for robotic production systems that operate under control
of the hardware and software controller that administers
the functioning of the technological cycle control system
according to the given programs. \par
At the same time the arising emergency situations due
to equipment failures, random external control actions,
including human factor, lead to deviation of the operation
parameters of the production system from the desired
values, which leads to the necessity of their correction
in real-time based on the neuroregulator models that
operate within the hardware-software coupling means of
the technological production cycle. \par
The existing special artificial intelligence models such
as neural networks have unique properties and can be
used as universal approximators that have a capability
to generalize the data against which they were trained.
Such features make it practical to use such models when
solving complex problems in the domain of adaptive
control. \par
The modern convergence tendencies in the sphere of
intellectual systems development [1] requires the development of the appropriate software that would feature
elements of cognitive abilities based on semantically
compatible technologies of artificial intelligence. This
sphere also includes the development of computer systems
that are able to provide intellectualization of the processes
of making analytical control decisions (which are directly
related to adapting control processes for complex dynamic
systems (technological objects) in real-time), building
semantically compatible knowledge bases in the domain
\end{multicols}
\newpage
\begin{multicols}{2}
\noindent of dynamic systems operation analysis and optimization
of the operation of complex control systems based upon
them through open-source intellectual decision support
systems development.
\begin{center}
\small IV. BUILDING A MATHEMATICAL MODEL OF A PRODUCTION SYSTEM IN THE CONTEXT OF INDUSTRY 4.0
\end{center}
\par
Implementation of Industry 4.0 concept at industrial
enterprises is accompanied by creation of a unified
ontological model of the production process. This model is
the core of integrated information service for the company
[13]. \par
In order to allow a wide implementation of the artificial
intelligence technologies for automating the company, all
corporate knowledge must be transformed into the formal
language for knowledge representation. Such knowledge
may be obtained from the existing documentation that
describes the enterprises’ operation within the framework
of accepted international standards. \par
It is necessary to transform each existing standard
into a knowledge base based on the appropriate set of
ontologies related to the standard. Such an approach
allows to significantly automate the processes of standard
development and its application. \par
As an example let us consider ISA-88 standard [16]
(the base standard for prescription production). Even
though this standard is widely used by American and
European companies (and is being actively introduced
in Republic of Belarus) it has a number of flaws that
are listed below. Authors’ experiences with ISA-88 and
ISA-95 have demonstrated the following problems related
to versions of the standard (see [8]):
\begin{itemize}
    \renewcommand{\labelitemi}{1)}
    \item American version of standard – ANSI/ISA-88.00.01-2010 – has been updated and is in its 3rd edition (as of 2010);
    \renewcommand{\labelitemi}{2)}
    \item ISA-88.00.02-2001 – includes data structures and guidelines of languages;
    \renewcommand{\labelitemi}{3)}
    \item ANSI/ISA-TR88.00.02-2015 – describes an implementation example of ANSI/ISA-88.00.01;
    \renewcommand{\labelitemi}{4)}
    \item ISA-88.00.03-2003 – an activity that describes the use of common site recipes with and across companies;
    \renewcommand{\labelitemi}{5)}
    \item ISA-TR88.0.03-1996 – shows possible recipe procedure presentation formats;
\renewcommand{\labelitemi}{6)}
    \item ANSI/ISA-88.00.04-2006 – structure for the batch production records;
\renewcommand{\labelitemi}{7)}
    \item ISA-TR88.95.01-2008 – describes how ISA-88 and ISA-95 can be used together;
\renewcommand{\labelitemi}{8)}
    \item At the same time, the European version approved in 1997 – IEC 61512-1 – is based on the older version ISA-88.01-1995;
\renewcommand{\labelitemi}{9)}
    \item Russian version of the standard – GOST R IEC 61512-1-2016 – is identical to IEC 61512-1, that is also outdated. Also it raises a number of questions related to the not very successful translation of the original English terms into Russian.
\end{itemize}
\par
Another standard that is often used in the context
of Industry 4.0 is ISA-95 [17]. ISA-95 is an industry
standard for describing high level control systems. Its
main purpose in to simplify the development of such
systems, abstract from the hardware and provide a single
interface to interact with ERP and MES layers. It consists
of the following parts (see [8]):
\begin{itemize}
    \renewcommand{\labelitemi}{1)}
    \item ANSI/ISA-95.00.01-2000, Part 1: «Models and terminology» — it consists of standard terminology and
object models that can be used to determine what
information is exchanged;
    \renewcommand{\labelitemi}{2)}
    \item ANSI/ISA-95.00.02-2001, Part 2: «Object Model
Attributes» — it consists of attributes for each object
defined in Part 1. Objects and attributes can be used
to exchange information between different systems
and can also be used as the basis for relational
databases;
    \renewcommand{\labelitemi}{3)}
    \item ANSI/ISA-95.00.03-2005, Part 3: «Models of Manufacturing Operations Management» — it focuses on
Level 3 (Production/MES) functions and activities;
    \renewcommand{\labelitemi}{4)}
    \item ISA-95.00.04 Part 4: «Object models and attributes
for Manufacturing Operations Management». The
SP95 committee is yet developing this part of ISA95. This technical specification defines an object
model that determines the information exchanged
between MES Activities (defined in Part 3 of ISA95). The model and attributes of Part 4 form the
basis for the design and implementation of interface
standards, ensuring a flexible flow of cooperation
and information exchange between various MES
activities;
    \renewcommand{\labelitemi}{5)}
    \item ISA-95.00.05 Part 5: «Business to manufacturing
transaction (B2M transactions)». Part 5 of ISA-95
is still in development. This technical specification
defines operations among workplace and manufacturing automation structures that may be used along
with Part 1 and Part 2 item models. Operations join
and arrange the manufacturing items and activities
described withinside the preceding a part of the
standard. Such operations arise in any respect ranges
withinside the organization, however the attention of
this technical specification is at the interface among
the organization and the management system;
\renewcommand{\labelitemi}{6)}
    \item ISA-95.00.06 Part 6: «Transactions between Manufacturing Operations»;
\renewcommand{\labelitemi}{7)}
    \item ISA-95.00.07 Part 7: «Model of Service Messages»;
\renewcommand{\labelitemi}{8)}
    \item ISA-95.00.08 Part 8: «Profiles of Information Exchange».
\end{itemize} \par
Models help define boundaries between business and
control systems. They help answer questions about which
functions can perform which tasks and what information
must be exchanged between applications. \par
The first phase of building a digital twin model requires
\end{multicols}
\end{document}
