\documentclass{article}
\usepackage[T2A]{fontenc}
\usepackage[utf8]{inputenc}
\usepackage[left=10mm, top=9mm, right=10mm, bottom=10mm, nohead, nofoot]{geometry}
\usepackage{enumitem}
\usepackage{float}
\usepackage{multicol}
\usepackage[justification=centering]{caption}
\setcounter{page}{277}

\setlist[itemize]{itemsep=0pt, parsep=0pt, partopsep=0pt, topsep=1.4pt}


\begin{document}
\begin{multicols}{2}
\fontsize{9}{11}\selectfont
\leftskip = 0.5cm [9] Jiang. H., Zhao. Y., Li. X. Intelligent education system: A comprehensive review. IEEE Access, 2020, Vol. 8, pp. 201016- 201026.

[10] Li. H. Research on item automatic generation based on DL and domain ontology. Journal of Changchun University of Technology (Natural Science Edition), 2012, Vol. 33(04), pp. 461-464.

[11] Kaur. M., Batra. S. Ontology-based question generation: a review. Journal of Intelligent and Fuzzy Systems, 2020, Vol. 39(01), pp. 49-57.

[12] Andreas. P., Konstantinos. K., Konstantinos. K. Automatic generation of multiple choice questions from domain ontologies. In: IADIS International Conference e-Learning, 2008, pp. 427-434.

[13] Arjun. S. B., Manas. K., Sujan. K. S. Automatic Generation of Multiple Choice Questions Using Wikipedia. International Conference on Pattern Recognition and Machine Intelligence, 2013, Vol. 8251, pp. 733-738.

[14] Chen. C. H. A comprehensive survey of text similarity measures. Journal of Information Science, 2019, Vol. 45(01), pp. 1-19.

[15] Singh. A., Singh. V. K. Text similarity techniques: a systematic literature review. International Journal of Computer Applications, 2019, Vol. 182(41), pp. 23-30.

[16] Baly. R., Jansen. P., De. Rijke. M. A review of evaluation methodologies in automatic short answer grading. Educational Research Review, 2014, Vol. 13, pp. 20-33.

[17] Kim. Y., Kwon. O. N., Kim. Y. Automatic short answer grading using latent semantic analysis with fuzzy matching. Journal of Educational Technology and Society, 2016, Vol. 19(03), pp. 257-269.

[18] Mingyu. Ji., Xinhai. Zhang. A Short Text Similarity Calculation Method Combining Semantic and Headword Attention Mechanism. Scientific Programming, 2022.

[19] Anderson. P., Fernando. B., Johnson. M., Gould. S. Spice: Semantic propositional image caption evaluation. In: European Conference on Computer Vision, Springer, 2016, pp. 382-398.

[20] Shu. X., Wang. S., Guo. L., Zhao. Y. Survey on multi-source information fusion: progress and challenges. Information Fusion, 2018, Vol. 42, pp. 146-169.

[21] Wang. X. Y., Hu. Z. W., Bai. R. J., et al. Review on Concepts, Processes, Tools and Methods Ontology Integration. Library and Information Service, 2011, Vol. 55(16), pp. 119-125.

[22] Liu. J., Liu. Y., Wang. X., Wang. B. A survey on ontology integration techniques. Future Generation Computer Systems, 2020, Vol. 105, pp. 17-30.

[23] Fujiwara. M., Kurahashi. T. Prenex normal form theorems in semi-classical arithmetic. The Journal of Symbolic Logic, 2022, pp. 1-31.

[24] Pan. M., Ding. Z. A simple method for solving prenex disjunction (conjunction) normal forms. Computer Engineering and Science, 2008, Vol. 30(10), pp. 82-84.

[25] Ge. L., Li. X. An overview of the methods for automated theorem proving in geometry. Journal of Symbolic Computation, 2016, Vol. 74, pp. 303-335.

[26] Kryakin. A., Nguyen. P. Automated theorem proving in geometry using Grobner bases. Journal of Automated Reasoning, 2014, Vol. ¨ 53(02), pp. 163-189.

[27] Reiter. R. A Logic for Default Reasoning. Artificial Intelligence, 1980, Vol. 13(1-2), pp. 81-132.

[28] Chou. S. C., Gao. X. S., Zhang. J. Machine Proofs in Geometry: Automated Production of Readable Proofs for Geometry Theorems. World Scientific, 1994.

\begin{center}
\fontsize{12}{11}\selectfont
\textbf{Автоматизированный подход к проверке уровня знаний пользователей в интеллектуальных обучающих системах}

\vspace{0.4cm}
Ли Вэньцзу
\end{center}
Данная работа посвящена проблеме автоматизации реализации быстрого тестирования знаний пользователей в интеллектуальных обучающих системах нового поколения. В данной работе подробно описывается основанный на семантике подход к автоматизации всего процесса от генерации тестовых вопросов и экзаменационных билетов до автоматической проверки ответов пользователей и автоматической оценки экзаменационных билетов.

На протяжении многих лет педагоги активно высказывают желание использовать компьютеры для автоматизации обучения и преподавания. С развитием технологии искусственного интеллекта в последние годы, это желание может наконец-то стать реальностью. Наиболее представительным продуктом, объединяющим искусственный интеллект и образование, являются интеллектуальные обучающие системы (ИОС), которые могут не только значительно повысить эффективность обучения пользователей, но и обеспечить справедливость и беспристрастность образовательного процесса. 

Автоматическая генерация тестовых вопросов и автоматическая проверка ответов пользователей являются самыми основными и важными функциями ИОС. Использование этих двух функций в комбинации позволит реализовать весь процесс от автоматической генерации тестовых вопросов до автоматической оценки экзаменационных билетов пользователей. Это не только значительно сократит повторяющуюся работу педагогов, но и снизит стоимость обучения для пользователей, что позволит большему числу людей получить доступ к различным знаниям.

Хотя в последние годы благодаря развитию таких технологий, как семантические сети, глубокое обучение и обработка естественного языка (NLP), было предложено несколько
подходов для автоматической генерации тестовых вопросов и проверки ответов пользователей, эти методы все еще имеют следующие основные недостатки:
\begin{itemize}
    \item существующие подходы к генерации тестовых вопросов позволяют генерировать толькj самые простые объективные вопросы;
    \item  некоторые из существующих подходов (например, сопоставление ключевых слов и использование статистической вероятности) для проверки ответов пользователей на субъективные вопросы не учитывают семантическое сходство между ответами;
    \item методы, использующие семантику для проверки ответов пользователей на субъективные вопросы, могут вычислять сходство только между ответами с простыми семантическими структурами;
    \item и т.д.
\end{itemize}

Поэтому на основе существующих методов и Технологии OSTIS в данной работе предлагается подход к разработке универсальной подсистемы для автоматической генерации тестовых вопросов и автоматической проверки ответов пользователя в обучающих системах, разработанных с использованием Технологии OSTIS (открытая семантическая технология проектирования интеллектуальных систем).

\vspace{0.3cm}
\hspace{5cm}\textbf{Received 29.03.2023}
\end{multicols}
\end{document}
