\begin{SCn}
\begin{small}
\scnheader{автоматизация}  
\scnidtf{применение технических средств, экономико-математических методов и систем управления, освобождающих человека 
частично или полностью от непосредственного участия в процессах получения, преобразования, передачи и использования энергии, материалов или информации }
\scnidtf{automation}
\begin{scnrelfromset}{автоматизируются}
\scnitem{производственные процессы}
\scnitem{проектирование сложных агрегатов, промышленных сооружений, производственных комплексов}
\scnitem{организация в рамках цеха, предприятия, 
строительства, отрасли}
\scnitem{планирование в рамках цеха, предприятия, 
строительства, отрасли}
\scnitem{управление в рамках цеха, предприятия, 
строительства, отрасли}
\scnitem{научные исследования}
\scnitem{медицинское диагностирование}
\scnitem{техническое диагностирование}
\scnitem{программирование}
\scnitem{инженерные расчеты}
\end{scnrelfromset}
\scntext{цель автоматизации}{повышение производительности и эффективности труда, улучшение качества продукции, устранение человека от работы в условиях, опасных для здоровья}
\begin{scnrelfromset}{основные виды автоматизации}
\scnitem{автоматический контроль}
\scnitem{автоматическая защита}
\scnitem{автоматическое 
управление}
\end{scnrelfromset}
\scnheader{автоматический контроль} 
\scnidtf{automatic control}
\begin{scnrelfromset}{разбиение}
\scnitem{автоматическая сигнализация}
\scnitem{автоматическое измерение}
\scnitem{автоматическая сортировка}
\scnitem{автоматический сбор информации}
\end{scnrelfromset}
\scnheader{автоматическая сигнализация} 
\scnidtf{automatic alarm}
\scntext{пояснение}{предназначен для получения информации о ходе технологического процесса, о качестве и количестве выпускаемой продукции и для дальнейшей обработки, хранения и выдачи информации обслуживающему персоналу.}
\begin{scnrelfromset}{сигнальные устройства}
\scnitem{лампы}
\scnitem{звонки}
\scnitem{сирены}
\scnitem{специальные мнемонические указатели}
\end{scnrelfromset}
\scnsubset{автоматический контроль}
\scnheader{автоматическое измерение} 
\scnidtf{automatic measurement}
\scntext{пояснение}{позволяет измерять и передавать на специальные указательные или регистрирующие приборы значения физических величин, характеризующих технологический процесс или работу машин}
\scnsubset{автоматический контроль}
\scnheader{автоматическая сортировка} 
\scnidtf{automatic sorting}
\scntext{пояснение}{осуществляет контроль и разделение продуктов по размеру, весу, твердости, вязкости и другим показателям}
\scnsubset{автоматический контроль}
\scnheader{автоматический сбор информации} 
\scnidtf{automatic collection of information}
\scntext{пояснение}{предназначен для получения информации о ходе ТП, о качестве и количестве выпускаемой продукции и для дальнейшей обработки, хранения и выдачи информации обслуживающему персоналу.}
\scnsubset{автоматический контроль}
\scnheader{автоматическая защита} 
\scnidtf{automatic protection}
\scntext{пояснение}{представляет собой совокупность технических средств, которые при возникновении ненормальных и аварийных режимов прекращают контролируемый производственный процесс}
\scnheader{автоматическое управление} 
\scnidtf{automatic control}
\scntext{пояснение}{включает комплекс технических средств 
и методов по управлению, обеспечивающих пуск и остановку основных и 
вспомогательных устройств, безаварийную работу, соблюдение требуемых 
значений параметров в соответствии с оптимальным ходом технологического 
процесса.}
\scnheader{система автоматического управления ТП} 
\scnidtf{TP automatic control system}
\scnidtf{САУ ТП}
\scnidtf{сочетание комплекса технических устройств с объектом управления.}
\begin{scnrelfromlist}{авторы}
    \scnitem{Фурсенко С.Н.}
    \scnitem{Якубовская Е.С.}
    \scnitem{Волкова Е.С.}
\end{scnrelfromlist}
\scnrelfrom{библиографический источник}{Автоматизация технологических процессов}
\end{small}
\end{SCn}