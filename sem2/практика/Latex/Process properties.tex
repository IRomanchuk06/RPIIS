\begin{SCn}
\begin{small}
\scnheader{точность}  
\scnidtf{степень соответствия параметров изготовленного изделия тем параметрам, которые указаны в нормативно-технологической документации}  
\scnidtf{accuracy}  
\scnheader{стабильность}  
\scnidtf{свойство технологического процесса сохранять значения показателей качества продукции в заданных границах на протяжении определенного времени}  
\scnidtf{stability}  
\scnheader{надёжность}  
\scnidtf{способность технологического процесса обеспечить изготовление годных изделий, как по точности отдельных параметров, так и по комплексу физикохимических характеристик}  
\scnidtf{reliability}  
\scnheader{уровень автоматизации}  
\scnidtf{это формализованная, выраженная числом степень роботизации выполняемых операций или независимость автоматизированного комплекса от человека}  
\scnidtf{automation level}  
\scnheader{уровень выхода годной продукции}  
\scnidtf{это формализованная, выраженная числом степень роботизации выполняемых операций или независимость автоматизированного комплекса от человека}  
\scnidtf{level of yield of suitable products}  
\scnheader{патентная чистота}  
\scnidtf{юридическое свойство объекта (устройства, изделия, вещества, технологии и т. д.), характеризующее возможность его использования без нарушения прав на действующие патенты других лиц на территории определенной страны или региона}  
\scnidtf{patent purity}  
\scnheader{материалоёмкость}  
\scnidtf{расход материалов в расчете на натуральную единицу или на единицу стоимости выпускаемой продукции. Материалоемкость измеряется в физических единицах, в денежном выражении или в процентах, которые составляют стоимость материалов в общих издержках производства продукции, в себестоимости}  
\scnidtf{material intensity}  
\scnheader{металлоёмкость}  
\scnidtf{количество металла, расходуемое на изготовление определённой машины, механизма, строительной конструкции}  
\scnidtf{metal intensity}  
\scnheader{энергоёмкость}  
\scnidtf{величина потребления энергии и (или) топлива на основные и вспомогательные технологические процессы изготовления продукции, выполнение работ, оказание услуг на базе заданной технологической системы}
\scnidtf{energy intensity}  
\scnheader{производительность}  
\scntext{пояснение}{свойство технологического процесса обеспечивать выпуск определенного количества изделий на протяжении указанного промежутка времени. Различают производительность часовую, сменную, месячную}
\scnidtf{performance}  
\scnheader{себестоемость}  
\scnidtf{стоимостная оценка текущих затрат предприятия на производство и реализацию продукции.}
\scnidtf{cost price}
\scnsuperset{технологическая себестоимость}
\scnheader{технологическая себестоимость}  
\scnidtf{часть производственной себестоимости изделия, которая непосредственно связанна с конкретным технологическим процессом и конструкцией этого изделия.}
\scnidtf{technological cost}  
\scnsubset{себестоемость}
\scnheader{трудоёмкость}  
\scnidtf{трудоемкость - количество рабочего времени человека, затрачиваемого на производство единицы продукции.}
\scnidtf{laboriousness} 
\scnsuperset{технологическая трудоёмкость}
\scnheader{технологическая трудоёмкость}  
\scnidtf{технологическая трудоемкость - это затраты труда рабочих, осуществляющих технологическое воздействие на предметы труда, учитываемые в товарной продукции предприятия.}
\scnidtf{technological laboriousness}  
\scnsubset{трудоёмкость}
\scnheader{экономичность}  
\scnidtf{себестоимость изготовления детали.}
\scnheader{взрывобезопасность} 
\scnidtf{состояние производственного процесса, при котором исключается возможность взрыва или, в случае его возникновения, предотвращается воздействие на людей избыточного давления в ударной волне, скоростного напора воздуха и др.}
\scnidtf{explosion-proof}
\scnheader{технологический процесс}
\scnidtf{т.п.}
\scnidtf{технологический цикл}
\scnidtf{т.ц.}
\scnidtf{установленная технологическими документами последовательность взаимосвязанных действий, направленных на объект процесса с целью получения требуемого конечного результата}
\begin{scnrelfromset}{свойства}
     \scnitem{технические свойства}
        \begin{scnindent}
\begin{scnrelfromset}{разбиение}
    \scnitem{точность}  
    \scnitem{стабильность}
    \scnitem{надёжность}
    \scnitem{уровень автоматизации}
    \scnitem{быстродействие}
    \scnitem{контролируемость}
    \scnitem{уровень выхода годной продукции}
    \scnitem{патентная чистота}
\end{scnrelfromset}
\end{scnindent}
\scnitem{экономические свойства}
\begin{scnindent}
\begin{scnrelfromset}{разбиение}
    \scnitem{материалоёмкость}
    \scnitem{металлоёмкость}
    \scnitem{энергоёмкость}
    \scnitem{производительность}
    \scnitem{технологическая трудоёмкость}
    \scnitem{технологическая себестоимость}
    \scnitem{экономичность}
\end{scnrelfromset}
\end{scnindent}
\scnitem{эргономические и эстетические свойства}
\begin{scnindent}
\begin{scnrelfromset}{разбиение}
    \scnitem{удобство обслуживания и управления}
    \scnitem{гигиеничность}
\end{scnrelfromset}
\end{scnindent}
\scnitem{безопасность}
\begin{scnindent}
\begin{scnrelfromset}{разбиение}
    \scnitem{уровень токсичности}
    \scnitem{уровень шума}
    \scnitem{взрывобезопасность}
    \scnitem{степень загрязнения окружающей среды}
\end{scnrelfromset}
\end{scnindent}
\end{scnrelfromset}
\scntext{пояснение}{качество технологических процессов можно оценивать по совокупности различных свойств. В основе такой методики лежит использование безразмерного обобщенного показателя, учитывающего всю совокупность необходимых потребителю свойств технологического процесса. В качестве такого показателя принимается обобщенная функция желательности Харрингтона.}
\scnheader{шкала желательности Харрингтона}
\scnidtf{количественный, однозначный, единый, универсальный показатель качества объекта, как параметра оптимизации.}
\scntext{назначение}{установление соответствия между полученными значениями показателей свойств и оценками экспериментатора желательности того или иного показателя для функции органа, системы и в целом организма человека}
\scnrelfrom{стандартные отметки по шкале желательности}{\scnfileimage[35em]{graphics/tab.png}}

\scnrelfrom{расчёт функции желательности}{\scnfileimage[15em]{graphics/formula.png}}
\scntext{пояснение}{Обобщенная функция желательности D рассчитывается как среднее геометрическое из частных функций желательности d с учетом значимости каждого свойства.}
\scntext{переменные}{\textsl{u}-номер свойства в ранжированной последовательности свойств;
n-число свойств технологического процесса;
$\beta_{\textit{\textsl{u}}}$-коэффициент весомости (показатель значимости) свойства технологического процесса}
\scnheader{частная функция желательности}
\scnidtf{Частная функция желательность это значение частного показателя, переведенного в безразмерную шкалу желательности}
\scntext{пояснение}{Шкала желательности имеет интервал от d=0, что соответствует неприемлемому уровню данного свойства, до d=1, что означает самое лучшее значение свойства.}
\scnrelfrom{математическая зависимость оценки от показателя свойства}{\scnfileimage[15em]{graphics/formula2.png}}
\scntext{переменные}{y-кодированное значение частного показателя, то есть его значение в условном масштабе}
\scntext{пояснение}{Коэффициенты весомости показателей свойств технологического процесса определяются экспертным опросом по методу рангов. Составляется ранжированный ряд свойств технологического процесса в порядке возрастания суммы рангов.}
\scnrelfrom{показатель значимости свойства}{\scnfileimage[15em]{graphics/formula31.png}}
\scnrelfrom{показатель значимости совйтсва}{\scnfileimage[35em]{graphics/tab2.png}}
\scntext{пояснение}{Качество технологического процесса будет тем выше, чем большее значение имеет обобщенная функция желательности.На основании анализа результатов можно сделать вывод о качестве технологического процесса, возможности его улучшения.}
\end{small}
\end{SCn}