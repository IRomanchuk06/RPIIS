\section*{Введение}
\addcontentsline{toc}{section}{Введение}

\textbf{Цель:}

Закрепить практические навыки формализации информации в интеллектуальных системах с использованием семантических сетей.
\bigskip

\textbf{Задачи:}

\begin{itemize}
    \item Формализовать методику и средства разработки естественно-языковых интерфейсов.
    \item Построение формальной семантической спецификации библиографических источников, соответствующих указанным выше фрагментам.
    \item Оформление конкретных предложений по развитию текущей версии Стандарта интеллектуальных компьтерных систем и технологий их разработки.
\end{itemize}

\newpage
\section{Постановка задачи}
%Пример

\begin{SCn}
\begin{small}

\scnheader{Часть 2 Учебной дисциплины "Представление и обработка информации в интеллектуальных системах"{}}
\begin{scnrelfromlist}{библиографическая ссылка}
    \scnitem{Стандарт OSTIS}
    \scnitem{Материалы конференций OSTIS}
    \scnitem{Толковый словарь по Искусственному интеллекту}
    \begin{scnindent}
        \scntext{URL}{http://raai.org/library/tolk/aivoc.html}     
     \end{scnindent}    
        \scnitem{Кравченко Е.Г..МетодОКТП-2016ст }
    \begin{scnindent}
        \scntext{URL}{https://cyberleninka.ru/article/n/metodika-otsenki-kachestva-tehnologicheskih-protsessov/viewer}     
     \end{scnindent}        
        \scnitem{Фурсенко С.Н..АвтомТП-2007кн }
    \begin{scnindent}     
        \scntext{URL}{https://rep.bsatu.by/bitstream/doc/135/1/Fursenko-S-N-Avtomatizaciya-tekhnologicheskih-processov.pdf}    
    \end{scnindent}
\end{scnrelfromlist}

\end{small}
\end{SCn}

\newpage
\section{Формализованные фрагменты теории интеллектуальных компьютерных систем и технологий их разработки}
\begin{small}
\scnheader{Глава 7.7}
\scntext{введение}{Современное направление конвергенции работ в области создания интеллектуальных систем (см. Голенков В.В..оОбучеИСкОС-2018ст ) требует разработки соответствующего программного обеспечения с элементами
когнитивных способностей на основе семантически совместимых технологий искусственного интеллекта. Концепция Industry 4.0 предполагает построение единой онтологической модели предприятия (см. Taberko V.V..Desig
oBMEitC-2018art), включающей в себя описание оборудования и технологических процессов производства. Технология OSTIS предоставляет средства для построения такой модели, обеспечивает возможность построения
“цифрового двойника” предприятия на основе формализованного описания соответствующей предметной области. Полученная онтологическая модель, таким образом, может выступать основой интеграции востребованных
интеллектуальных решений по автоматизации и информационному обеспечению производственной деятельности.
В первой части главы рассматривается подход к построению системы адаптивного управления технологическим
процессом производства в виде решателя соответствующей ostis-системы на основе онтологии предметной области “технологические процессы производства с вероятностными характеристиками”. В основу функционирования
предлагаемой системы положено применение нейросетевых контроллеров. В основу формализации контура управления и математических моделей объекта исследования положены результаты научных разработок авторов в
области имитационного моделирования сложных технических систем (см. Смородин В.С..Метод иСИМТ-2007кн).
Такая реализация позволяет обеспечить возможность интеграции предлагаемого решения с другими разработками,
программными средствами предприятия для обеспечения построения интеллектуальных систем автоматизированного управления, рекомендательных систем и систем поддержки принятия решений, систем информационного
обеспечения персонала предприятия.
Во второй части главы рассматриваются вопросы построения онтологической модели предприятия на примере предметной области “рецептурные производства” с применением общепринятых международных стандартов описания
содержания производственной деятельности предприятия. Формализация стандартов является основой подхода
к проектированию предприятия, в ее процессе требуется принять во внимание сложную специфику предметной
области, возможность неоднозначной трактовки положений и необходимость обеспечения актуализации используемых стандартов. Онтологический подход к построению умных предприятий в рамках Экосистемы OSTIS показан
на примере формализованного описания физической модели предприятия ОАО “Савушкин продукт” и построения
системы автоматизации деятельности инженера-технолога.}
\section*{\S 7.7.1 Адаптивное управление технологическим циклом производства на основе
Технологии OSTIS}
\end{small}
\begin{small}
\section*{Методика оценки качества технологических процессов}
\end{small}
\begin{SCn}
\begin{small}
\scnheader{точность}  
\scnidtf{степень соответствия параметров изготовленного изделия тем параметрам, которые указаны в нормативно-технологической документации}  
\scnidtf{accuracy}  
\scnheader{стабильность}  
\scnidtf{свойство технологического процесса сохранять значения показателей качества продукции в заданных границах на протяжении определенного времени}  
\scnidtf{stability}  
\scnheader{надёжность}  
\scnidtf{способность технологического процесса обеспечить изготовление годных изделий, как по точности отдельных параметров, так и по комплексу физикохимических характеристик}  
\scnidtf{reliability}  
\scnheader{уровень автоматизации}  
\scnidtf{это формализованная, выраженная числом степень роботизации выполняемых операций или независимость автоматизированного комплекса от человека}  
\scnidtf{automation level}  
\scnheader{уровень выхода годной продукции}  
\scnidtf{это формализованная, выраженная числом степень роботизации выполняемых операций или независимость автоматизированного комплекса от человека}  
\scnidtf{level of yield of suitable products}  
\scnheader{патентная чистота}  
\scnidtf{юридическое свойство объекта (устройства, изделия, вещества, технологии и т. д.), характеризующее возможность его использования без нарушения прав на действующие патенты других лиц на территории определенной страны или региона}  
\scnidtf{patent purity}  
\scnheader{материалоёмкость}  
\scnidtf{расход материалов в расчете на натуральную единицу или на единицу стоимости выпускаемой продукции. Материалоемкость измеряется в физических единицах, в денежном выражении или в процентах, которые составляют стоимость материалов в общих издержках производства продукции, в себестоимости}  
\scnidtf{material intensity}  
\scnheader{металлоёмкость}  
\scnidtf{количество металла, расходуемое на изготовление определённой машины, механизма, строительной конструкции}  
\scnidtf{metal intensity}  
\scnheader{энергоёмкость}  
\scnidtf{величина потребления энергии и (или) топлива на основные и вспомогательные технологические процессы изготовления продукции, выполнение работ, оказание услуг на базе заданной технологической системы}
\scnidtf{energy intensity}  
\scnheader{производительность}  
\scntext{пояснение}{свойство технологического процесса обеспечивать выпуск определенного количества изделий на протяжении указанного промежутка времени. Различают производительность часовую, сменную, месячную}
\scnidtf{performance}  
\scnheader{себестоемость}  
\scnidtf{стоимостная оценка текущих затрат предприятия на производство и реализацию продукции.}
\scnidtf{cost price}
\scnsuperset{технологическая себестоимость}
\scnheader{технологическая себестоимость}  
\scnidtf{часть производственной себестоимости изделия, которая непосредственно связанна с конкретным технологическим процессом и конструкцией этого изделия.}
\scnidtf{technological cost}  
\scnsubset{себестоемость}
\scnheader{трудоёмкость}  
\scnidtf{трудоемкость - количество рабочего времени человека, затрачиваемого на производство единицы продукции.}
\scnidtf{laboriousness} 
\scnsuperset{технологическая трудоёмкость}
\scnheader{технологическая трудоёмкость}  
\scnidtf{технологическая трудоемкость - это затраты труда рабочих, осуществляющих технологическое воздействие на предметы труда, учитываемые в товарной продукции предприятия.}
\scnidtf{technological laboriousness}  
\scnsubset{трудоёмкость}
\scnheader{экономичность}  
\scnidtf{себестоимость изготовления детали.}
\scnheader{взрывобезопасность} 
\scnidtf{состояние производственного процесса, при котором исключается возможность взрыва или, в случае его возникновения, предотвращается воздействие на людей избыточного давления в ударной волне, скоростного напора воздуха и др.}
\scnidtf{explosion-proof}
\scnheader{технологический процесс}
\scnidtf{т.п.}
\scnidtf{технологический цикл}
\scnidtf{т.ц.}
\scnidtf{установленная технологическими документами последовательность взаимосвязанных действий, направленных на объект процесса с целью получения требуемого конечного результата}
\begin{scnrelfromset}{свойства}
     \scnitem{технические свойства}
        \begin{scnindent}
\begin{scnrelfromset}{разбиение}
    \scnitem{точность}  
    \scnitem{стабильность}
    \scnitem{надёжность}
    \scnitem{уровень автоматизации}
    \scnitem{быстродействие}
    \scnitem{контролируемость}
    \scnitem{уровень выхода годной продукции}
    \scnitem{патентная чистота}
\end{scnrelfromset}
\end{scnindent}
\scnitem{экономические свойства}
\begin{scnindent}
\begin{scnrelfromset}{разбиение}
    \scnitem{материалоёмкость}
    \scnitem{металлоёмкость}
    \scnitem{энергоёмкость}
    \scnitem{производительность}
    \scnitem{технологическая трудоёмкость}
    \scnitem{технологическая себестоимость}
    \scnitem{экономичность}
\end{scnrelfromset}
\end{scnindent}
\scnitem{эргономические и эстетические свойства}
\begin{scnindent}
\begin{scnrelfromset}{разбиение}
    \scnitem{удобство обслуживания и управления}
    \scnitem{гигиеничность}
\end{scnrelfromset}
\end{scnindent}
\scnitem{безопасность}
\begin{scnindent}
\begin{scnrelfromset}{разбиение}
    \scnitem{уровень токсичности}
    \scnitem{уровень шума}
    \scnitem{взрывобезопасность}
    \scnitem{степень загрязнения окружающей среды}
\end{scnrelfromset}
\end{scnindent}
\end{scnrelfromset}
\scntext{пояснение}{качество технологических процессов можно оценивать по совокупности различных свойств. В основе такой методики лежит использование безразмерного обобщенного показателя, учитывающего всю совокупность необходимых потребителю свойств технологического процесса. В качестве такого показателя принимается обобщенная функция желательности Харрингтона.}
\scnheader{шкала желательности Харрингтона}
\scnidtf{количественный, однозначный, единый, универсальный показатель качества объекта, как параметра оптимизации.}
\scntext{назначение}{установление соответствия между полученными значениями показателей свойств и оценками экспериментатора желательности того или иного показателя для функции органа, системы и в целом организма человека}
\scnrelfrom{стандартные отметки по шкале желательности}{\scnfileimage[35em]{graphics/tab.png}}

\scnrelfrom{расчёт функции желательности}{\scnfileimage[15em]{graphics/formula.png}}
\scntext{пояснение}{Обобщенная функция желательности D рассчитывается как среднее геометрическое из частных функций желательности d с учетом значимости каждого свойства.}
\scntext{переменные}{\textsl{u}-номер свойства в ранжированной последовательности свойств;
n-число свойств технологического процесса;
$\beta_{\textit{\textsl{u}}}$-коэффициент весомости (показатель значимости) свойства технологического процесса}
\scnheader{частная функция желательности}
\scnidtf{Частная функция желательность это значение частного показателя, переведенного в безразмерную шкалу желательности}
\scntext{пояснение}{Шкала желательности имеет интервал от d=0, что соответствует неприемлемому уровню данного свойства, до d=1, что означает самое лучшее значение свойства.}
\scnrelfrom{математическая зависимость оценки от показателя свойства}{\scnfileimage[15em]{graphics/formula2.png}}
\scntext{переменные}{y-кодированное значение частного показателя, то есть его значение в условном масштабе}
\scntext{пояснение}{Коэффициенты весомости показателей свойств технологического процесса определяются экспертным опросом по методу рангов. Составляется ранжированный ряд свойств технологического процесса в порядке возрастания суммы рангов.}
\scnrelfrom{показатель значимости свойства}{\scnfileimage[15em]{graphics/formula31.png}}
\scnrelfrom{показатель значимости совйтсва}{\scnfileimage[35em]{graphics/tab2.png}}
\scntext{пояснение}{Качество технологического процесса будет тем выше, чем большее значение имеет обобщенная функция желательности.На основании анализа результатов можно сделать вывод о качестве технологического процесса, возможности его улучшения.}
\end{small}
\end{SCn}
\section*{\S 7.7.2 Построение умных предприятий рецептурного производства с помощью
ostis-систем}
\section*{Автоматизация технологических процессов}
\begin{SCn}
\begin{small}
\scnheader{автоматизация}  
\scnidtf{применение технических средств, экономико-математических методов и систем управления, освобождающих человека 
частично или полностью от непосредственного участия в процессах получения, преобразования, передачи и использования энергии, материалов или информации }
\scnidtf{automation}
\begin{scnrelfromset}{автоматизируется}
\scnitem{производственный процесс}
\scnitem{проектирование сложных агрегатов, промышленных сооружений, производственных комплексов}
\scnitem{организация в рамках цеха, предприятия, 
строительства, отрасли}
\scnitem{планирование в рамках цеха, предприятия, 
строительства, отрасли}
\scnitem{управление в рамках цеха, предприятия, 
строительства, отрасли}
\scnitem{научное исследование}
\scnitem{медицинское диагностирование}
\scnitem{техническое диагностирование}
\scnitem{программирование}
\scnitem{инженерный расчет}
\end{scnrelfromset}
\scntext{цель автоматизации}{повышение производительности и эффективности труда, улучшение качества продукции, устранение человека от работы в условиях, опасных для здоровья}
\begin{scnrelfromset}{основные виды автоматизации}
\scnitem{автоматический контроль}
\scnitem{автоматическая защита}
\scnitem{автоматическое 
управление}
\end{scnrelfromset}
\scnheader{автоматический контроль} 
\scnidtf{automatic control}
\begin{scnrelfromset}{разбиение}
\scnitem{автоматическая сигнализация}
\scnitem{автоматическое измерение}
\scnitem{автоматическая сортировка}
\scnitem{автоматический сбор информации}
\end{scnrelfromset}
\scnheader{автоматическая сигнализация} 
\scnidtf{automatic alarm}
\scntext{пояснение}{автоматическая сигнализация предназначена для получения информации о ходе технологического процесса, о качестве и количестве выпускаемой продукции и для дальнейшей обработки, хранения и выдачи информации обслуживающему персоналу.}
\begin{scnrelfromset}{сигнальные устройства}
\scnitem{лампы}
\scnitem{звонки}
\scnitem{сирены}
\scnitem{специальные мнемонические указатели}
\end{scnrelfromset}
\scnsubset{автоматический контроль}
\scnheader{автоматическое измерение} 
\scnidtf{automatic measurement}
\scntext{пояснение}{позволяет измерять и передавать на специальные указательные или регистрирующие приборы значения физических величин, характеризующих технологический процесс или работу машин}
\scnsubset{автоматический контроль}
\scnheader{автоматическая сортировка} 
\scnidtf{automatic sorting}
\scntext{пояснение}{осуществляет контроль и разделение продуктов по размеру, весу, твердости, вязкости и другим показателям}
\scnsubset{автоматический контроль}
\scnheader{автоматический сбор информации} 
\scnidtf{automatic collection of information}
\scntext{пояснение}{предназначен для получения информации о ходе ТП, о качестве и количестве выпускаемой продукции и для дальнейшей обработки, хранения и выдачи информации обслуживающему персоналу.}
\scnsubset{автоматический контроль}
\scnheader{автоматическая защита} 
\scnidtf{automatic protection}
\scntext{пояснение}{представляет собой совокупность технических средств, которые при возникновении ненормальных и аварийных режимов прекращают контролируемый производственный процесс}
\scnheader{автоматическое управление} 
\scnidtf{automatic control}
\scntext{пояснение}{включает комплекс технических средств 
и методов по управлению, обеспечивающих пуск и остановку основных и 
вспомогательных устройств, безаварийную работу, соблюдение требуемых 
значений параметров в соответствии с оптимальным ходом технологического 
процесса.}
\scnheader{система автоматического управления ТП} 
\scnidtf{TP automatic control system}
\scnidtf{САУ ТП}
\scnidtf{сочетание комплекса технических устройств с объектом управления.}
\end{small}
\end{SCn}
\newpage
\section{Формальная семантическая спецификация библиографических источников}
\begin{SCn}
\begin{small}

\scnheader{Гусева А.В.ГеоинфС-2013ст}
\begin{scnrelfromlist}{ключевой знак}
    \scnitem{геоинформационные системы}
    \scnitem{пространственно-распределенная информация}
    \scnitem{геопространство}
    \scnitem{геоинформация}
    \scnitem{пространственный запрос}
\end{scnrelfromlist}
\scntext{аннотация}{Описаны геоинформационные системы, области их применения и перспективы развития. Предполагается использование ГИС с целью ознакомления читателя с новейшими средствами IT-коммуникаций.}
\scntext{цитата}{Пространственно-распределенная информация -- это то, с чем человек сталкивается практически каждый день вне зависимости от рода своей деятельности. Это может быть схема метро или план здания, топографическая карта или схема взаимосвязей между офисами компании, атлас автомобильных дорог или контурная карта и многое другое. ГИС дает возможность накапливать и анализировать подобную информацию, оперативно находить нужные сведения и отображать их в удобном для использования виде. Применение ГИС-технологий позволяет резко увеличить оперативность и качество работы с пространственно-распределенной информацией по сравнению с традиционными методами картографирования.}
\begin{scnindent}
    \scnrelto{пояснение}{ГИС}
\end{scnindent}
\scntext{цитата}{Геопространство - разновидность пространства, характеризующаяся протяженностью, динамичностью, структурностью, непрерывностью.}
\begin{scnindent}
    \scnrelto{пояснение}{Геопространство}
\end{scnindent}
\scntext{цитата}{Геоинформация - это координированная информация о геопространстве и его объектах в цифровой компьютерно-воспринимаемой форме, предназначенная в качестве исходного материала для моделирования геопространства.}
\begin{scnindent}
    \scnrelto{пояснение}{Геоинформация}
\end{scnindent}
\scntext{цитата}{ГИС-технология объединяет традиционные операции при работе с базами данных, такими как запрос и статистический анализ, с преимуществами полноценной визуализации и географического (пространственного) анализа, которые предоставляет карта. Возможность визуализации и пространственного анализа отличают ГИС от других информационных систем и обеспечивают уникальные возможности для ее применения в широком спектре задач.}
\begin{scnindent}
    \scnrelto{пояснение}{ГИС}
\end{scnindent}

\newpage

\scnheader{Самодумкин С.А.Next-genIGS-2022ст}
\begin{scnrelfromlist}{ключевой знак}
    \scnitem{OSTIS}
    \scnitem{интеллектуальная геоинформационная система}
    \scnitem{частная технология проектирования}
    \scnitem{онтология}
\end{scnrelfromlist}
\scntext{аннотация}{В статье рассматривается подход к строительству
интеллектуальных геоинформационных систем на базе OSTIS
Рассматривается технология. Формальная онтология синтаксиса
язык отображения задан явно, что, в свою очередь, позволяет
определение типов картографических объектов и настройка пространственного
семантические отношения; формальная онтология обозначения
задана семантика языка отображения, что, в свою очередь, позволяет
установление семантики отображения геообъектов на
карты в зависимости от типов объектов рельефа; формальный
в качестве необходимого условия задана онтология объектов рельефа
для интеграции с предметными областями в интересах ГИС.}
\scntext{цитата}{Для расширения задач, решаемых геоинформационными системами, унификации различных типов представления информации
в ГИС о пространстве, времени и Земле
необходимо интегрировать существующие веб-геосервисы и
технологии проектирования интеллектуальных систем с целью разработки
геоинформационных систем нового поколения как класса
интеллектуальных компьютерных систем, основанных на едином способе
кодирования информации и функциональной совместимости (interoperability)
что является необходимым требованием.}
\scntext{цитата}{Для решения проблем, поставленных в рамках этой статьи, предлагается
разработать сложную предметную область геоинформатики
и соответствующую онтологию объектов рельефа.}
\scntext{цитата}{Основой для построения онтологической модели объектов
рельефа является классификатор топографической информации, отображаемой на топографических картах и планах городов, разработанный
и действующий в настоящее время в Республике Беларусь NCRB
012-2007 [10].}


\end{small}
\end{SCn}


\newpage
\section*{Заключение}
\addcontentsline{toc}{section}{Заключение}

Во время работы были изучены основы формализации научных текстов и основы SCn кода. Был дополнен параграф Адаптивное управление технологическим циклом производства на основе Технологии OSTIS новыми понятиями и библиографическими источниками.
