\documentclass[a4paper,14pt,oneside,final]{report}

\usepackage[utf8]{inputenc}
\usepackage[english,main=russian]{babel}
\usepackage[T1,T2A]{fontenc}
%\usepackage{pscyr}
%\renewcommand{\rmdefault}{cmr}
\usepackage[final,hidelinks]{hyperref}
\usepackage[square,numbers,sort&compress]{natbib}
%\usepackage{times}
\usefont{T2A}{ftm}{m}{sl}
\setlength{\bibsep}{0em}
\PassOptionsToPackage{hyphens}{url}

\usepackage{hyphenat}

% Курсив и жирность для кириллицы
\usepackage{substitutefont}

\substitutefont{T2A}{\familydefault}{Tempora-TLF}
\makeatletter
\input{t2atempora-tlf.fd}
\DeclareFontShape{T2A}{Tempora-TLF}{m}{sc}{
         <-> ssub * Tempora-TLF/m/n
}{}

\addto\extrasrussian{%
  \def\equationautorefname{формула}%
  \def\figureautorefname{рисунок}%
  \def\listingautorefname{листинг}%
  \def\tableautorefname{таблица}%
}

\addto\captionsrussian{
  \renewcommand\contentsname{\centerline{\bfseries\large{\MakeUppercase{Содержание}}}}
  \renewcommand{\bibsection}{\sectioncentered*{Cписок использованных источников}}
  \renewcommand{\listingscaption}{Листинг}
}

\AtEndDocument{

  \addcontentsline{toc}{section}{Cписок использованных источников}%
}
\usepackage{extsizes}


\sloppy

\usepackage{microtype}
\newlength{\fivecharsapprox}
\setlength{\fivecharsapprox}{6ex}


\usepackage{indentfirst}
\setlength{\parindent}{\fivecharsapprox} 

\usepackage[left=3cm,top=2.0cm,right=1.5cm,bottom=2.7cm]{geometry}

\frenchspacing

\usepackage{perpage}
\MakePerPage{footnote}

\makeatletter 
\def\@makefnmark{\hbox{\@textsuperscript{\normalfont\@thefnmark)}}}
\makeatother

\usepackage[bottom]{footmisc}

\makeatletter
\renewcommand{\thesection}{\arabic{section}}
\makeatother

\setcounter{secnumdepth}{3}


% Зачем: Настраивает отступ между таблицей с содержанимем и словом СОДЕРЖАНИЕ
\usepackage{tocloft}
\setlength{\cftbeforetoctitleskip}{-1em}
\setlength{\cftaftertoctitleskip}{1em}


% Зачем: Определяет отступы слева для записей в таблице содержания.
\makeatletter
\renewcommand{\l@section}{\@dottedtocline{1}{0.5em}{1.2em}}
\renewcommand{\l@subsection}{\@dottedtocline{2}{1.7em}{2.0em}}
\makeatother


% Зачем: Работа с колонтитулами
\usepackage{fancyhdr} 
\pagestyle{fancy}


\fancyhf{} 
\fancyfoot[R]{\thepage}
\renewcommand{\footrulewidth}{0pt} 
\renewcommand{\headrulewidth}{0pt}
\fancypagestyle{plain}{ 
    \fancyhf{}
    \rfoot{\thepage}}


% Зачем: Задает стиль заголовков раздела жирным шрифтом, прописными буквами, без точки в конце
\makeatletter
\renewcommand\section{%
  \@startsection {section}{1}%
    {\fivecharsapprox}%
    {-1em \@plus -1ex \@minus -.2ex}%
    {1em \@plus .2ex}%
    {\raggedright\hyphenpenalty=10000\normalfont\bfseries\MakeUppercase}}
\makeatother


% Зачем: Задает стиль заголовков подразделов

\makeatletter
\renewcommand\subsection{
  \@startsection{subsection}{2}
    {\fivecharsapprox}
    {-1em \@plus -1ex \@minus -.2ex}
    {1em \@plus .2ex}
    {\raggedright\hyphenpenalty=10000\normalfont\normalsize\bfseries}}
\makeatother


% Зачем: Задает стиль заголовков пунктов
\makeatletter
\renewcommand\subsubsection{
  \@startsection{subsubsection}{3}%
    {\fivecharsapprox}%
    {-1em \@plus -1ex \@minus -.2ex}%
    {\z@}%
    {\raggedright\hyphenpenalty=10000\normalfont\normalsize\bfseries}}
\makeatother

% Зачем: для оформления введения и заключения, они должны быть выровнены по центру.
\makeatletter
\newcommand\sectioncentered{%
  \clearpage\@startsection {section}{1}%
    {\z@}%
    {-1em \@plus -1ex \@minus -.2ex}%
    {1em \@plus .2ex}%
    {\centering\hyphenpenalty=10000\normalfont\large\bfseries\MakeUppercase}%
    }
\makeatother



% Зачем: Задает стиль библиографии
\bibliographystyle{gost71s2003}

\usepackage[final]{graphicx}
\DeclareGraphicsExtensions{.png,.jpg}

% Зачем: Добавление подписей к рисункам
\usepackage{caption}
\usepackage{subcaption}

% Зачем: поворот ячеек таблиц на 90 градусов
\usepackage{rotating}
\DeclareRobustCommand{\povernut}[1]{\begin{sideways}{#1}\end{sideways}}

\DeclareRobustCommand{\x}[1]{\text{#1}}

% Зачем: Задание подписей, разделителя и нумерации частей рисунков
\DeclareCaptionLabelFormat{stbfigure}{Рисунок \emph{#2}}
\DeclareCaptionLabelFormat{stbtable}{Таблица \emph{#2}}
\DeclareCaptionLabelFormat{stblisting}{Листинг \emph{#2}}
\DeclareCaptionLabelSeparator{stb}{~--~}
\captionsetup{labelsep=stb}
\captionsetup[figure]{labelformat=stbfigure, justification=centering, font={small}}
\captionsetup[listing]{labelformat=stblisting,justification=centering, font={small}}
\captionsetup[table]{labelformat=stbtable,justification=raggedright,  font={small}}
\renewcommand{\thesubfigure}{\asbuk{subfigure}}
% Зачем: Окружения для оформлени<я формул
\usepackage{calc}
\newlength{\lengthWordWhere}
\settowidth{\lengthWordWhere}{где}
%\newenvironment{explanation}
 %   {
 %   \begin{itemize}[leftmargin=0cm, itemindent=\lengthWordWhere + \labelsep , labelsep=\labelsep]

 %   \renewcommand\labelitemi{}
    %}
    %{
    %\\[\parsep]
    %\end{itemize}
    %}


\usepackage{tabularx}

\newenvironment{explanationx}
    {
    \noindent 
    \tabularx{\textwidth}{@{}ll@{ --- } X }
    }
    { 
    \\[\parsep]
    \endtabularx
    }

\usepackage{amsmath}

\usepackage{amsfonts}
\usepackage{amssymb}
\usepackage{amsthm}
\usepackage{calc}
\usepackage{fp}
\usepackage{enumitem}

\makeatletter
 \AddEnumerateCounter{\asbuk}{\@asbuk}{щ)}
\makeatother


\setlist{nolistsep}
\renewcommand{\labelenumi}{\arabic{enumi}.}
\renewcommand{\labelenumii}{\alph{enumii}.}

%\setlist[itemize,0]{itemindent=\parindent + 2.2ex,leftmargin=0ex,label=--}
%\setlist[enumerate,1]{leftmargin=4em}
%\setlist[enumerate,2]{leftmargin=0em}


% Зачем: Включение номера раздела в номер формулы. Нумерация формул внутри раздела.
\AtBeginDocument{\numberwithin{equation}{section}}

% Зачем: Включение номера раздела в номер таблицы. Нумерация таблиц внутри раздела.
\AtBeginDocument{\numberwithin{table}{section}}

% Зачем: Включение номера раздела в номер рисунка. Нумерация рисунков внутри раздела.
\AtBeginDocument{\numberwithin{figure}{section}}

% Зачем: Включение номера раздела в номер листинга. Нумерация листингов внутри раздела.
\AtBeginDocument{\numberwithin{listing}{section}}


\usepackage{makecell}
\usepackage{multirow}
\usepackage{array}


\usepackage{textcomp}

\usepackage{siunitx}
\sisetup{
  binary-units = true,
  output-decimal-marker = {,},
  per-mode = symbol,
  range-phrase = --,
}
\DeclareSIUnit{\sample}{S}


\newcommand{\ignore}[2]{\hspace{0in}#2}


\usepackage{verbatim}
\usepackage{xcolor}
\usepackage{minted}

%\AtBeginDocument{\numberwithin{lstlisting}{section}}

\usepackage[normalem]{ulem}

\renewcommand{\UrlFont}{\small\rmfamily\tt}

% Магия для подсчета разнообразных объектов в документе
\usepackage{lastpage}
\usepackage{totcount}
\regtotcounter{section}

\usepackage{etoolbox}

\newcounter{totfigures}
\newcounter{tottables}
\newcounter{totreferences}
\newcounter{totequation}

\providecommand\totfig{} 
\providecommand\tottab{}
\providecommand\totref{}
\providecommand\toteq{}

\makeatletter
\AtEndDocument{%
  \addtocounter{totfigures}{\value{figure}}%
  \addtocounter{tottables}{\value{table}}%
  \addtocounter{totequation}{\value{equation}}
  \immediate\write\@mainaux{%
    \string\gdef\string\totfig{\number\value{totfigures}}%
    \string\gdef\string\tottab{\number\value{tottables}}%
    \string\gdef\string\totref{\number\value{totreferences}}%
    \string\gdef\string\toteq{\number\value{totequation}}%
  }%
}
\makeatother

\pretocmd{\section}{\addtocounter{totfigures}{\value{figure}}\setcounter{figure}{0}}{}{}
\pretocmd{\section}{\addtocounter{tottables}{\value{table}}\setcounter{table}{0}}{}{}
\pretocmd{\section}{\addtocounter{totequation}{\value{equation}}\setcounter{equation}{0}}{}{}
\pretocmd{\bibitem}{\addtocounter{totreferences}{1}}{}{}



% Для оформления таблиц не влязящих на 1 страницу
\usepackage{longtable}

\usepackage{gensymb}

% Зачем: преобразовывать текст в верхний регистр командой MakeTextUppercase
\usepackage{textcase}

%  Переносы в словах с тире \hyph.

\def\hyph{-\penalty0\hskip0pt\relax}

% Добавляем левый отступ для библиографии

\makeatletter
\renewenvironment{thebibliography}[1]
     {\sectioncentered*{Cписок использованных источников}
      \@mkboth{\MakeUppercase\refname}{\MakeUppercase\refname}%
      \list{\@biblabel{\@arabic\c@enumiv}}%
           {\settowidth\labelwidth{\@biblabel{#1}}%
            \setlength{\itemindent}{\dimexpr\labelwidth+\labelsep+1em}
            \leftmargin\z@
            \@openbib@code
            \usecounter{enumiv}%
            \let\p@enumiv\@empty
            \renewcommand\theenumiv{\@arabic\c@enumiv}}%
      \sloppy
      \clubpenalty4000
      \@clubpenalty \clubpenalty
      \widowpenalty4000%
      \sfcode`\.\@m}
     {\def\@noitemerr
       {\@latex@warning{Empty `thebibliography' environment}}%
      \endlist}
\makeatother

\newcommand{\intro}[3]{
    \stepcounter{section}
        \sectioncentered*{ПРИЛОЖЕНИЕ \MakeUppercase{#1}}
     \begin{center} 
        \bf{(#2)}\\
        \bf{#3}
    \end{center}
    \markboth{\MakeUppercase{#1}}{}
    \addcontentsline{toc}{section}{Приложение \MakeUppercase{#1} (#2) #3}
}