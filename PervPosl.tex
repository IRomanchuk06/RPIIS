\documentclass[a4paper, 10pt,twocolumn]{article}

\usepackage[english, russian]{babel}
\usepackage[T2A]{fontenc}
\usepackage[utf8]{inputenc}

\usepackage{geometry}
\geometry{top=22mm}
\geometry{bottom=22mm}
\geometry{left=22mm}
\geometry{right=22mm}

\usepackage{enumitem}
\setlist{nolistsep}

\usepackage{soulutf8}

\newcommand{\RomanNumeralCaps}[1]
    {\MakeUppercase{\romannumeral #1}}

\title{ПервыйПоследний}
\author{Zaprudski Denis }
\date{November 2023}

\begin{document}
\twocolumn
\setcounter{page}{258}
subject area, but also train the user in knowledge and skills  from this subject area. At the same time,when developing modern computer learning systems (CLS),it became necessary to use methods and tools of artificial intelligence, which led to the emergence of a new class of CLS - intelligent learning systems. At present, due to the growing requirements for systems of this class, the problem of developing The intelligent learning systems have to become relevant, which can be characterized by the following features they should provide:
\begin{enumerate}
    \item[1)] processing large volumes of complexly structured information of various types;
    \item[2)] flexibility and easy modifiability of the system;
    \item[3)] integration of various models and mechanisms for solving problems;
    \item[4)] support for various models of learning and user interaction management;
    \item[5)] integration of various software systems within one system and management of their operation and interaction;
    \item[6)] wide use of multimedia tools;
    \item[7)] work in real time;
\end{enumerate}

Currently, there are CLSs in which these problems
are solved with the elaboration of only some individual
issues. This is due to the fact that it is not easy to solve
all the problems in a complex by the currently existing
information and intellectual means. The peculiarity of the
implementation of the learning process in ILS is that, in
addition to representing and processing knowledge about
the subject area, the system must contain information
about its users, be able to process it and thus adapt to
the individual characteristics of each specific user [10]. In
addition, one of the most important issues in the design of
a learning system is the management of a dialogue with
a user (usually untrained in computer technology). User
interaction, in contrast to the interaction of subsystems
in a computer system, is a more complex process, since
it contains an element of unpredictability.

Consider the features inherent in the initial stage of
education - general secondary education. School education is an initial, but very important stage that shapes the
entire further development of an individual’s education.
If schooling is too low, then no further steps will fix it. It
is impossible to get a good engineer out of an illiterate,
unprepared person. If there is no knowledge base, then it
is impossible to build a knowledge base of a higher level.
Therefore, we will begin our consideration of the issue of
education with school education, especially since many
problems and issues that arise in the field of education
at higher levels coincide with the problems of school
education, or are their consequences. The purpose of
education is not only and not so much to expand the
horizons and knowledge of the student, the main task is
to help him decide on the choice of a field of activity
according to his abilities and the needs of society.

An integrated approach to the formation of a strategy
for the intellectualization and digitalization of this stage
of education requires an analysis of all aspects of education - academic disciplines, pedagogical technologies
and methods, psychological characteristics of children of
different ages, from primary school students to graduates,
organizational features of the educational process, technical equipment of educational institutions, the capabilities
of telecommunication networks etc. Such an approach
should provide a deep modernization of school education
based on the use of artificial intelligence technologies,
data and systems analysis, multi-agent technologies, synergistic approaches, modeling, ontologies and semantic
technologies. The latter play a special role in the formation of student knowledge at all stages of education.
It is imperative to take into account the individuality
of the student when choosing pedagogical methods and
technologies, which will make it possible to find the
right approach to the effective teaching of students with
varying degrees of preparedness for the perception of
certain disciplines [11].
\\

The main task of the learning intellectual system is
to explain, teach and continuously unobtrusively control
the process of human learning. Here, a special role is
played by semantic intelligent systems that use semantic
relationships. Semantic intelligent systems allow, on the
one hand, to speed up the process of obtaining knowledge
based on obtaining instant access to a huge information
field, and on the other hand, intelligent systems should
help choose the best learning path, the path along which
complete knowledge will be obtained in the shortest
possible time.
\\

The socialized characteristics of the quality and quantity of knowledge of graduates of various educational
institutions include:
\\
\begin{itemize}
    \item competencies – a combination of knowledge, skills
and experience necessary for the high-quality performance of tasks;
    \item breadth of knowledge – a set of knowledge from
various fields that can complement each other and
form a single picture of the world around;
    \item depth of knowledge – a characteristic showing the
extent and the level of complexity of a person’s
knowledge on a particular issue or phenomenon. For
a correct construction of the educational process at a
particular level, it is necessary to specifically outline
the set of knowledge that a graduate must possess
in a particular academic subject;
    \item stage of education - an independent completed stage
of training and education of the education system;
    \item academic subject - a system of knowledge, skills
and abilities selected from a specific branch of
human activity.
\end{itemize}
\centerline{\RomanNumeralCaps{3}. \caps{Organization of the educational process}}

An important task in organizing an effective educational process is to fulfill the requirement that the
curricula of each year of study in subjects complement
and expand, and not duplicate the knowledge gained
in previous years of study. Curricula in subjects and
the learning process itself must comply with certain
principles (rules, requirements), which will be discussed
below.

When compiling curricula, it is necessary to take into
account the relationship between various academic disciplines. Knowledge gained in one academic discipline is
used in the study of other disciplines. At the same time,
the use of knowledge gained in other subjects allows
one to consolidate this knowledge. For example, the use
of knowledge about trigonometric functions, projections,
vectors in solving physical problems allows one to consolidate, deepen and substantiate this knowledge obtained
from mathematics.

Consideration of new material, its development in the
framework of solving problems, should contribute to the
repetition of previously covered material both in this and
other related subjects. At the initial stage, tasks are given
directly on this material (for example, on the application
of a certain formula). For real knowledge of the topic,
one must be able to solve problems, in the solution of
which it is necessary to use, along with the material of
the topic being studied, the knowledge obtained earlier
in the study of other topics within this or other subjects.

Consideration of phenomena and laws studied in various sections of educational subjects (regardless of the
time of study) at each stage must be complete and cannot
be incomplete due to the fact that students do not have
any preliminary data. It is also necessary to take into
account the consistency and connectivity of the acquired
knowledge so that knowledge does not turn into a set
of fragmentary information and definitions that require
banal memorization. Logical comprehension of the material, building connections of this material with the
available knowledge and the surrounding reality are the
main components of individual experience. Knowledge
appears in the form of concepts and relations between
them, as well as judgments and conclusions of the student
derived from them. It is such knowledge in the form of
skills and abilities that is best stored in the memory of the
student.It is necessary to adjust the programs for studying
disciplines for the timely use of knowledge in the study
of other disciplines.

It is also necessary to consider the issue of filling the
content of educational material on each topic in each
individual discipline. It should be borne in mind that
there are certain restrictions both in complexity and in
the volume of new material that a student can perceive in
the time allotted for this, in order to avoid overloads that
may adversely affect his health. Topics that are displayed
in the physical and information space around us due to
their vital relevance are mastered most effectively. Topics
that do not find reflection in the environment and are
not required to obtain the necessary life skills should be
derived from the compulsory school curriculum material
and given in the most compact form at the familiarization
level. Satisfaction of educational programs and the very
process of teaching these principles contributes to the
fact that various disciplines will form comprehensive
knowledge about the world around them.

When considering issues related to education, one
cannot ignore the aspect related to the individuality of
each student. Each student has abilities and predispositions for certain subjects. Taking into account these
factors ensures the maximum possible disclosure of the
creative potential of each person. The development of
these abilities and the preparation of schoolchildren for
the choice of professional activity in later life should be
served by additional and optional education. At the level
of such education, it is possible to implement a more personalized approach to each student, in which it becomes
possible to fully reveal the creative potential of each. At
the same time, it is necessary to give more extensive and
in-depth knowledge in the chosen disciplines. Also, the
issues of organizing secondary education must be closely
linked with the organization of education at subsequent
levels. The foundation of knowledge, their base formed
at school is the starting point from which the next stage
in the student’s life begins.

Unfortunately, it should be noted that in recent years
there has been a decline in the level of school education of graduates of secondary educational institutions.
The introduction of new curricula does not correct, and
sometimes exacerbates this situation. There are various
subjective and objective reasons leading to this. Among
the main ones are a sharp increase in the amount of
information in various sections, caused by the development of science and an increase in the availability of
information, the inertia and conservatism of school programs and educational technologies. One of the reasons
is that programs and textbooks for different courses are
compiled by different people who specialize in certain
areas of knowledge. These people do not see the general
picture of emerging knowledge, they try to fill their
subject with as deep new data and definitions as possible,
to increase the number of hours allotted for studying
the subject at school. Sometimes such an increase in
material leads to the fact that the school curriculum in
some sections of the subject practically does not differ
from the programs of higher educational institutions.
According to the authors of these programs, this should
lead to interest in their subject, ”improvement” of the
quality of knowledge on the subject. In fact, this only
leads to the fact that students have to remember a lot
more information (sometimes unrelated), which leads
259
to overload and confusion of students. No group of
specialists will be able to fully take into account and
calculate all the issues related to education, since these
problems are multidirectional and multilevel. The traditional educational system cannot provide graduates with
a timely and proper level of knowledge. To maintain a
high level of demand in the labor market, students must
rapidly update the necessary knowledge, the volume of
which doubles on average every year and a half, which
requires constant retraining. The problem can be solved
by creating intelligent computer education systems based
on very large knowledge bases. Such systems should
make it possible to correct mistakes made in approaches
to the development of knowledge, to take into account
changes in the requirements for graduates’ competencies
that appear with the development of scientific and technical knowledge and the material and technical base of
society more quickly and efficiently. Digital technologies
will create an education system that will be more efficient
and balanced.

What should an intelligent system be able to do
and what knowledge base should it possess? In the
intellectual system, a knowledge base should be formed,
covering all the knowledge that a graduate should have
at the end of school. The knowledge base should be
multi-level. Otherwise, there will be no deep convergence
between the preparation of students and the knowledge
that they should have at the end of school. This part
of the general knowledge base should cover all subjects of the school curriculum, but only to the extent
sufficient to understand and assimilate information on
each subject. Also, the knowledge base should take
into account the problem of the amount of information,
taking into account the complexity of the material that
can be mastered in a particular period of study. Based
on this knowledge, the intellectual system must form a
training program, distribute during which period of study
specific sections of various subjects are studied and at
what level knowledge is formed during this period. At
the same time, the intellectual system should take into
account what knowledge the student has when he starts
studying a new topic. By this time, the student should
know all the necessary prerequisites, have knowledge
in this and other subjects necessary to study the topic.
As a result, a balanced distribution of all studied material from various subjects over time should be formed.
Since the volume of the studied material, taking into
account the complexity and the training time, have a
finite value, the intellectual system must limit the amount
of materials given for training in each period of time,
and cut off materials that only increase the amount
of memorized data, but do not carry any additional
information necessary to understand and master the basic
laws and phenomena. As a result, a model of the subject
area should be created for each discipline, taking into
account interdisciplinary links. In this regard, it is very
important to provide technological means of ”transition”
of boundaries between educational materials of different
academic disciplines. The domain model plays a major
role, since it is used to solve the problems of structuring
and systematizing educational material, implementing
navigation and search algorithms for educational material
and implementing adaptive learning management, etc.
In ideal case, the student should be able to work with
educational material in solving a number of problems
not on the scale of a single academic discipline, but
on the scale of all disciplines related to the issue under
study. It is necessary to mention the logical organization
of educational material within the framework of the
specialty, which allows you to identify the connections
of academic disciplines, certain topics of these academic
disciplines, their constituent fragments (theorems, definitions of concepts, etc.) with other academic disciplines,
topics, fragments of educational material, subsequent
and previous. It becomes possible to determine a more
rational sequence for studying educational material. The
knowledge management system located in the knowledge
base should also ensure the collection and systematic
organization, analysis of data and knowledge from various sources, replenishment of the knowledge base from
within the system itself.

An intelligent subsystem with an open, selfsupplementing database should also be created, which
will grow over time as topics in various subjects are
covered. An educational approach based on the gradual
filling of the student with knowledge and skills through
their gradual presentation within a set of disciplines
remains relevant. At each stage of training, the student
should have access to the level of knowledge in the
database that corresponds to the material he has studied.
learning based on these assessments, i.e. an intellectual
system should itself consist of a number of subsystems
containing knowledge bases semantically correlated with
each other.

The individual activity of students is an important
component of education, which forms the skills of the
student, contributes to the most effective assimilation of
knowledge, determines his inclinations and contributes
to their development. Individual activity is present in
various ways of studying information: training in a
mandatory program, additional optional individual training. Additional types of education may use information
that is not included in the mandatory school curriculum.
Such broader information should be available to students,
but it should be in the general knowledge base and
used for additional, optional or independent education.
The development of this information should take place
outside the main school education, both in terms of time
and information and program components, and contribute
to the development of individual abilities of students.
\end{document}
