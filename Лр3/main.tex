\documentclass{article}
\usepackage{graphicx} % Required for inserting images
\usepackage{geometry}
\usepackage{multicol}
\usepackage{blindtext}
\usepackage[russian]{babel}
\graphicspath{{images/}}
 \geometry{
 a4paper,
 total={170mm,257mm},
 left=20mm,
 right=20mm,
 top=20mm,
 bottom=20mm,
 }
 \setcounter{page}{240}
\begin{document}
\begin{multicols*}{2}
of intelligent computer systems. Interoperability should
be understood as the interaction of intelligent computer
systems based on mutual understanding between these
systems themselves. Achieving the quality of interoperability allows the construction of a multi-agent network
of interoperable intelligent computer systems for medical
purposes.

The multi-agent architecture of ostis-systems involves
the development of communicative aspects of machine
”thinking”, which can be traced in multi-agent systems
[7]–[9].

OSTIS Technology makes it possible to effectively
solve the problems of integrating various medical systems into a single interoperable network. To do this,
it is necessary and sufficient to create an intelligent
user interface (IPI) for each individual medical system,
many of which form a single interoperable network. The
IPI network will combine (integrate) traditional medical
information systems (MIS), conventional diagnostic systems (DS), intelligent diagnostic systems (IDS), consulting systems (CS), telemedicine systems (TMS), external
medical knowledge bases (MBZ) and other computer
systems for medical purposes.

As part of the OSTIS development strategy, the concept of a personal intellectual assistant (secretary, referent) has been developed.

In line with this concept, the intelligent health FSDmonitoring system will act as a personal intellectual
consultant on individual health-improving and preventive
regimes of a healthy lifestyle, on the expediency of
additional diagnostic studies, on the desired timing of
treatment to a doctor of a particular specialty. The ostissystems development strategy includes the position of
developing a set of tools for individual comprehensive
permanent medical control and health monitoring within
the framework of a personal intelligent assistant.
\begin{center}
    V. CONCLUSION
\end{center}
Doctors have understood the need for regular monitoring of the health of an individual and the population
for a very long time. It is also well understood that
the possibilities of such control are limited by the set
of diagnostic technologies used, in which there are no
positions with a sufficiently high efficiency/cost ratio,
that is, with high efficiency and low cost. In relation to
the tasks of health control, effectiveness is determined,
first of all, from the point of view of the possibilities of
early diagnosis of diseases.

Today, the system of medical examination can be
conditionally attributed to the monitoring of an individual’s health, but the frequency of a year or two is too
high for effective monitoring. The cost of functioning
of the medical examination system is also high and,
unfortunately, the diagnostic effectiveness of the existing
medical examination system is low, which was shown in
the pilot project of using FSD diagnostics in medical
examination with subsequent verification of diagnoses
[10].
\begin{center}
   REFERENCES
\end{center}
 {\small [1] The medical spectral-dynamic complex. Available at: http://http:
//www.kmsd.su (accessed 2009, Sep)}\\
{\small [2] V. V. Golenkov, N. A. Gulyakina Printsipy postroeniya massovoi
semanticheskoi tekhnologii komponentnogo proektirovaniya intellektual’nykh sistem [Principles of construction of mass semantic technology of component design of intelligent systems]. {\emph {Open
semantic technologies of design of intelligent systems}} (OSTIS-2011): materials of the International Scientific and Technical
Conference. Minsk, February 10-12, 2011). Minsk: BSUIR, 2011,
pp. 21–58. }\\
{\footnotesize [3] J. Jung, I. Choi, M. Song An integration architecture for knowledge management systems and business process management
systems. \emph{Computers in Industry}, 2007, Vol. 58, pp. 21–34.}\\
 {\footnotesize [4] V. V. Golenkov, N. A. Gulyakina Semanticheskaya tekhnologiya
komponentnogo proektirovaniya sistem, upravlyaemykh
znaniyami [Semantic technology of component design of
knowledge-driven systems]. \emph{Open semantic technologies of
intelligent systems design}, 2015, No. 5, pp. 57–78. }\\
{\footnotesize [5] B. A. Kobrinskii, A. E. Yankovskaya Convergence of Applied
Intelligent Systems with Cognitive Component.\emph{ Open Semantic
Technologies for Intelligent System}. Cham, Communications in
Computer and Information Science, 2020, vol. 1282, pp. 34–47.}\\
{\footnotesize [6] B. A. Kobrinsky Argumentatsiya i intuitsiya v estestvennom i
iskusstvennom intellekte [Argumentation and intuition in natural
and artificial intelligence]. VI National Conference on Artificial
Intelligence with international participation KII’98: Proceedings
of the conference, Pushchino, 1998, Vol. 1, pp. 7–14. }\\
 {\footnotesize [7] V. B. Tarasov Ot mnogoagentnykh sistem k intellektual’nym organizatsiyam: filosofiya, psikhologiya, informatika [From multiagent systems to intellectual organizations: philosophy, psychology, informatics]. M.,Editorial URSS, 2002, P. 352.
 }\\
 {\footnotesize [8] V. I. Gorodetsky Samoorganizatsiya i mnogoagentnye sistemy. I.
Modeli mnogoagentnoi samoorganizatsii [Self-organization and
multi-agent systems. I. Models of multi-agent self-organization].
\emph{News of the Russian Academy of Sciences. Theory and control
systems}, 2012, No. 2, pp. 92–120. }\\
{\footnotesize [9] V. I. Gorodetsky Samoorganizatsiya i mnogoagentnye sistemy.
II. Prilozheniya i tekhnologiya razrabotki [Self-organization and
multi-agent systems. II. Applications and development technology]. \emph{News of the Russian Academy of Sciences. Theory and
control systems}, 2012, No. 3, pp. 102–123.}\\
 {\footnotesize [10] V. N. Rostovtsev, T. I. Terekhovich, A. N. Leaders, I. B.
Marchenkova Diagnosticheskii skrining v sisteme dispanserizatsii
[Leaders Diagnostic screening in the system of medical examnation].\emph{ Issues of organization and informatization of healthcare},
2018, No. 2 (95), pp. 39–46.}
\begin{center}
   \textbf{Интеллектуальные системы мониторинга здоровья} % ошибка
\end{center}
\begin{center}
 Ростовцев В. Н.
\end{center}
\emph{В работе рассмотрены проблемы современного состояния
комплексного мониторинга здоровья человека, а также соответствующих диагностических технологий. Предложен подход к интеллектуализации процесса регулярного контроля
здоровья на основе систем ФСД-диагностики}
\begin{right}
  Received 13.03.2023
\end{right}
\end{multicols*}
\newpage
\begin{center}
\huge{\textbf{Technology of Neurological Disease Recognition Using Gated Recurrent Unit Neural Network and Internet of Things}}
\vspace{4mm} %5mm vertical space
\end{center}
 \begin{center}
\large{Uladzimir Vishniakou and Yiwei Xia and Chuyue Yu

\textit{Belarusian State University of Informatics and Radioelectronics}

Minsk, Belarus

Email: vish@bsuir.by, minskxiayiwei@qq.com, ycy18779415340@gmail.com}
\end{center}
\begin{multicols*}{2}
\textbf{\textit{Abstract}—In this paper, authors proposed a neurological
disease recognition technique using gated recurrent unit
neural network and supporting Internet of Things (IoT),
which was checked by taking Alzheimer’s disease (AD)
and Parkinson’s disease (PD) as examples. In this method
first pre-emphasized and denoised the voice data, then segmented the voice signals with a sliding fixed window using
the Hamming window function. Then we were extracted the
eGeMAPSv02 voice features from the window signal, fed
the features into the gated recurrent unit neural network
model for its training, testing and achieve the disease
diagnosis. The results of the study showed that despite
the limited generalization ability of the gated recurrent
unit model, it can still efficiently achieve voice recognition
detection of a portion of neurological diseases. The model is
implemented on the basis of the IoT platform for building
a subsystem of IT diagnostics of patients as part of the
smart city project. The code is stored in https://github.com/
HkThinker/Technology-of-neural-disease-recognition. %нужен  абзац
\textit{Keywords}—gated recurrent unite neural network, Internet of things, voice recognition, neurological disease}
\begin{center}
I. INTRODUCTION
\end{center}
Neurological disease usually result in structural or
functional changes in the nervous system, causing patients to suffer from perception, thinking, emotion and
behavior, and present a significant challenge to the global
healthcare system. They are a group of diseases that
affect the nervous system and include a variety of disorders such as neurodegenerative diseases, autoimmune
diseases, cerebrovascular diseases, and brain injuries. For
example, PD is a neurodegenerative disease that affects
motion management and is characterized by symptoms
such as hand tremors, limb stiffness, slow movements,
and postural instability. AD is similar and results in
memory loss, cognitive decline, and abnormal language
and behavior. They tend to occur in older age groups,
currently have no complete medical cure, but early diagnosis and prompt treatment can alleviate symptoms and
slow progression. Traditional diagnosis of neurological
diseases is usually based on doctors’ clinical experience,
medical history, physical examination and specific tests,
which has limitations and requires a lot of labor and
resources. In recent years, with the rapid development of
artificial intelligence and IoT technologies, neurological
disease identification technologies using neural networks
and supporting IoT are expected to become a new
breakthrough point.
The main purpose of this paper is to investigate the
Gated Recurrent Unit (GRU) neural networks and IoT
technologies to recognition for neurological diseases. To
be specific, our research aims to achieve the following
objectives:
\begin{enumerate}
\item To develop a GRU neural network model, which
was trained through a publicly available database
to implement the diagnosis and prediction of PD
and AD.
\item By using IoT technology, we collected patients’
voice data and combined these data with the GRU
neural network model to improve the precision and
accuracy of diagnosis and prediction of neurological diseases.
\item To deploy the GRU neural network model to the
Thingspeak IoT platform.
\end{enumerate}
\begin{center}
II. RELATED WORK
\end{center}
\textit{A. Application of IoT in Neurological Disease Diagnosis}

Neurological disease diagnosis systems that are based
on neural network technology and IoT technology have
been widely used.

B. Lu [1] built a practical brain MRI-based AD diagnostic classifier using deep learning/transfer learning on
datasets of unprecedented size and diversity. The purpose
of Mukherji [2] was to identify non-invasive, inexpensive
markers and develop neural network models that learn
the relationship between those markers and the future
cognitive state. David Payares-Garcia [3] proposed a
classification technique that incorporates uncertainty and
spatial information for distinguishing between healthy
subjects and patients from four distinct neurodegenerative diseases: AD, mild cognitive impairment, PD,
and Multiple Sclerosis. Abbas Sheikhtaheri [4] aimed
to identify and classify the IoT technologies used for
AD dementia as well as the healthcare aspects addressed
by these technologies and the outcomes of the IoT
interventions.
\newpage
Researchers had identified the feasibility of integrating deep learning, cloud, and IoT, Syed Saba Raoof
[5] explained a summary of various techniques utilized
in smart healthcare, i.e., deep learning, cloud-basedIoT applications in smart healthcare, fog computing in
smart healthcare, and challenges and issues faced by
smart healthcare and it presents a wider scope as it is
not intended for a particular application such aspatient
monitoring, disease detection, and diagnosing and the
technologies used for developing this smarta systems are
outlined. Reyazur Rashid Irshad [6] proposed a novel
healthcare monitoring system that tracks disease processes and forecasts diseases based on the available data
obtained from patients in distant communities. Rafael
A Bernardes [7] presented a perspective on integrating
wearable technology and IoT to support telemonitoring
and self-management of people living with PD in their
daily living environment.\\
B.\textit{ Classification of Voice Features}

Since more than 90 \% of PD patients have varying
degrees of dysphonia in the early stages of the disease,
the diagnosis of PD based on voice features has the
merits of being non-invasive and convenient. Darley [8]
first used voice to diagnose aphasia in 1969. Saker et al.
[9] preprocessed the voice data and extracted features,
then applied SVM and KNN classification algorithms to
the feature matrix for classification, eventually obtaining
an average accuracy and a maximum accuracy of 55 \%
and 85 \%, respectively, which initially confirmed the
feasibility of voice features to classify PD. To further
improve the accuracy of model prediction and simplify
the algorithmic model, scholars have applied different
feature selection algorithms.
\begin{center}
III. METHODOLOGY AND DATASETS
\end{center}
\textit{A. Pre-emphasis and Denoising of Voice Signals}

It is difficult to obtain the high-frequency part of
the unprocessed voice signal because the power of the
voice signal will be significantly attenuated after the
sound gate excitation as well as the influence of mouth
and nose radiation, combined with the smaller energy
corresponding to the high frequency while the larger
energy corresponding to the low frequency in the spectrogram of the voice signal. In order to facilitate the
spectrum analysis, this paper adopted a first-order FIR
high-pass digital filter for the pre-emphasis processing
of the voice signal. The purpose of pre-emphasis is to
improve the high-frequency part, so that the spectrum of
the voice signal becomes flat, thus the spectrum can be
obtained with the same signal-to-noise ratio in the whole
frequency band.

Voice denoising is an effective part of signal preprocessing, mainly to improve the quality of voice and
obtain more pure voice signals. The Fig. 1 shows the
process of voice signal denoising. 

\includegraphics[width=0.5\textwidth]{фото лр3.png} 
\begin{center}
Figure 1. Flow chart of voice denoising.
\end{center}
%подпись?
According to the different parts of the noise introduction, voice noise can be divided into background noise
and transmission noise. In this paper, spectral subtraction
algorithm was used to denoise the voice. The spectral
subtraction algorithm is designed based on the principle
that pure voice is statistically independent of the noise
signals.

\vspace{3mm}
\hspace{-5mm}\textit{B. Framing and Windowing of Voice Signals}
\vspace{2mm}

The Fourier transform commonly used in voice signal
processing calls for a smooth signal, but the main feature
of the voice signal is the short-time smoothness, i.e.,
the stability of the voice signal in 10–30 ms period.
Therefore, if we want to characterize the voice signal,
it is necessary to analyze the short-time characteristics
of the voice signal, the original signal is framed, and
the frame frequency signal with short-time smoothness
is derived. In the process of frame splitting, the signal
tends to produce spectral deficiencies, so a windowing
process must be performed between frames to keep the
signal at the truncation without distortion. The windowing function used in this paper is the Hamming window
function, with window size of 1024, frequency of voice
signal is 44.1khz, and the overlap rate of window is 50
\%, hence the voice time of one window is about 23 ms.

\vspace{3mm}
\hspace{-5mm}\textit{C.Feature Extraction of Voice Signals}
\vspace{2mm}

We used an extended version of GeMAPS (Basic
Affective Parameter Set), eGeMAPSv02 [10], a speech
feature set. It uses acoustic features and spectral-based
features to describe the speech signal, with a total of
88 features. It contains 25 low-level descriptor features,
namely pitch, jitter, gating frequency, gating bandwidth,
gloss, loudness, harmonic-to-noise ratio (HNR), Alpha
ratio, Hammarberg index, spectral slope 0–500 Hz, spectral slope 500–1500 Hz, 3 gating relative energies, 3 relative energies, 3 harmonic differences, 4 Mel–Frequency
Cepstral Coefficients, 1 spectral flux. 53 other parameters
are derived from these basic parameters.

\vspace{3mm}
\hspace{-5mm}\textit{D. 6–layer Gated Recurrent Unit Model}
\vspace{2mm}

In the paper, a multi-layer GRU model is constructed
for voice data recognition. two mechanisms, an update
gate and a reset gate, are included in the GRU module.
The internal equation of a single GRU model is :
\end{multicols*}
\end{document}
