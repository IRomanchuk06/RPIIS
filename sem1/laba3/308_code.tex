\documentclass{article}
\usepackage[utf8]{inputenc}
\usepackage[left=17mm, top=17mm, right=17mm, bottom=25mm, nohead]{geometry}
\usepackage{enumitem}
\usepackage{multicol}
\usepackage{ulem}
\usepackage{mathtools}
\setlist[itemize]{itemsep=0pt, parsep=0pt, partopsep=0pt, topsep=1.4pt}
\setcounter{page}{308}
\date{}
\begin{document}
\begin{figure}
    \centering
    \includegraphics{image/Снимок экрана 2023-11-08 124534.png}
    \caption{OSTIS Ecosystem Architecture}
    \label{fig:enter-label}
\end{figure}
\begin{multicols}{2}
\begin{description}[leftmargin=!, labelwidth=1cm, itemsep=-1.5mm]

\begin{description}[leftmargin=!, labelwidth=1.2cm, itemsep=-1.5mm]
   \item \textbf{[}violation of the generality of concepts and
in the generality of basic knowledge\textbf{]}
\end{description}
 \vspace{-0.2cm} \item[$\supset$] \textit{threat. destruction of knowledge base semantics
(semantic viruses)}

   \vspace{-0.2cm}  $\Rightarrow$  \hspace{0.5cm} \textit{explanation}:* 
   \par
   \begin{description}[leftmargin=!, labelwidth=0.9cm, itemsep=-1.5mm]
  \item \vspace{-0.3cm}  \textbf{[}substitution or removal of nodes and links
between them in the knowledge base\textbf{]}
\end{description}


   \vspace{-0.2cm} \item[$\supset$] \textit {threat. excessive amount of incoming information} \\
    \vspace{-0.4cm}\item[$\supset$] \textit {threat. breach of non-repudiation} \\
   \vspace{0.1cm} \hspace{-0.23cm}  $\Rightarrow$  \hspace{0.5cm} \textit{explanation}:* 
  \begin{description}[leftmargin=!, labelwidth=0.9cm, itemsep=-1.5mm]
   \item \vspace{-0.4cm}\textbf{[}issuance of unauthorized actions as legal,
as well as concealment or substitution of
information about the actions of subjects\textbf{]}
\end{description}
    
\item[$\supset$] \vspace{-0.2cm} \textit {threat. breach of accountability} \\
\vspace{0.1cm} \hspace{-0.23cm}  $\Rightarrow$  \hspace{0.5cm} \textit{explanation}:* 
 \begin{description}[leftmargin=!, labelwidth=0.9cm, itemsep=-1.5mm]
   \item \vspace{-0.4cm}\textbf{[}unauthorized or erroneous change, distortion or destruction of information about
the performance of actions by the subject\textbf{]}
\end{description}

\item[$\supset$] \vspace{-0.1cm} \textit {threat. violation of authenticity (authenticity)} \\
\vspace{0.1cm} \hspace{-0.23cm}  $\Rightarrow$  \hspace{0.5cm} \textit{explanation}:* 
\begin{description}[leftmargin=!, labelwidth=0.9cm, itemsep=-1.5mm]
   \item \vspace{-0.4cm}\textbf{[}performing actions in the system on behalf
of another person or issuing unreliable
resources (including data) as genuine\textbf{]}
\end{description}

\item[$\supset$] \vspace{-0.1cm} \textit {threat. breach of credibility
} \\
\vspace{0.1cm} \hspace{-0.23cm}  $\Rightarrow$  \hspace{0.5cm} \textit{explanation}:*
\begin{description}[leftmargin=!, labelwidth=0.9cm, itemsep=-1.5mm]
   \item \vspace{-0.4cm}\textbf{[}intentional or unintentional provision and
use of erroneous (incorrect) or irrelevant
(at a specific point in time) information, as
well as the implementation of procedures
in violation of the regulations (protocol)\textbf{]}
\end{description}
\end{description}

\par 
\vspace{-0.1cm} Let’s present the main directions of ensuring the
information security of ostis-systems to prevent emerging
threats:

\begin{itemize}
   \vspace{0.18cm} \item limitation of information traffic analyzed by the 
intelligent system;
    \item policy of differentiation of access to the knowledge
base;
    \item connectivity;
    \item introduction of semantic metrics;
    \item  semantic compatibility;
    \item activity.
\end{itemize}
\par
It should be noted that at the design stage of the
OSTIS technology itself, the basic principles of ensuring
information security were already laid down as part of
the design of individual components of the system. So
already initially, support for semantic compatibility and
cohesion is provided in ostis systems due to the system’s
ability to detect malicious processes in the knowledge
base
\par
\textbf{Restriction of information traffic analyzed by the
intelligent system
}
\par
The exponential growth of the volume of information
circulating in information flows and resources under the
conditions of well-defined quantitative restrictions on
the capabilities of the means of its perception, storage,
transmission and transformation forms a new class of information security threats characterized by the redundancy
of the total incoming information traffic of intelligent
systems.
\par
As a result, the overflow of information resources of
an intelligent system with redundant information can
provoke the spread of distorted (destructive semantic)
information. The general methodology for protecting
intelligent systems from excessive information traffic is
carried out through the use of axiological filters that
implement the functions of numerical assessment of the
value of incoming information, selection of the most
valuable and screening (filtering) of less valuable (useless
or harmful) using well-defined criteria.

\end{multicols}
\end{document}
