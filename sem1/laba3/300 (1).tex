\documentclass{article}
\usepackage[utf8]{inputenc}
\usepackage[left=17mm, top=17mm, right=17mm, bottom=9mm, nohead, nofoot]{geometry}
\usepackage{enumitem}
\usepackage{multicol}
\usepackage{ulem}
\setlist[itemize]{itemsep=0pt, parsep=0pt, partopsep=0pt, topsep=1.4pt}
\setcounter{page}{300}
\date{}
\begin{document}
\begin{multicols}{2}

\begin{itemize}
    \item analysis problem and services provided by numerous sources of \textit{geoinformation};
    \item implementation of knowledge output using spatial and thematic information as components of \textit{knowledge} about \textit{terrain objects} using the Question Language;
    \item implementat ion of a \textit{cartographic interface} in \textit{intelligent ostis-systems} as a natural way for a human to represent information about \textit{terrain objects}.
\end{itemize}
\par
The constant evolution of models and means of ontological description of subject domains, using spatial and temporal components, their heterogeneity, and ambiguity, poses new challenges in terms of interaction, integration, and compatibility of various applied systems due to:
\begin{itemize}
    \item the integration of \textit{subject domains} and their corresponding ontologies (vertical level);
    \item expanding the functionality of the systems using
reusable components of these systems (horizontal
level), in particular, designing components for new
territories and/or in a new time interval.
\end{itemize}
\par
In order to implement the requirements represented,
it is proposed to consider the map as an \textit{information construction}, the elements of which are \textit{terrain objects}, and to offer:
\begin{itemize}
    \item \textit{the Subject domain and ontology of terrain objects};
    \item \textit{Map Language Syntax};
    \item \textit{Denotational semantics of the Map Language.}
\end{itemize}
\par
The transition from maps to their \textit{meaning} is based on:
\begin{itemize}
    \item the formal description of the Map Language Syntax;
    \item the formal description of the Denotational semantics
of the Map Language.
\end{itemize}
\par
At the same time, \textit{semantic compatibility of geoinformation systems} and their components are provided due to the common ontology of \textit{terrain objects}, which is necessary for the interoperability of \textit{geoinformation systems} for various purposes and their components.

Thus, \uline{structural and semantic interoperability of geoinformation systems is ensured due to the transition from the map to the semantic description of map elements, that is, terrain objects and connections (spatial relations) between them.}

The presence of these circumstances determines the existence of a scientific and technical problem of intel-lectualization of geoinformation systems and the creation of the Technology for intelligent geoinformation systems design, which are based on the principles of designing ostis-systems.
\begin{center}
    III. SYSTEMATIZATION OF PROBLEMS SOLVED BY
INTELLIGENT GEOINFORMATION SYSTEMS
\end{center}
One of the ways to increase the efficiency of using
information and computing tools is \textit{intellectualization of geoinformation systems.}

\textit{Intellectualization of geoinformation systems} implies:
\begin{itemize}
    \item the possibility of end-user communication with the
system on the \textit{Question Language};
    \item the use of various interoperable problem solvers with
    \item the use of \textit{cartographic interface} to visualize the
source data and results.
the possibility of explaining the solutions obtained.
\end{itemize}
\par
The implementation of the capabilities of \textit{intelligent geoinformation systems} can be carried out using:
\begin{itemize}
    \item \textit{knowledge base} management systems;
    \item multimedia \textit{knowledge and databases} by application areas;
    \item interoperable \textit{problem solvers};
    \item an intelligent \textit{cartographic interface};
    \item \textit{expert systems} in various fields of human activities;
    \item \textit{decision support systems};
    \item \textit{intelligent assistance systems}.
\end{itemize}
\par
\textit{Intellectualization of geoinformation systems} involves solving the following problems:
\begin{itemize}
    \item the use of digital cartographic material and data from \textit{remote sensing of the Earth} in problem-oriented areas [2];
    \item planning actions in a dynamically changing situation in conditions of incomplete or fuzzy data using expert knowledge [3];
    \item analysis of emergency situations and preparation of materials for decision-making on prevention or elimination of their consequences;
    \item creation of decision support systems for applied
\textit{geoinformation systems} of territorial planning and
management [4];
    \item development of diagnostic expert systems for geological exploration activities with remote access to them;
    \item logistics planning, creation of expert systems and
enterprise management software;
    \item creation of control and navigation systems;
    \item creation of \textit{expert systems} for forecasting the occurrence and development of technogenic and natural situations: floods, earthquakes, extreme weather conditions (precipitation, temperature), epidemics, spread of radionuclides, chemical emissions, meteorological forecast, etc.;
    \item creation of \textit{expert systems} for the selection of terrain compartments for the construction of various objects;
    \item creation of \textit{expert systems} for planning the efficient use of agricultural land;
    \item creation of \textit{expert systems} and software tools for geodata analysis;
    \item creation of image and picture recognition systems based on data from \textit{remote sensing of the Earth};
    \item creation of banks of digital cartographic information with means of remote access to them;
    \item image processing;
    \item retrospective analysis of events (see [5], [6];   
\end{itemize}
\end{multicols}
\end{document}
