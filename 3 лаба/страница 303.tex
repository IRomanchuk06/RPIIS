\documentclass{article}
\usepackage[utf8]{inputenc}
\usepackage[left=17mm, top=17mm, right=17mm, bottom=5mm, nohead, nofoot]{geometry}
\usepackage{enumitem}
\usepackage{multicol}
\usepackage{setspace}
\usepackage{ragged2e}
\usepackage{fancyhdr}
\usepackage{titlesec}
\usepackage{changepage}
\setlist[itemize]{itemsep=0pt, parsep=0pt, partopsep=0pt, topsep=0pt}
\setcounter{page}{303}
\begin{document}
\begin{multicols}{2}

\begin{tabbing}
	\hspace{-0.17cm} localization type into:\textit{ area objects\^\,\ linear (multilinear)}\\ \ \textit{objects\^\ }, and \textit{point objects\^\ } .
\end{tabbing}
          At the next stage of developing the ontology of
\textit{terrain objects}, we will set the subdivision of \textit{terrain
objects} on orthogonal bases, which corresponds to the
placement of objects in accordance with thematic layers
in \textit{geoinformation systems}.

      For each \textit{terrain object}, the main semantic characteristics inherent only 
to it are highlighted. It should be
particularly noted that metric characteristics do not have
such a property. According to this classifier, each class
of \textit{terrain objects} has a unique unambiguous designation.
The classifier hierarchy has eight classification stages and
consists \textit{of the class code, subclass code, group code,
subgroup code, order code, suborder code, species code,
subspecies code}. Thus, thanks to the coding method,
generic relations have already been defined, reflecting
the correlation of various \textit{terrain object classes}, and the
characteristics of a specific \textit{terrain object class} have also
been established. Due to the fact that the basic properties
and relations are set not of specific \textit{physical objects} but of
their classes, then such information is meta-information in
relation to specific \textit{terrain objects}, and the totality of this
meta-information is an ontology of \textit{terrain objects}, which
in turn is part of the \textit{knowledge base} of the \textit{intelligent
geoinformation system}.
\begin{tabbing}
\hspace{0cm}\textbf{terrain object}\\
 \(\Rightarrow\) \ \ \ \ \ \   \textit{subdividing*:}\\
 \ \ \ \ \ \ \ \ \ \ \ \textbf{Typology of terrain objects by localization\^\\ \ \ \ \ \ \ \ \ \ \ \  =  \{ }
\end{tabbing}
\begin{adjustwidth}{15mm}{}
 \begin{itemize}  
 •  \ \ \ \     point terrain object
    \begin{itemize}
        \ \(\Rightarrow\) inclusion*:\\
        \begin{itemize}
     \ \ \ \ • well\\
       \ \ \ \      • light post
        \end{itemize}
    \end{itemize}
    • \ \ \ \ linear terrain object
    \begin{itemize}
         \(\Rightarrow\)inclusion*:\\
        \begin{itemize}
            \ \ \ \ • bridge
        \end{itemize}
    \end{itemize}
    • \ \ \ \ multilinear terrain object
    \begin{itemize}
         \(\Rightarrow\)inclusion*:\\
        \begin{itemize}
        \ \ \ \  • river\\
        \ \ \ \ • road
        \end{itemize}
    \end{itemize}
    • \ \ \ \ area terrain object
    \begin{itemize}
         \(\Rightarrow\)inclusion*:\\
        \begin{itemize}
           \ \ \ \  • lake\\
          \ \ \ \  • administrative area
        \end{itemize}
    \end{itemize}
\end{itemize}
\ \ \ \ \ \}
\end{adjustwidth}
\begin{center}
 VII. SPECIFICATION OF THE MAP LANGUAGE
 \end{center}
\ \ \ The \textit{Map Language} belongs to the family of semantic
compatible languages – \textit{sc-languages} – and is intended for
the formal description of \textit{terrain objects} and the relations
between them in \textit{geoinformation systems}. Therefore, the
\textbf{Map Language Syntax}, like \textit{syntax} of any other \textit{sclanguage}, is the \textit{Syntax} of the \textit{SC-code}. This approach
allows:
	\begin{itemize}
		\item using a minimum of means to interpret the specified
\textit{terrain objects} on the map;

		\item using the \textit{Question Language for ostis-systems};
	
		\item  reducing the search to most of the given \textit{questions}
to searching for information in the current state of
the \textit{ostis-system knowledge base}
	\end{itemize}
\ \ \  \textbf{Denotational semantics of the Map Language} includes
the \textit{Subject domain and the ontology of terrain objects}
and \textit{their geosemantic elements}.
\begin{center}
VIII. AUTOMATION TOOLS FOR THE INTELLIGENT
GEOINFORMATION SYSTEMS DESIGN
\end{center}
The design of intelligent geoinformation systems is
carried out in stages. At the first stage, the knowledge
base of the subject domain is formed and for this purpose
an electronic map (voluntary cartographic information)
is analyzed and translated into the knowledge base of
terrain objects with the establishment of geosemantic
elements for the corresponding territory. At this stage, it
is determined, firstly, to which class the terrain object
under study belongs and, further, depending on the type
of object, the concept of a knowledge base corresponding
to a specific physical terrain object is formed. Thus, many
concepts are created that describe specific terrain objects
for each class of terrain objects. It should be noted that
it is at this stage of the formation of the knowledge base
that semantic elements are established. At the second
stage of designing an intelligent geoinformation system,
the knowledge base obtained at the first stage is integrated
with external knowledge bases. At this stage, in addition
to geographical knowledge, knowledge of related subject
domains is added, thereby it becomes possible to establish
interdisciplinary connections. An illustrative example is
integration with biological classifiers, which in implementation represent an ontology of flora and fauna objects.
Such integration expands the functional and intelligent
capabilities of the applied intelligent geoinformation
system. Note that at this stage, homonymy is removed in
the names of geographical objects belonging to the classes
of settlements. For settlements of the Republic of Belarus,
this is achieved by using the \textit{system of designations of
administrative-territorial division objects and settlements}
and semantic comparison of geographical terrain objects
is carried out according to the following principle:
\begin{itemize}
  \item the terrain object class is determined;
  \item the terrain object subclass, species, subspecies, etc.
is determined in accordance with the classifier of
terrain objects, i.e. types of terrain objects in the
ontology;
  \item the attributes and characteristics that are inherent in
this terrain object class are determined;
  \item the values of the characteristics for this object class
are determined;
  \item the homonymy of identification is eliminated;
 \end{itemize}	
\end{multicols}
\end{document}
