\documentclass[10pt, a4paper]{proc}
\usepackage[utf8]{inputenc}
\usepackage{enumitem}
\usepackage{fancyhdr}
\usepackage{titlesec}
\usepackage[left=2.5cm,right=2.5cm,
    top=2.5cm,bottom=2cm]{geometry}
\cfoot{\vskip -1.5cm \thepage}
\pagestyle{fancy}
\linespread{0.84}
\setcounter{page}{305}
\titleformat{\section}{\large\centering\sc}{\thesection. }{0cm}{}[]
\title{
 \begin{spacing}
  \textbf{\LARGE{Ensuring Information Security of the OSTIS
Ecosystem}}
 \end{spacing}
}

\author{
 Valery Chertkov\\\textit{Euphrosyne Polotskaya State}\\\textit{University of Polotsk}\\Polotsk, Belarus\\v.chertkov@psu.by\\
 \and
 Vladimir Zakharau\\\textit{Belarusian State University of}\\\textit{Informatics and Radioelectronics
}\\Minsk, Belarus\\zakharau@bsuir.by
}
\begin{document}
 \maketitle
 \textbf{
  \textit{Abstract}—The development of artificial intelligence systems, associated with the transition to working with
knowledge bases instead of data, requires the formation of
new approaches to ensuring information security systems.
The article is devoted to the review of approaches and
principles of ensuring security in intelligent systems of
the new generation. The current state of methods and
means of ensuring information security in intelligent systems
is considered and the main goals and directions for
the development of information security ostis-systems are
formed. The information security methods presented in the
article are extremely important when designing the ostissystems security system and analyzing their security level.
}\\

\textbf{ \textit{Keywords}—information security, new generation intelligent system, Information security threats
}
\begin{center}
 I. INTRODUCTION
 \end{center}\\

 A wide variety of information security models, the
growing amount of data that needs to be analyzed to
detect attacks on information systems, the variability of
attack methods and the dynamic change in protected
information systems, the need for a rapid response to
attacks, the fuzziness of the criteria for detecting attacks
and the choice of methods and means of responding to
them, the lack of highly qualified security specialists
entails the need to use artificial intelligence methods to
solve security problems.
\begin{center}
 II. THE SPECIFICS OF ENSURING INFORMATION
SECURITY OF INTELLIGENT SYSTEMS OF A NEW
GENERATION
 \end{center}
 
 Information security of intelligent systems should be
considered from two points of view:
\begin{itemize}[noitemsep]
    \item application of artificial intelligence in information
security;
    \item organization of information security in intelligent
systems.
\end{itemize}

\textbf{The use of artificial intelligence in information
security}

Artificial intelligence is actively used to monitor and
analyze security vulnerabilities in information transmission networks [1]. The artificial intelligence system allows
machines to perform tasks more efficiently, such as:
\begin{itemize}[noitemsep]
    \item visual perception, speech recognition, decision making and translation from one language to another;
    \item invasion detection - artificial intelligence can detect
network attacks, malware infections and other cyber
threats;
systems.
     \item cyber analytics - artificial intelligence is also used
to analyze big data in order to identify patterns and
anomalies in the organization’s cyber security system
in order to detect not only known, but also unknown
threats;
    \item secure software development - artificial intelligence
can help create more secure software by providing
real-time feedback to developers.

\end{itemize}
Artificial intelligence is used not only for protection,
but also for attack, for example, to emulate acoustic,
video and other images in order to deceive authentication
mechanisms and further impersonation, deceive checking
a person or robot capcha, etc.

Currently, it is possible to define the following classes
of systems in which artificial intelligence is used [2]:
\begin{itemize}[noitemsep]
    \item UEBA (User and Entity Behavior Analytics) —
a system for analyzing the behavior of subjects
(users, programs, agents, etc.) in order to detect nonstandard behavior and use them to detect potential
threats using threat templates (patterns);

    \item IP (Threat Intelligence Platform) — platforms for
early detection of threats based on the collection
and analysis of information from indicators of
compromise and response to them. The use of
machine learning methods increases the efficiency
of detecting unknown threats at an early stage;

    \item EDR (Endpoint Detection and Response) — attack
detection systems for rapid response at the end
points of a computer network. Can detect malware,
automatically classify threats and respond to them
independently;
    \item SIEM (Security Information and EventManagement)
— systems for collecting and analyzing information
about security events from network devices and
applications in real time and alerts;
    \item NDR (Network Detection and Response) — sys305
tems for detecting attacks at the network level
and promptly responding to them. AI uses the
accumulated statistics and knowledge base about
threats;
   \item  SOAR (Security Orchestration and Automated Response) —systems that allow you 

\end{itemize}
 \end{document}
