\documentclass{article}
\usepackage[utf8]{inputenc}
\usepackage[russian]{babel}
\usepackage[left=17mm, top=17mm, right=17mm, bottom=5mm, nohead, nofoot]{geometry}
\usepackage{enumitem}
\usepackage{multicol}
\setlist[itemize]{itemsep=0pt, parsep=0pt, partopsep=0pt, topsep=0pt}
\setlength{\parindent}{0,5cm}
\setcounter{page}{304}
\begin{document}
\begin{multicols}{2}
\begin{itemize}
  \item appropriate connections are established between the
map object, the concept in the knowledge base with
the established geosemantic elements;
  \item spatial relations are established between terrain
objects assigned to certain classes.
 \end{itemize}\\
 
\begin{center}
 CONCLUSION
 \end{center}\\
 
\ \ \ \ Let us list the main provisions of this article:
\begin{itemize}
  \item  the development of geoinformation systems consists
in their intellectualization, thereby expanding the
range of applied problems using knowledge about
terrain objects;
  \item it is proposed to consider the map as an \textit{information
construction}, the elements of which are \textit{terrain
objects}, thereby ensuring the structural and semantic
interoperability of geoinformation systems due to the
transition from the map to the semantic description of
map elements, that is, terrain objects and connections
(spatial relations) between them;
  \item  ensuring semantic interoperability is achieved
through the development of ontologies of subject
domains, and the establishment of  \textit{geosemantic
elements} allows setting spatial characteristics of
terrain objects;
  \item availability of a particular  \textit{Technology for intelligent
geoinformation systems design} provides the process
of designing intelligent geoinformation systems built
on the principles of ostis-systems.
 \end{itemize}
\begin{center}
 ACKNOWLEDGMENT
 \end{center}
\ \ \ \  \ The author would like to thank the research group of
the Department of Intelligent Information Technologies
of the Belarusian State University of Informatics and
Radioelectronics for its help in the work and valuable
comments.
\begin{center}
 REFERENCES
 \end{center}
\begin{itemize}
 \footnotesize{
  \renewcommand{\labelitemi}{[1]}
		\item A. Kryuchkov, S. Samodumkin, M. Stepanova, and N. Gulyakina,
 \textit{Intellektual’nye tekhnologii v geoinformacionnyh sistemah [Intelligent technologies in geoinformation systems]}. BSUIR, 2006, p.
202 (In Russ.).
\renewcommand{\labelitemi}{[2]}
		\item S. Ablameyko, G. Aparin, and A. Kryuchkov,  \textit{Geograficheskie
informacionnye sistemy. Sozdanie cifrovyh kart [Geographical
information systems. Creating digital maps]}. Institute of Technical
Cybernetics of the National Academy of Sciences of Belarus, 2000,
p. 464 (In Russ.).
\renewcommand{\labelitemi}{[3]}	
\item  Ya. Ivakin, “Metody intellektualizacii promyshlennyh
geoinformacionnyh sistem na osnove ontologij [Methods
of intellectualization of industrial geoinformation systems based
on ontologies] ,” Doct. diss.: 05.13.06, Saint-Petersburg, 2009,
(In Russ.).
\renewcommand{\labelitemi}{[4]}
\item M. Belyakova,  \textit{Intellektual’nye geoinformacionnye sistemy dlya
upravleniya infrastrukturoj transportnyh kompleksov [Intelligent
geoinformation systems for infrastructure management of transport
complexes]}. Taganrog : Southern Federal University Press, 2016,
p. 190 (In Russ.).
\renewcommand{\labelitemi}{[5]}
\item A. Gubarevich, O. Morosin, and D. Lande, “Ontologicheskoe
proektirovanie intellektual’nyh sistem v oblasti istorii [Ontological
design of intelligent systems in the field of history],”  \textit{Otkrytye
semanticheskie tekhnologii proektirovaniya intellektual’nyh sistem
[Open semantic technologies for designing intelligent systems]},
pp. 245–250, 2017, (In Russ.).\\
\ \ \ \\
\ \ \\ 
\renewcommand{\labelitemi}{[6]}
\item A. Gubarevich, S. Vityaz, and R. Grigyanets, “Struktura baz znanij
v intellektual’nyh sistemah po istorii [Structure of knowledge
bases in intelligent systems on history],”  \textit{Otkrytye semanticheskie
tekhnologii proektirovaniya intellektual’nyh sistem [Open semantic
technologies for designing intelligent systems]}, pp. 347–350, 2018,
(In Russ.).
\renewcommand{\labelitemi}{[7]}
\item A. Bliskavitsky,  \textit{Konceptual’noe proektirovanie GIS i upravlenie
geoinformaciej. Tekhnologii integracii, kartograficheskogo
predstavleniya, veb-poiska i rasprostraneniya geoinformacii
[Conceptual GIS design and geoinformation management.
Technologies of integration, cartographic representation, web
search, and distribution of geoinformation]}. \textbf{LAP LAMBERT}
Academic Publishing, 2012, p. 484 (In Russ.).
\renewcommand{\labelitemi}{[8]}
\item  ——,“Semantika geoprostranstvennyh ob"ektov, funkcional’naya
grammatika i intellektual’nye GIS [Semantics of geospatial objects,
functional grammar, and intelligent GIS],” in  \textit{Izvestiya vysshih
uchebnyh zavedenij. Geologiya i razvedka [News of higher
educational institutions. Geology and exploration]}, no. 2, 2014,
pp. 62–69, (In Russ.).
\renewcommand{\labelitemi}{[9]}
\item  Y. Hu, “Geospatial semantics,”  \textit{Comprehensive Geographic Information Systems}, pp. 80–94, 2018.
\renewcommand{\labelitemi}{[10]}
\item K. Janowicz, S. Simon, T. Pehle, and G. Hart, “Geospatial
semantics and linked spatiotemporal data — past, present, and
future,”  \textit{Semantic Web}, vol. 3, pp. 321–332, 10 2012.
\renewcommand{\labelitemi}{[11]}
\item S. Samodumkin, “Next-generation intelligent geoinformation
systems,”  \textit{Otkrytye semanticheskie tekhnologii proektirovaniya intellektual’nykh system [Open semantic technologies for intelligent systems]}, 2022.
\renewcommand{\labelitemi}{[12]}
\item  \textit{Cifrovye karty mestnosti informaciya, otobrazhaemaya na
topograficheskih kartah i planah naselennyh punktov : OKRB
012-2007 [Digital maps of the area information displayed on
topographic maps and plans of settlements} : NKRB 012–2007],
Minsk, 2007, (In Russ.).  
\end{itemize}\\
\
\

\begin{center}
 \textbf{\Large{Поддержка жизненного цикла
интеллектуальных геоинформационных
систем различного назначения }}
 \end{center}
 \begin{center}
\Large{Самодумкин С.А}\\

 \end{center}
 Работа посвящена частной технологии проектирования интеллектуальных геоинформационных систем,
построенных по принципам ostis-систем. Структурная и
семантическая интероперабельность геоинформационных систем, построенных по предлагаемой технологии,
обеспечивается за счет перехода от карты к семантическому описанию элементов карты.

\ \ \ \ \ \ \ \ \ \ \ \ \ \ \ \ \ \ \ \ \ \ \ \ \ \ \ \ \ \ \ \ \ \ \ \ \ \ \ \ \ \ \ \ \ \ Received 01.03.2023
\end{multicols}
\end{document}
